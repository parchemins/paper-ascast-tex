
\section{Introduction}

There is an ongoing evlution of storing data from the cloud to the
edges of the network. Either because it is where it has been created
and it is too expensive to be transmitted through the network, or
because it has been replicated to bring data closer to the end users
~\cite{shi2016edge, fogstore, foggy_cache, cachier}.
%
To favour this transition, the design and the development of building
blocks to manage data sets across Edge infrastructures have been
investigated for the last few years \cite{confais2017performance,
  confais2017object, hasenburg2020towards}.  Although these systems,
provide good properties such as favouring network traffic confinement
by writting data locally, determining the location where to get the
content might be more expensive than retrieving the content itself.
%
Indeed, those systems, when not using a centralized index stored in a
Cloud, rely on a Distributed Hash Table (DHT) across the different
peers composing the infrastructure: When a client wants to access a
specific content, it has to perform a request to a first node to
determine from which site it will retrieve the content. Once it has
retrieved the content from the location it selected, a local replica
is created to improve the performance for future accesses and the
server in charge of maintaining the index is updated to reflect this
new location.
%
Accessing a remote node to find the location of an object (or one of
its copies) has many drawbacks: (i) The \textit{lookup penalty:} the
network latency to reach this remote node incurs an additional delay
to get the object, before being able to start its
downloading~\cite{asrese2019measuring, doan2019tracing}, (ii)
Answering external queries increases the load on the remote node,
potentially impacting its performance, (iii) It prevents nodes to
access objects in case of network partitioning: if the node in charge
of maintaining the object locations is isolated from the rest of the
infrastructure, nodes cannot access the object replicas even if some
would be reachable.

A simple way to tackle the aforementioned limitations would be to
maintain an index of all replicas on each node composing the
system. In addition to being complicated \TODO{impossible} for large-scale
systems, maintaining such a global index on each node is useless as there
is no interest to inform a node of the creation of a replica at the
opposite of the network (each node trying to get content from the closest node).
%
In this paper, we propose to limit the broadcast of creation/deletions
of replicas only to a subset of the nodes composing the
infrastructures.  Concretely, our protocol, entitled \NAME, transmits
messages from peer to peer until a certain condition is reached. In
the current scenario, the stop condition is related to the distance to
the nearest copy: if the message that a node receives indicates an
object has a longer distance than that which the node knows then the
transfer of the message is useless and so stopped.


\TODO{add something about the paper work}


\begin{figure*}
  \newcommand{\SCALE}{0.95} %% scale of sub figures
  \newcommand\X{50pt}
  \newcommand\Y{-50pt}
  
  \newcommand{\SMSG}{\tiny} %% font size of messages
  \newcommand{\OACK}{0.5} %% opacity of acknowledgement alpha messages

  \newcommand{\LEFT}{\triangleleft}
  \newcommand{\RIGHT}{\triangleright}
  
  \thickmuskip=0mu %% to remove annoying math spacing from caption
  \medmuskip=0mu
  \thinmuskip=0mu
  \begin{center}
    \subfloat[Part A][\label{fig:addA}Both $a$ and $d$
      become sources.  $w_{ab} = 2$; $w_{bc} = w_{bd} = 1$; $w_{cd} =
      3$.]  {
\begin{tikzpicture}[scale=0.87]

  \thickmuskip=0mu
  \medmuskip=0mu
  \thinmuskip=0mu
  
  \newcommand\X{50pt}
  \newcommand\Y{-50pt}

  \newcommand\ADD{\alpha}


  
  \draw (-\X + 5pt, 0) --
  node[shape=circle, draw, fill=white, inner sep=0.5pt, font=\footnotesize]{2}
  (0 - 5pt, 0); %% a - b

  \draw (0 +5pt, 0) --
  node[shape=circle, draw, fill=white, inner sep=0.5pt, font=\footnotesize]{1}
  (\X -5pt, 0); %% b - c
  
  \draw (0, 0 - 5pt) --
  node[shape=circle, draw, fill=white, inner sep=0.5pt, font=\footnotesize]{1}
  (0, \Y + 5pt); %% d - b
  
  \draw (\X + 3pt, 0 - 5pt) --
  node[shape=circle, draw, fill=white, inner sep=0.5pt, font=\footnotesize]{3}
  (0 + 5pt, \Y - 3pt); %% c - d


  
  \draw[fill=white] (-\X, 0) node[color=\PA]{$\bm{a}$} +(-5pt, -5pt) rectangle +(5pt, 5pt);  
  \draw[fill=white] (0, 0) node{$\bm{b}$} +(-5pt, -5pt) rectangle +(5pt, 5pt);
  \draw[fill=white] (\X, 0) node{$\bm{c}$} +(-5pt, -5pt) rectangle +(5pt, 5pt);
  \draw[fill=white] (0, \Y) node[color=\PD]{$\bm{d}$} +(-5pt, -5pt) rectangle +(5pt, 5pt);
  
  \draw (-\X, 5pt) node[above, font=\small, color=\PA]{$\ADD_a^0$};
  \draw (-5pt, \Y) node[left, font=\small, color=\PD]{$\ADD_d^0$};

\end{tikzpicture}
}
    \hspace{1pt}
    \subfloat[Part B][\label{fig:addB}Messages transit through %communication
      links and carry increasing weights.]{
\begin{tikzpicture}[scale=\SCALE]

  \thickmuskip=0mu
  \medmuskip=0mu
  \thinmuskip=0mu
  
  \newcommand\X{50pt}
  \newcommand\Y{-50pt}

  \newcommand\ADD{\alpha}


  
  \draw (-\X + 5pt, 0) --
  node[above=-0.3em, left=-0.3em, above left, font=\tiny]{$\textcolor{\PA}{\ADD_a^2} \rightarrow$}
  (0 - 5pt, 0); %% b - a 

  \draw (0 +5pt, 0) --
  (\X -5pt, 0); %% c - b

  \draw[opacity=0] (0, 0 - 5pt) --
  % node[opacity=1, above=-0.3em, font=\tiny, sloped]{$\textcolor{\PA}{\ADD_a^{3}} \rightarrow$}
  (0, \Y + 5pt); %% b - d
  \draw (0, \Y + 5pt) --
  node[above=-0.3em, font=\tiny, sloped]{$\textcolor{\PD}{\ADD_d^1} \rightarrow$}
  (0, 0 - 5pt);  %% d - b
  
  \draw (\X + 3pt, 0 - 5pt) --
  node[above=-0.3em, sloped, font=\tiny]{$\textcolor{\PD}{\ADD_{d}^{3}} \rightarrow$}
  (0 + 5pt, \Y - 3pt); %% c - d



  \draw[fill=white] (-\X, 0)
  node[color=\PA]{$\bm{a}$}
  +(-5pt, -5pt) rectangle +(5pt, 5pt);  
  \draw[fill=white] (0, 0) node{$\bm{b}$} +(-5pt, -5pt) rectangle +(5pt, 5pt);
  \draw[fill=white] (\X, 0) node{$\bm{c}$} +(-5pt, -5pt) rectangle +(5pt, 5pt);
  \draw[fill=white] (0, \Y) node[color=\PD]{$\bm{d}$} +(-5pt, -5pt) rectangle +(5pt, 5pt);
  
  \draw (-\X, 5pt) node[above, font=\small, color=\PA]{$\ADD_a^0$};
  \draw (-5pt, \Y) node[left, font=\small, color=\PD]{$\ADD_d^0$};

\end{tikzpicture}
}
    \hspace{1pt}
    \subfloat[Part C][\label{fig:addC}$b$ and
      $c$ receive, deliver, and forward $\alpha_{d}^{1}$ and
      $\alpha_d^3$ respectively.]{
\begin{tikzpicture}[scale=\SCALE]

  \draw (-\X + 5pt, 0) --
  node[above=-0.3em, right=-0.5em, above right, font=\SMSG]{$\textcolor{\PA}{\alpha_a^2} \RIGHT$}
  node[below=-0.3em, font=\SMSG]{$\LEFT \textcolor{\PD}{\alpha_d^3}$}
  (0 - 5pt, 0); %% b - a 

  \draw (0 +5pt, 0) --
  node[above=-0.3em, font=\SMSG]{$\LEFT \textcolor{\PD}{\alpha_d^4}$}  
  node[below=-0.3em, font=\SMSG]{$\textcolor{\PD}{\alpha_d^2} \RIGHT$}  
  (\X -5pt, 0); %% b - c

  \draw[opacity=0] (0, 0 - 5pt) --
  node[opacity=\OACK, above=-0.3em, sloped, font=\SMSG]{$\alpha_d^2 \RIGHT$}
  (0, \Y + 5pt); %% b - d
  \draw[->] (0, \Y + 5pt) --
  (0, 0 - 5pt);  %% d - b
  
  \draw[<-] (\X + 3pt, 0 - 5pt) --
  node[opacity=\OACK, below=-0.3em, sloped, font=\SMSG]{$\LEFT \alpha_d^6$}
  (0 + 5pt, \Y - 3pt); %% c - d


  
  \draw[fill=white] (-\X, 0)
  node[color=\PA]{$\bm{a}$}
  +(-5pt, -5pt) rectangle +(5pt, 5pt);  
  \draw[fill=white] (0, 0)
  node[color=\PC]{$\bm{b}$}
  +(-5pt, -5pt) rectangle +(5pt, 5pt);
  \draw[fill=white] (\X, 0)
  node[color=\PC]{$\bm{c}$}
  +(-5pt, -5pt) rectangle +(5pt, 5pt);
  \draw[fill=white] (0, \Y)
  node[color=\PC]{$\bm{d}$}
  +(-5pt, -5pt) rectangle +(5pt, 5pt);

  \draw ( 0, 5pt) node[above, font=\small, color=\PD]{$\alpha_d^1$}; % b
  \draw ( \X, 5pt) node[above, font=\small, color=\PD]{$\alpha_d^3$}; % c
  \draw (-\X, 5pt) node[above, font=\small, color=\PA]{$\alpha_a^0$}; % a
  \draw (-5pt, \Y) node[left, font=\small, color=\PD]{$\alpha_d^0$}; % d
  
\end{tikzpicture}
}
    \hspace{1pt}
    \subfloat[Part D][\label{fig:addD}$a$ and
      $b$ discarded their received messages. $c$ still improved with $\alpha_d^2$.]
    {
\begin{tikzpicture}[scale=\SCALE]

  \thickmuskip=0mu
  \medmuskip=0mu
  \thinmuskip=0mu
  
  \newcommand\X{50pt}
  \newcommand\Y{-50pt}

  \newcommand\ADD{\alpha}


  
  \draw (-\X + 5pt, 0) -- (0 - 5pt, 0); %% a - b

  \draw (0 +5pt, 0) --
  (\X -5pt, 0); %% b - c

  \draw (0, \Y + 5pt) --
  (0, 0 - 5pt);  %% d - b
  
  \draw (\X + 3pt, 0 - 5pt) --
  node[below=-0.3em, sloped, font=\tiny]{$\LEFT \textcolor{\PD}{\ADD_{d}^{5}}$}
  (0 + 5pt, \Y - 3pt); %% c - d


  
  \draw[fill=white] (-\X, 0)
  node[color=\PA]{$\bm{a}$}
  +(-5pt, -5pt) rectangle +(5pt, 5pt);  
  \draw[fill=white] (0, 0)
  node[color=\PC]{$\bm{b}$}
  +(-5pt, -5pt) rectangle +(5pt, 5pt);
  \draw[fill=white] (\X, 0)
  node[color=\PC]{$\bm{c}$}
  +(-5pt, -5pt) rectangle +(5pt, 5pt);
  \draw[fill=white] (0, \Y)
  node[color=\PC]{$\bm{d}$}
  +(-5pt, -5pt) rectangle +(5pt, 5pt);
  
  \draw ( 0, 5pt) node[above, font=\small, color=\PD]{$\ADD_d^1$}; % b
  \draw ( \X, 5pt) node[above, font=\small, color=\PD]{$\ADD_d^2$}; % c
  \draw (-\X, 5pt) node[above, font=\small, color=\PA]{$\ADD_a^0$}; % a
  \draw (-5pt, \Y) node[left, font=\small, color=\PD]{$\ADD_d^0$}; % d

  
\end{tikzpicture}
}
    \caption{\label{fig:add}Efficient consistent partitioning using
      \NAMEB. Partition~$P_a$ includes $a$ while Partition~$P_d$
      includes $b$, $c$, and $d$. \Process~$c$ and \Process~$d$ never
      acknowledge the existence of Source~$a$, for \Process~$b$ stops
      the propagation of the latter's notifications.}
  \end{center}
\end{figure*}

%%% Local Variables: 
%%% mode: latex
%%% TeX-master: "../paper"
%%% ispell-local-dictionary: "english"
%%% End: 

 %% positioning (belong to problem & motivation)

\TODO{The rest of this paper is organized as follows. Section II
  introduces background, related works and details the motivation
  behind our proposal.  Section III describes the protocol behavior,
  as well as the formal definition of the underlying
  algorithms. Section IV presents the evaluation of the A3
  implementation and provides an extensive comparison with a DHT-based
  approach. We discuss some extensions of this work and future
  directions in Section V before concluding in Section VI.}


%%% Local Variables: 
%%% mode: latex
%%% TeX-master: "../paper"
%%% ispell-local-dictionary: "english"
%%% End: 
