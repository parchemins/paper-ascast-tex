
\section{Introduction}

\TODO{With the advent of Fog infrastructures, content will be increasingly
located at the edges of the network. Either because it is where it has
been created and it is too expensive to be transmitted to a cloud
platform, or because it has been replicated to bring data closer to
the end users [1]. Thanks to the locality, caching at the edge can
benefit many diverse applications, as confirmed by recent proposals
[2], [3], [4]. To favor this transition, the design and development of
building blocks to manage data sets across Fog infrastructures are
critical. One of the key challenges of such systems resides in
locating a given content inside the network. Some systems, such as
Cassandra [5] or CEPH [6] only rely on Hash-based functions to
directly find an object location. A given hash is associated with a
specific node according to a mapping that all nodes agree with,
removing the need for any localization service. However, this implies
moving the content to a specific node in the network according to its
hash, thus violating the locality properties inherent to Fog
architectures [7]. To ensure network traffic containment and avoid
unnecessary data movements, we argue that content should be written
locally and replicated on other nodes only if solicited (e.g.,
according to the popularity of the object in a social media). Thereby,
such requirements prevent relying on hash-based solutions and calls
for a localization service [8].}

\TODO{Most existing implementations of such a localization service imply
contacting a remote node, either a centralized server [9], or a node
of a DHT [10]: When a client wants to access a specific content, first
it has to perform a request to a specific node to determine where to
retrieve this content from.  Once it has retrieved the data from the
location it selected, a local replica is created to improve the
performance for future accesses. The node in charge of maintaining the
index is finally updated to reflect this new location. Yet, accessing
a remote node to find the location of an object (or one of its copies)
has many drawbacks: (i) The lookup penalty: the network latency to
reach this remote node incurs an additional delay to get the object,
before being able to start its downloading, (ii) Answering external
queries increases the load on the remote node, potentially impacting
its performance, (iii) It prevents nodes to access objects in case of
network partitioning: if the node in charge of maintaining the object
locations is isolated from the rest of the infrastructure, nodes
cannot access the object replicas even if some would be reachable.}

\TODO{To tackle the aforementioned limitations of a remote service, we
propose A3, an epidemic protocol that propagates information about
replica localizations asynchronously. A3 leverages the underlying
infrastructure topology to avoid flooding the network so that it only
informs the relevant nodes, according to the existing replicas in the
system. Upon receiving messages, nodes maintain a personalised index
about replica localization, according to their place in the topology,
ensuring that each request will be routed towards its closest
replica. We implemented our protocol and deployed it on a dedicated
testbed emulating real-world topologies from the Internet Toplogy Zoo
[11]. Our evaluation shows that A3 is able to provide a similar
localization service when compared to a DHT-based approach while
removing the lookup penalty.  This directly translates into faster
data access as shown by our measurements, with a time reduction from
20\% to 60\%, depending on the sizes of the objects. Finally, we show
that by slightly delaying the sending of messages, A3 can match the
number of messages of a DHT-based approach, at the cost of a moderate
increase in message sizes.}

\TODO{The rest of this paper is organized as follows. Section II
  introduces background, related works and details the motivation
  behind our proposal.  Section III describes the protocol behavior,
  as well as the formal definition of the underlying
  algorithms. Section IV presents the evaluation of the A3
  implementation and provides an extensive comparison with a DHT-based
  approach. We discuss some extensions of this work and future
  directions in Section V before concluding in Section VI.}


%%% Local Variables: 
%%% mode: latex
%%% TeX-master: "../paper"
%%% End: 
