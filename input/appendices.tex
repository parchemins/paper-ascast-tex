
\appendices

\section{Table of notations}


Table~\ref{table:notations} summarizes the notations used throughout
this paper. For the sake of clarity, we divide them into three
categories: network, \process, and message.


\begin{table*}
  \centering
  \caption{\label{table:notations}Notation table.}
  \begin{tabularx}{\textwidth}{@{}lll@{}}
    \toprule
    Notation & Short & Description \\
    \midrule

    $G$ & \underline{G}raph    & Represents a network.\\
    $E$ & \underline{E}dges    & Represents the set of asynchronous communication links.\\
    $V$ & \underline{V}ertices & Represents the set of nodes, or processes.\\
    $\pi_{xz}$ and $\Pi_{xz}$  & \underline{p}ath and best \underline{P}ath & List of contiguous edges from \Process $x$ to \Process $z$.\\
    $|\pi_{xz}|$ and $|\Pi_{xz}|$ & sum of path weights & Positive sum of weights of the path.\\
    $w_{xy}$     & \underline{w}eight & Positive weight of the edge $\langle x, y \rangle$.\\
    
    \midrule

    $\sigma_x$  & \underline{s}tate     & The local state of \Process $x$.\\
    $b_x(m)$    & \underline{b}roadcast & \Process $x$ creates a new message $m$ that must be delivered by all \processes.\\
    $d_{x}(m)$  & \underline{d}eliver   & \Process $x$ delivers the message $m$.\\
    $r_y(m)$ or $r_{yx}(m)$  & \underline{r}eceive   & \Process $y$ receives the message $m$ from any neighboring node or \Process $x$ if specified.\\
    $s_{xy}(m)$ & \underline{s}end      & \Process $x$ sends the message $m$ to \Process $y$.\\
    $f_x(m)$    & \underline{f}orward   & \Process $x$ forwards the message $m$ to its neighbors.\\
    $m \oplus \sigma$   & aggregator    & Aggregates $\sigma$ into the metadata of message $m$.\\
    $\phi(\mu, \sigma)$ & predicate \underline{f}unction & Checks the metadata $\mu$ using the state $\sigma$.\\
    $\eventually P$     & eventually    & Eventually predicate $P$ is true.\\
    $e_1 \rightarrow e_2$ & happens before  & The event $e_1$ happened before the event $e_2$. Delivery, sending, or receipt are events.\\
    $D_x$ & \underline{d}elivered & Set of delivered messages by \Process $x$.\\
    
    \midrule

    $\alpha_x^d$ or $\alpha_{\pi_{xz}}^d$      & \underline{a}dd source & Message that notifies the adding of Source $x$ in the network.\\
    $\delta_x$   & source \underline{d}eletion & Message that notifies the deletion of Source $x$, or the detection by \Process $x$ of a possible source deletion.\\
    $\mathcal{S}(m)$ & \underline{S}tale message & Message $m$ conveys stale control information since its broadcaster broadcast a newer message.\\
    
    \bottomrule
  \end{tabularx}
\end{table*}



\section{Related work summary}

Table~\ref{table:relatedwork} provides a summary of state-of-the-art
approaches along with the reason they fail to enable quick content
indexing in large scale dynamic systems.

\newcommand{\rxmark}{\textcolor{\WRONG}{\xmark}}
\newcommand{\rcmark}{\textcolor{\WRONG}{\cmark}}
\newcommand{\NO}[1]{\textcolor{\WRONG}{#1}}

\begin{table}[h!]
  \scriptsize
  \centering
  \caption{\label{table:relatedwork}Related work summary.}
  \begin{tabularx}{\columnwidth}{@{}lcccc@{}}
  \toprule
  
  Approaches & \multirow{2}{5em}{\centering Third-party service} & Update & \multirow{2}{3.5em}{Eventually Consistent} & \multirow{2}{5em}{\centering\underline{R}eactive or \underline{C}yclic} \\
  \\
  \midrule

  \NO{Centralized}~\cite{snamp, p2p-oracle, fogstore, p2p-alto} & \rcmark & one & \cmark & R\\
  DHT~\cite{ipfs, mdht, squirrel}                               & \rcmark & few & \cmark & R\\

  \midrule
  
  Vector-based routing~\cite{nlsr, ospf}   & \xmark & \NO{all} & \cmark & R\\
  Replicated store~\cite{shapiro2011crdts} & \xmark & \NO{all} & \cmark & R\\
  
  \midrule
  
  Timeout-based ads~\cite{garcia-lopez, hemmati2015namebased}   & \xmark & scope & \rxmark & \NO{C}\\
  Random walks~\cite{barjon2014maintaining, sohier2012physarum} & \xmark & scope & \cmark  & \NO{C}\\
  
  \addlinespace
  
  \textbf{This paper} & \textbf{\xmark} & \textbf{scope} & \textbf{\cmark} & \textbf{R}\\
  
  \bottomrule
  \end{tabularx}  
\end{table}
