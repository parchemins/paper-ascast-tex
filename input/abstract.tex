
\begin{abstract}

\TODO{Locating the closest copy of a specific content across a
  distributed infrastructure requires to keep up-to-date an index of
  all replicas and their respective locations. Existing methods mostly
  rely on an external service to store this index, either on a
  centralized server or using a Distributed Hash Table (DHT). However,
  contacting a remote node to obtain localization information
  contradicts with emerging Fog infrastructures in terms of locality
  and robustness. In this paper, we present $A^3$, an epidemic
  protocol that will efficiently inform the relevant nodes about the
  presence of replicas in their neighborhood. Upon the exchange of
  messages, each node maintains a personalized index that will
  eventually contain the location of the closest copy for each object
  according to its place in the network topology. We implemented $A^3$
  and deployed it on a dedicated testbed with real-world
  topologies. Experimental results show that the time to retrieve
  objects is significantly decreased from 10\% up to 60\%, when
  compared to a DHT-based localization service.}

\end{abstract}


%%% Local Variables: 
%%% mode: latex
%%% TeX-master: "../paper"
%%% End: 
