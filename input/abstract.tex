\begin{abstract}
%% Context and problem
  Edge infrastructures have been proposed as an alternative to Cloud
  Computing to provide computing and storage capabilities closer to
  the locations where data is generated/consumed, with the objectives
  of reducing the latency and the volume of data passing through the
  network. A few systems have been proposed to store data in those
  infrastructures. Unfortunately, locating the closest copy of a
  specific content in these systems requires to keep up-to-date an index
  of all replicas and their respective locations. While flooding
  approaches will not enable the maintenance of such an index for each
  node in large systems, relying on a dedicated service contradicts
  with the properties of Edge infrastructures in terms of locality and
  robustness (locating sources may effectively take more time than
  actually downloading contents).

%%% Our proposal
  In this paper, we introduce \NAME, an efficient broadcast protocol
  that relies on dynamic logical partitioning of the
  infrastructure. \NAME only floods the relevant nodes (\ie the ones
  that should be informed of a the creation of a closest copy),
  enabling to keep up-to-date an index for each node eventually.  In
  addition to demonstrating formally the correctness of our protocol \TODO{Is it true?},
  we discuss several evaluatons performed by simulations and show
  \TODO{blabla}.  While \NAME tackles our location problem, it can be
  used as a building block to design new advanced services such as
  \TODO{blabla}.
  
\end{abstract}

%\TODO{source automatically create
%    partitions.} \NAME uses epidemic dissemination (\TODO{scoped
%    flooding}). \TODO{costs of operations.} When the system becomes
%  quiescent, every process eventually converges to its closest
%  partition, then \NAME incurs no overhead. \TODO{Proof, simulation,
%+    testbed?}


%%% Local Variables: 
%%% mode: latex
%%% TeX-master: "../paper"
%%% ispell-local-dictionary: "english"
%%% End: 
