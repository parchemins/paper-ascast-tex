\begin{abstract}
  %% Context and problem
  Edge infrastructures constitute an alternative to Cloud Computing
  that aims at improving on latency and traffic by providing computing
  and storage capabilities closer to data producers and consumers.
  %% Edge infrastructures have been proposed as an alternative to Cloud
  %% Computing to provide computing and storage capabilities closer to
  %% the locations where data is generated/consumed, with the objectives
  %% of reducing the latency and the volume of data passing through the
  %% network.
  A few systems, such as content delivery networks, replicate and
  distribute produced data in these infrastructures.
  % A few systems have been proposed to store data in those
  % infrastructures.
  Locating the closest replica of a specific content requires to
  maintain an up-to-date index of all live replicas along with their
  respective locations. 
  %
  Unfortunately, relying on a centralized service would bring
  scalability issues, single point of failure, as well as additional
  middleman latency where locating replicas may effectively take more
  time than actually downloading contents. Instead, data consumer
  \processes must maintain an up-to-date index
  themselves. Unfortunately, maintaining an index of all live replicas
  at every \process would prove overly costly in terms of memory and
  generated traffic, especially in large scale dynamic systems where
  \processes create and destroy replicas at any time.
  % while flooding approaches will not enable the maintenance of such
  % an index for each node in large systems, relying on a dedicated
  % service contradicts with the properties of Edge infrastructures in
  % terms of locality and robustness

%%% Our proposal
  \noindent In this paper, we introduce \NAME, an efficient and
  decentralized partitioning protocol for dynamic distributed systems:
  When a \node hosts a replica, it notifies all and only \nodes (new
  members of its partition) that are closer from it than any other
  replica; When a \node destroys a replica, it notifies members of its
  partition to switch to another closer partition. \NAME guarantees
  that eventually, every \process knows its best partition, hence its
  closest replica, despite concurrent operations and receipt orders.
  Our complexity analysis supported by simulations shows that \NAME
  scales well in terms of generated traffic and convergence time, for
  the effect of operations remains lock down to its logical
  partition. While \NAME tackles our content indexing problem, it can
  be used as a building block to design new advanced services such as
  decentralized recommendation systems.
\end{abstract}

%%% Local Variables: 
%%% mode: latex
%%% TeX-master: "../paper"
%%% ispell-local-dictionary: "english"
%%% End: 
