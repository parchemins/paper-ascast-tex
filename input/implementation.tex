
\section{Implementation and complexity analysis}
\label{sec:implementation}

\TODO{To assign and maintain \processes to their best partition according to
replica creations and removals, as well as dynamic infrastructure
changes, we designed and implemented \NAME.  \NAME stands for
\underline{A}daptive \underline{S}coped broad\underline{cast}. It
relies on a primitive that allows a \process to broadcast a message
within a limited scope. We first use this primitive to guarantee
consistent partitioning when a \process can only create a new replica
within the system. We highlight the issue when a \process can also
destroy a replica, and provide a second algorithm that handles replica
removals as well as dynamic changes of the infrastructure. Finally, we
propose \NAMEC that improves \NAME by leveraging the network topology
to further reduce generated traffic at the Edge.  This section
describes the communication primitive called scoped broadcast, then
discusses the properties that guarantee consistent partitioning, and
finally details our implementation \NAME.}



\begin{asparadesc}
\item [Dynamic consistent partitioning:]
  Algorithm~\ref{algo:ascast}
  provides the instructions of \NAME that implement three-phase purge
  and eventual forwarding of best to enable dynamic consistent
  partitioning. For the sake of simplicity, it does not handle
  multiple sessions nor optimisation such as message packing.
\end{asparadesc}

\begin{algorithm}
  
\SetInd{0.2em}{0.8em}

\newcommand{\algoAnd}{~\textbf{\textup{and}}~}
\newcommand{\algoOr}{~\textbf{\textup{or}}~}

\SetKwProg{Function}{func}{}{}

\small

\DontPrintSemicolon
\LinesNumbered

$O_p$, $W_p$ \tcp*[r]{set of neighbors and weights}
$A_{s, c}^{d, \pi} \leftarrow \alpha_{\infty, 0}^{\infty, \varnothing} $ \tcp*[r]{best $\alpha$ so far}
$V \leftarrow \varnothing$;~ $V[p] \leftarrow 0$ \tcp*[r]{vector of versions}

\BlankLine
\BlankLine

\begin{multicols}{2}
\Function{\textup{\texttt{Add ( )}}} { % \hfill [if $V[p]\equiv 0 \mod 2 $]} } {
  \textup{\smash{receiveAdd($p$, $\alpha_{p, V[p] + 1}^{0, \varnothing}$)}}\label{line:ascast_add}
}

% \BlankLine

\Function{\textup{\texttt{Del ( )}}} { % \hfill [if $V[p]\equiv 1 \mod 2 $]} } {
  \textup{\smash{receiveDel($p$, $\delta_{p, V[p] + 1}$)}}\label{line:ascast_del}
}
\end{multicols}

\BlankLine
\BlankLine
\BlankLine

\begin{multicols}{2}
\Function{\textup{receiveAdd($q$, $\alpha_{s', c'}^{d', \pi'}$)}}
  %\tcp*[f]{\footnotesize{receive $\alpha$ from $q$}}}
  {
    \uIf{\hphantom{$($}$q = \pi[|\pi| - 1] \algoAnd$\tcp*[h]{{\footnotesize{from parent?}}}\\
    \hphantom{\textbf{if}} $(c' < V[s'] \,\, \algoOr$\tcp*[h]{\footnotesize{history or state?}\label{line:ascast_version}}\\
    \hphantom{\textbf{if}} \smash{$(\alpha_{s', c'}^{d', \pi'} > A_{s, c}^{d, \pi} \algoAnd  p\not\in \pi'))$}\label{line:detectA}} {
        \textup{receiveDel($q$, $\delta_{s, c}^{\pi}$)} \tcp*[h]{\footnotesize{(II) send $\Delta$}}
    }
    \uElseIf {$A_{s, c}^{d, \pi} < \alpha_{s', c'}^{d', \pi} \algoAnd p \not\in \pi'$  } {
          $V[s'] \leftarrow c'$ \;
          $\smash{A_{s, c}^{d, \pi} \leftarrow \alpha_{s', c'}^{d', \pi'}}$ \;
          \ForEach{$n \in O_p$} {
            send$_n$(\smash{$\alpha_{s', c'}^{d' + W_{pq}, \pi' \cup p}$)} \tcp*[h]{\footnotesize{forward}\label{line:ascast_better}}
          }

     }
}



\Function{\textup{receiveDel($q$, $\delta_{s', c'}^{\pi'}$)}
%  \tcp*[f]{\footnotesize \smash{$r_p (\delta_{s', c'})$ \textup{\texttt{or}} $r_p (\delta_{s', c'}^{\pi'})$}}
} {
  \uIf(\tcp*[h]{\hspace{-0.5em}\footnotesize{$\Delta$?}}){\hphantom{$(($}$p \not\in \pi'$ \algoAnd \label{line:notloopingB}\tcp*[h]{\footnotesize{looping?}}\\
    \hphantom{\textbf{if} }$((\delta_{s', c'} \algoAnd \alpha_{s', c'-1} = A_{s, c})  \algoOr$\tcp*[h]{$\delta$?}\\
    \hphantom{\textbf{if} }\smash{$(\delta_{s',c'}^{\pi'}  \algoAnd  \alpha_{s', c'}^{\_, \pi'} = A_{s, c}^{\_, \pi}))$}} {
     $V[s'] \leftarrow \max(V[s'], c')$ \;
     $A_{s, c}^{d, \pi} \leftarrow \alpha_{\infty, 0}^{\infty, \varnothing}$ \;

    \ForEach{$n \in O_p$}{
      \lIf(\tcp*[h]{\footnotesize{(I)}}) {\smash{$\delta_{s', c'}$}} {
        send$_n$(\smash{$\delta_{s', c'}$)}
      } 
      \hphantom{\textbf{if} $\delta_{s', c'}^{\pi'}$}~\textbf{else} send$_n$(\smash{$\delta_{s', c'}^{\pi'\cup p}$)}~\tcp*[h]{\footnotesize{(III)}}
    }

  }\uElseIf{$A_{s, c}^{d, \pi} \neq \alpha_{\infty, 0}^{\infty, \varnothing}$} {
     \textup{send$_q$($\alpha_{s, c}^{d + W_{pq}, \pi \cup p}$)}
     \tcp*[h]{\label{line:ascast_compete}\footnotesize{compete again}}
  }
}

\end{multicols}

\BlankLine
\BlankLine
\BlankLine

%% \Function{\textup{edgeUp($q$)} \tcp*[f]{new link to $q$}}  {
%%     \If {$A_{s, c}^{d, \pi} \neq \alpha_{\infty, 0}^{\infty, \varnothing}$} {
%%          \textup{send$_q$($\alpha_{s, c}^{d + W_{pq}, \pi \cup p}$)}
%%          \tcp*[r]{\label{line:compete}\footnotesize{compete with $q$}}}
%% }

%% \BlankLine

%% \Function{\textup{edgeDown($q$)} \tcp*[f]{link to $q$ removed}} {
%%   \If {$q = \pi[|\pi| - 1]$} {
%%     \textup{receiveDel($q$, $\delta_{s, c}^{\pi}$)} \tcp*[r]{\footnotesize IV: $\delta?$ propagation}
%%   }
%% }

\begin{multicols}{2}
\Function{\textup{edgeUp($q$)} \tcp*[h]{new link to $q$}}  {
    \If {$A_{s, c}^{d, \pi} \neq \alpha_{\infty, 0}^{\infty, \varnothing}$} {
         \textup{send$_q$($\alpha_{s, c}^{d + W_{pq}, \pi \cup p}$)}
         \tcp*[h]{\label{line:compete}\footnotesize{compete with $q$}}}
}

% \BlankLine

\Function{\textup{edgeDown($q$)} \tcp*[h]{link to $q$ removed}} {
  \If {$q = \pi[|\pi| - 1]$} {
    \textup{receiveDel($q$, $\delta_{s, c}^{\pi}$)} \tcp*[h]{\footnotesize (III) send $\Delta$}
  }
}

\end{multicols}

\BlankLine

%%% Local Variables: 
%%% mode: latex
%%% TeX-master: "../paper"
%%% ispell-local-dictionary: "english"
%%% End: 

  \caption{\label{algo:ascast}\NAME at \Process~$p$.}
\end{algorithm}

As stated in Theorem~\ref{theo:dcp}, every \process must purge stale
messages.  To identify staleness, each \process maintains a vector of
versions that associates the respective known local counter of each
known source, or has-been source. Its size increases linearly with the
number of sources that the \process has ever known, which is the
number of \processes in the system $\mathcal{O}(V)$ in the worst case.
Nevertheless, following the principles of scoped broadcast, we expect
that every \process only acknowledges a small subset of sources.  In
addition, each message carries the identifier and counter of each
\node that forwarded it. Again, following the principles of scoped
broadcast, we expect that versioned paths remain small. Using these
data structures, each \process delivers and forwards only new messages
that improve their best known partition (see
Line~\ref{line:ascast_improve}).

This vector of versions constitutes a summary of known progress of
other \nodes. It discards stale messages and it enables detecting
possible inconsistencies in message delivery: the current best
partition comes from a neighbor that just delivered a stale message
(see Line~\ref{line:ascast_detect}). Figure~\ref{fig:problem} shows
that such a behavior may hide a delete notification. The detecting
\process must act to reflect this possible change: all \processes
whose last message comes from the detecting \process must remove it
and accept the competition of other \processes. Since this behavior is
similar to that of source deletion, we implement it in a similar way.

A \process that removes its self-appointed status of source, or
detects a possible inconsistency, increments its local counter and
disseminates a delete notification message $\delta_{s, c}$
piggybacking its identifier $s$ and counter $c$. Any \process that
receives such delete notification delivers and forwards it if its best
message contains the has-been source or detector associated to a lower
counter (see Line~\ref{line:ascast_delete}). This allows $\delta$
messages to quickly reach every downstream \process.

As stated in Theorem~\ref{theo:dcp}, dynamic consistent partitioning
not only requires eventual purging of stale messages, but also the
retrieval of the best up-to-date partitions.  In that regard, $\delta$
messages have a dual use: since they already reach the borders of
their partition by design, they trigger a competition when reaching
bordering.  When a \process receives a delete notification from a
neighbor $q$ but does not deliver it, it assumes that the just
received $\delta$ message purged possibly stale messages, creating a
gap that its own best partition could help fill. Consequently, it
sends to $q$ its own best $\alpha$ message (see
Line~\ref{line:ascast_echo}).

\begin{algorithm}
  \SetKwProg{Function}{func}{}{}

\SetInd{0.2em}{1em}

\small

\DontPrintSemicolon
\LinesNumbered

% \begin{multicols}{2}
\Function(\tcp*[f]{new link to $q$}){\textup{edgeUp($q$)}}  {
    \lIf {$A_{\pi}^{d} \neq \alpha_\varnothing^\infty$} {
         send$_q$($\alpha_{\pi}^{d + W_{pq}}$)
         }
}

% \BlankLine

\Function(\tcp*[f]{link to $q$ removed}){\textup{edgeDown($q$)}} {
  \lIf {\textup{isParent($q$)}} {
       receiveDel($q$, $\delta_{p, V[p]+1}$)
  }
}

% \end{multicols}

% \BlankLine

  \caption{\label{algo:edges}\NAME at \Process~$p$ in dynamic networks.}
\end{algorithm}

\begin{asparadesc}
\item [Dynamic network:] Algorithm~\ref{algo:edges} enables dynamic
  consistent partitioning in dynamic networks where \processes can
  join, leave, or crash at any time. Removing a \process is equivalent
  to removing all its edges, and adding a \process is equivalent to
  add as many edges as necessary.  Adding an edge between two
  \processes may only improve the shortest path of one of these
  \processes. Therefore, it triggers a competition between the two
  \processes. If one or the other \process improves, the propagation
  of $\alpha$ messages operates normally.  Removing an edge between
  two \processes may invalidate the shortest path of one of involved
  \processes plus downstream \processes. As a side effect, removing an
  edge may also impair the propagation of a partition delete. To
  implement this behavior, \NAME uses its implementation of $\delta$
  messages. This enables \NAME even in dynamic networks subject to
  physical partitioning.
\end{asparadesc}

%% \NAME guarantees dynamic consistent partitioning by making extensive
%% use of \process-to-\process communications. It requires
%% \begin{inparaenum}[(i)]
%% \item purging stale control information, hence propagating $\delta$
%%   messages; and
%% \item retrieving the closest source, hence propagating $\alpha$
%%   messages.
%% \end{inparaenum}
In terms of number of messages, in the average case, a \process $i$
chosen uniformly at random among all \processes creates a logical
partition. Its messages $\alpha_i$ spread through the network until
reaching \processes that belong to another partition closer to
them. This splits partitions in half on average. Overall, the $a^{th}$
new partition comprises \smash{$\mathcal{O}(\frac{|V|}{2^{\lfloor
      \log_2 a \rfloor}})$} \processes. This decreases every new
partition until reaching one \process per new partition: its
source. The average number of messages per \process is
\smash{$\mathcal{O}(\frac{\overline{|O|}}{2^{\lfloor \log_2 a
      \rfloor}})$}, where \smash{$\overline{|O|}$} is the average
number of neighbors per \process.
% \TODO{Multiple receipt and
% multiple
%  delivery imply more messages (receipt bounded by $|O|$ as well).}
Deleting the $a^{th}$ partition generates twice as many messages as
creating the $a^{th}$ partition: $\delta$ messages travel through the
partition, then $\alpha$ messages compete to fill the gap.  Overall,
the communication complexity shows that \NAME scales well to the
number of partitions.
%% In the
%% worst-case, every new partition includes all but one \process
%% belonging to the previous partition. The total number of messages
%% after the $a^{th}$ new partition is $\mathcal{O}(\overline{|O|}\cdot
%% a^2)$. As for the average-case, the number of messages for the
%% partition deletion is identical to the number of messages of the
%% corresponding partition creation.
%% \begin{algorithm}[h!]
%%   \SetKwProg{Function}{func}{}{}

\SetInd{0.2em}{1em}

\small

\DontPrintSemicolon
\LinesNumbered

% \begin{multicols}{2}
\Function(\tcp*[f]{new link to $q$}){\textup{edgeUp($q$)}}  {
    \lIf {$A_{\pi}^{d} \neq \alpha_\varnothing^\infty$} {
         send$_q$($\alpha_{\pi}^{d + W_{pq}}$)
         }
}

% \BlankLine

\Function(\tcp*[f]{link to $q$ removed}){\textup{edgeDown($q$)}} {
  \lIf {\textup{isParent($q$)}} {
       receiveDel($q$, $\delta_{p, V[p]+1}$)
  }
}

% \end{multicols}

% \BlankLine

%%   \caption{\label{algo:edges}\NAME at \Process~$p$ in dynamic networks.}
%% \end{algorithm}


\begin{asparadesc}
\item [Lazy partitioning:]
  To enable lazy partitioning in networks of networks, we designed and
  implemented \NAMEC (stands for both \underline{cross}
  \underline{a}utonomous \underline{s}ystems and \underline{ex}tended
  \underline{a}daptive \underline{s}coped broad\underline{cast}). The
  main idea behind \NAMEC is that every node maintains two \NAME:
  \begin{inparaenum}[(i)]
\item the first maintaining the closest source intra-AS and
  acknowledges the interests of its AS;
\item the second maintaining the closest source cross-AS that also
  competes against the local closest source.
\end{inparaenum}
\NAMEC enables nodes to discard irrelevant indexes while maintaining
the closest source between connected ASes. 

\end{asparadesc}


\begin{algorithm}
  
\SetInd{0.2em}{0.8em}

\newcommand{\algoAnd}{~\textbf{\textup{and}}~}
\newcommand{\algoOr}{~\textbf{\textup{or}}~}

\SetKwProg{Function}{func}{}{}

\small

\DontPrintSemicolon
\LinesNumbered


\newcommand{\XADD}{\Gamma}
\newcommand{\xadd}{\gamma}

$X_p$ \tcp*[r]{neighbors in a different network}
$\XADD_{\pi}^{d} \leftarrow \gamma_{\varnothing}^{\infty} $
\tcp*[r]{best $\gamma$ so far (\underline{g}lobal $\alpha$)}

\BlankLine



% \begin{multicols}{2}

\Function{\textup{receiveAdd($q$, $\alpha_{\pi'}^{d'}$)}} {
  \If {$q \in X_p \algoAnd \min(A, \XADD)\neq \alpha_\varnothing^\infty$} {
    send$_q$($\min(A, \XADD) \oplus ^{w_{pq}}$)\;
    receiveXAdd(\smash{$q, \xadd_{\pi'}^{d'}$})
  }
  \ElseIf {\smash{$\textup{isParent}(q, \TODO{\XADD}) \algoAnd \textup{isStale}(\alpha_{\pi'}^{d'})$}}
  { receiveDel($q$, $\delta_{p, V[p] +1}$) }
  \lElse {\smash{\textsc{ascast}.receiveAdd($q$, $\alpha_{\pi'}^{d'}$)}}
  \lIf {$A<\XADD \algoOr A = \alpha_\varnothing^\infty$} {$\XADD_\pi^d \leftarrow \alpha_\varnothing^\infty$}
}

\BlankLine



\Function{\textup{receiveXAdd($q$, $\xadd_{\pi'}^{d'}$)}} {
  \If {$\xadd_{\pi'}^{d'} < A \algoAnd \xadd_{\pi}^{d'} < \XADD \algoAnd A_\pi^d \neq \alpha_\varnothing^\infty
    \algoAnd$ $\neg \textup{isStale}(\xadd_{\pi'}^{d'}) \algoAnd \neg \textup{isLoop}(\xadd_{\pi'}^{d'})$} {
    $\XADD \leftarrow \xadd_{\pi'}^{d'} \cup _{\langle p, V[p] \rangle}$ \;
    
    \lForEach {$n \in O_p$}{send$_n$($\XADD_{\pi}^d \oplus ^{w_{pq}}$)}

  } \ElseIf {$\textup{isStale}(\xadd_{\pi'}^{d'}) \algoAnd \textup{isParent}(q, \XADD)$} {
    receiveDel($q$, $\delta_{p, V[p] + 1}$)
  }
  updateVersions($\pi'$)
}

\BlankLine



\Function{\textup{\TODO{receiveDel}($q$, $\delta_{s, c}$)}} {
  \uIf {$\exists \langle s, c' \rangle \in \pi: c' < c$\label{line:ascast_delete}} {
    $\smash{A^d_\pi \leftarrow \alpha^\infty_\varnothing}$ \;
    \lForEach {$n \in O_p \setminus q$}{send$_n$($\delta_{s, c}$)}
  }
  \lElseIf {$A_s^d \neq \alpha^\infty_\varnothing$} {
    send$_q$($\alpha_\pi^{d + w_{pq}}$)\label{line:ascast_echo}
  }

  updateVersions($[\langle s, c \rangle]$)
}

% \end{multicols}

\BlankLine

\footnotesize\lFunction{\textup{fromX($q$, $\alpha_{\pi'}^{d'}$)}} {
  %% ugly if then else but w/e
  \textbf{if} $q \in X_p$ \textbf{then} {\Return x\smash{$\alpha_{\pi'}^{d'}$} \textbf{else}
    \Return \smash{$\alpha_{\pi'}^{d'}$}}
}

%%% Local Variables: 
%%% mode: latex
%%% TeX-master: "../paper"
%%% ispell-local-dictionary: "english"
%%% End: 

  \caption{\label{algo:xascast}\NAMEC at \Process~$p$. \TODO{TODO.}}
\end{algorithm}

Next Section provides the details of our simulations that assess the
proposed implementations.


%%% Local Variables: 
%%% mode: latex
%%% TeX-master: "../paper"
%%% ispell-local-dictionary: "english"
%%% End: 
