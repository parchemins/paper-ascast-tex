
\section{\NAMEC}
\label{sec:proposal2}

\NAME enables indexing the closest replica of each \process without
flooding the whole network when a new replica appears or disappears.
Nevertheless, every node must partake in the indexing of each
content. When the networks accounts for billions of content, this
raises scalability issues. Inspired by the Internet topology that
comprises tens of thousands of interconnected autonomous systems
serving billions of people~\cite{nur2018crossas}, we want to further
lock-down traffic by leveraging the edges of networks as barriers, and
enable on-demand indexing.

To enable cross-network indexing, we designed and implemented \NAMEC
(stands for both \underline{cross} \underline{a}utonomous
\underline{s}ystems and \underline{ex}tended \underline{a}daptive
\underline{s}coped broad\underline{cast}). The main idea behind \NAMEC
is that every node maintains two \NAME:
\begin{inparaenum}[(i)]
\item the first maintaining the closest source intra-AS and
  acknowledges the interests of its AS;
\item the second maintaining the closest source cross-AS that also
  competes against the local closest source.
\end{inparaenum}
\NAMEC enables nodes to discard irrelevant indexes while maintaining
the closest source between connected ASes. \TODO{This section
  describes our proposal.}


\TODO{Added invariant: global is always smaller than local when they
  exist.}


%%% Local Variables: 
%%% mode: latex
%%% TeX-master: "../paper"
%%% ispell-local-dictionary: "english"
%%% End: 
