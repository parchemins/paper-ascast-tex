
\section{Adaptive scoped broadcast}
\label{sec:adaptive}

To provide lazy and dynamic logical partitioning in dynamic networks
using scoped broadcast, all live \processes must collaborate to
disseminate messages that notify new or removed sources to all and
only interested \processes. This section reviews step-by-step the
properties that allow \processes to converge to the desired state
together. It first defines scoped broadcast, then uses it to guarantee
consistent partitioning when a \process can only become a new source
in the system. It highlights the issue when a \process can also remove
its status of source. It shows that using local knowledge and scoped
broadcast, \processes can still reach dynamic consistent partitioning
when they are able to detect possible blocking conditions in the
dissemination of required notifications. Finally, it further constrain
the system by allowing only subsets of \processes to reach lazy
consistent partitioning. \TODO{It shows that it only needs what?}


\subsection{Scoped broadcast}
\label{subsec:scoped}

In this paper, we consider Edge infrastructures as a set of
interconnected autonomous systems comprising heterogeneous \nodes
interconnected by communication links. \Processes involved in the
management of content may crash but are not byzantine.  \Processes can
reliably communicate through asynchronous message passing to other
known \processes called neighbors.  We define scoped broadcast as a
communication primitive that propagates a message around its
broadcaster within an application-dependant scope.

\begin{definition}[Autonomous systems]
  An autonomous system is a network comprising nodes and communication
  links that we represent as a \underline{g}raph of
  \underline{v}ertices and \underline{e}dges: $G = \langle V, E
  \rangle$ with $E \in V \times V$. A \underline{p}ath $\pi_{ij}$ from
  \Process~$i$ to \Process~$j$ is a sequence of vertices $[i, k_1,
    k_2, \ldots k_n, j]$ with $\forall m: 0\leq m \leq n, \langle
  \pi_{ij}[m], \pi_{ij}[m+1] \rangle \in E$.
\end{definition}

%% \begin{definition}[Edge infrastructure]
%%   An Edge infrastructure is a connected \underline{g}raph $G(V, E)$ of
%%   \underline{v}ertices $V$ and bidirectional \underline{e}dges $E
%%   \subseteq V \times V$.  A \underline{p}ath $\pi_{ij}$ from
%%   \Process~$i$ to \Process~$j$ is a sequence of vertices $[i, k_1,
%%     k_2, \ldots k_n, j]$ with $\forall m: 0\leq m \leq n, \langle
%%   \pi_{ij}[m], \pi_{ij}[m+1] \rangle \in E$.
%% \end{definition}

\begin{definition}[\label{def:scoped}\underline{S}coped broad\underline{cast} (\NAMEB)]
  When \Process~$x$ scoped \underline{b}roadcasts $b_x(m)$ a
  \underline{m}essage $m$, every correct \process $y$ within a scope
  \underline{r}eceives $r_y(m)$ and \underline{d}elivers it
  $d_y(m)$. The scope depends on the \underline{s}tate $\sigma$ of
  each \process, the \underline{m}etadata $\mu$ piggybacked by each
  message, and a \underline{p}redicate $\phi$ verified from \process to
  \process: $b_x(m) \wedge r_y(m) \implies \exists \pi_{xy}: \forall z
  \in \pi_{xy}, \phi(\mu_z, \sigma_z)$.
\end{definition}

This definition encompasses more specific definitions of related
work~\cite{hsiao2005scoped, lue2006scoped, wang2015prodiluvian}.  It
also highlights that epidemic propagation and scoped broadcast have
well-aligned preoccupations. More precisely, it underlines the
transitive relevance of messages, thus a \process can stop forwarding
messages as soon as its predicate becomes unverified.
%For instance, a \process
%from Paris could scoped broadcast messages to all \processes in
%Paris. This requires \processes to store and maintain their city
%location in their local state. \Processes stop delivering and
%forwarding their received messages when they come from a different
%city.  In another instance, a \process from Paris could scoped
%broadcast messages to all \processes in Paris plus neighboring
%cities. This requires \processes to overload forwarded messages the
%first time they reach another city. The predicate checks if messages
%already reached two distinct cities before delivery. Similarly to
%uniform reliable broadcast, scoped broadcast implementations expose
%different trade-offs on space, time, and communication.
%
We use \NAMEB to efficiently modify the state of each \process
depending on the partitions that exist in the system.



\subsection{Consistent partitioning}
\label{subsec:consistent}

At any time, a \process can decide to become a \emph{source}, hence
creating a new partition in the system by executing an \texttt{Add}
operation. This partition includes at least its source plus
neighboring \processes that estimate they are closer to this source
than any other one. Such a distance (or \emph{weight}) is
application-dependant: in the context of maintaining distributed
indexes, this would be about link latency that \nodes could monitor by
aggregating \texttt{ping}s; or operational costs when dealing with
multiple tenants.

\begin{definition}[\label{def:partitioning}Partitioning]
  Let $S \subseteq V$ be the set of \underline{s}ources, and $P_{s\in
    S}$ be the \underline{p}artition including at least \Process~$s$,
  each \process belongs to at most one partition $\forall p,q \in V,
  \forall s,s' \in S: (p \in P_{s} \wedge q \in P_{s'}) \implies (p \neq
  q \vee s = s')$, and there exists at least one path $\pi_{ps}$ of
  \processes that belong to this partition $\forall q \in \pi_{ps}: q
  \in P_s$.
\end{definition}

Definition~\ref{def:scoped} and Definition~\ref{def:partitioning}
share the transitive relevance of \process states. However, we further
constrain the partitioning in order to guarantee the existence of
exactly one consistent partitioning that \processes eventually converge
to.

\begin{definition}[\underline{C}onsistent \underline{p}artitioning (CP)]
  Let $W_{xy}$ be the positive \underline{w}eight from $x$ to $y$,
  $\Pi_{xz}$ be the shortest \underline{p}ath from $x$ to $z$ the
  weight of which $|\Pi_{xz}|$ is lower than any other path weight,
  with $|\Pi_{xx}|$ being the greatest lower bound of $x$, the only
  consistent partitioning $\mathcal{P}$ is a set of partitions
  $P_{s\in S}$ such that each \process belongs to a logical partition
  comprising its closest source: $\forall p \in P_{s}: \nexists
  P_{s'}$ such that $|\Pi_{s'p}| < |\Pi_{sp}|$.
\end{definition}

Unfortunately, \processes do not share a common global knowledge of
the network state. For \processes to reach consistent partitioning
together, a \Process $s$ \underline{a}dding a partition must send a
notification $\alpha_s$ to every \process that is closer to it than
any other source. Since epidemic dissemination and scoped broadcast
have well-aligned objectives, we assume implementations relying on
message forwarding from neighbor-to-neighbor.

\begin{definition}[\label{def:forwarding}Forwarding]
  A \process $p$ \underline{f}orwards a message $m$ by
  \underline{s}ending it to its neighbors: $f_p(m) \implies \forall
  \langle p, q\rangle \in E: \TODO{s_{pq}(m)}$.
\end{definition}

\begin{theorem}[\label{theo:efb}\underline{F}orwarding of \underline{B}est (FB)
  $\implies\eventually$CP] Assuming reliable communication links
  ($s_{pq}(m) \iff \eventually r_{qp}(m)$), \processes eventually
  reach consistent partitioning if each \process delivers the message
  notifying its shortest path weight to a source among its set of
  \underline{r}eceived messages $R_p$ ($(m = \min R_p) \implies
  d_p(m)$), and forwards the message notifying its shortest path
  weight to a source $s$ among its set of \underline{d}elivered
  messages $D_p$ ($(m = \min D_p) \implies f_p(m)$).
\end{theorem}


\begin{figure*}
  \newcommand{\SCALE}{0.95} %% scale of sub figures
  \newcommand\X{50pt}
  \newcommand\Y{-50pt}
  
  \newcommand{\SMSG}{\tiny} %% font size of messages
  \newcommand{\OACK}{0.5} %% opacity of acknowledgement alpha messages

  \newcommand{\LEFT}{\triangleleft}
  \newcommand{\RIGHT}{\triangleright}
  
  \thickmuskip=0mu %% to remove annoying math spacing from caption
  \medmuskip=0mu
  \thinmuskip=0mu
  \begin{center}
    \subfloat[Part A][\label{fig:addA}Both $a$ and $d$
      become sources.  $w_{ab} = 2$; $w_{bc} = w_{bd} = 1$; $w_{cd} =
      3$.]  {
\begin{tikzpicture}[scale=0.87]

  \thickmuskip=0mu
  \medmuskip=0mu
  \thinmuskip=0mu
  
  \newcommand\X{50pt}
  \newcommand\Y{-50pt}

  \newcommand\ADD{\alpha}


  
  \draw (-\X + 5pt, 0) --
  node[shape=circle, draw, fill=white, inner sep=0.5pt, font=\footnotesize]{2}
  (0 - 5pt, 0); %% a - b

  \draw (0 +5pt, 0) --
  node[shape=circle, draw, fill=white, inner sep=0.5pt, font=\footnotesize]{1}
  (\X -5pt, 0); %% b - c
  
  \draw (0, 0 - 5pt) --
  node[shape=circle, draw, fill=white, inner sep=0.5pt, font=\footnotesize]{1}
  (0, \Y + 5pt); %% d - b
  
  \draw (\X + 3pt, 0 - 5pt) --
  node[shape=circle, draw, fill=white, inner sep=0.5pt, font=\footnotesize]{3}
  (0 + 5pt, \Y - 3pt); %% c - d


  
  \draw[fill=white] (-\X, 0) node[color=\PA]{$\bm{a}$} +(-5pt, -5pt) rectangle +(5pt, 5pt);  
  \draw[fill=white] (0, 0) node{$\bm{b}$} +(-5pt, -5pt) rectangle +(5pt, 5pt);
  \draw[fill=white] (\X, 0) node{$\bm{c}$} +(-5pt, -5pt) rectangle +(5pt, 5pt);
  \draw[fill=white] (0, \Y) node[color=\PD]{$\bm{d}$} +(-5pt, -5pt) rectangle +(5pt, 5pt);
  
  \draw (-\X, 5pt) node[above, font=\small, color=\PA]{$\ADD_a^0$};
  \draw (-5pt, \Y) node[left, font=\small, color=\PD]{$\ADD_d^0$};

\end{tikzpicture}
}
    \hspace{1pt}
    \subfloat[Part B][\label{fig:addB}Messages transit through %communication
      links and carry increasing weights.]{
\begin{tikzpicture}[scale=\SCALE]

  \thickmuskip=0mu
  \medmuskip=0mu
  \thinmuskip=0mu
  
  \newcommand\X{50pt}
  \newcommand\Y{-50pt}

  \newcommand\ADD{\alpha}


  
  \draw (-\X + 5pt, 0) --
  node[above=-0.3em, left=-0.3em, above left, font=\tiny]{$\textcolor{\PA}{\ADD_a^2} \rightarrow$}
  (0 - 5pt, 0); %% b - a 

  \draw (0 +5pt, 0) --
  (\X -5pt, 0); %% c - b

  \draw[opacity=0] (0, 0 - 5pt) --
  % node[opacity=1, above=-0.3em, font=\tiny, sloped]{$\textcolor{\PA}{\ADD_a^{3}} \rightarrow$}
  (0, \Y + 5pt); %% b - d
  \draw (0, \Y + 5pt) --
  node[above=-0.3em, font=\tiny, sloped]{$\textcolor{\PD}{\ADD_d^1} \rightarrow$}
  (0, 0 - 5pt);  %% d - b
  
  \draw (\X + 3pt, 0 - 5pt) --
  node[above=-0.3em, sloped, font=\tiny]{$\textcolor{\PD}{\ADD_{d}^{3}} \rightarrow$}
  (0 + 5pt, \Y - 3pt); %% c - d



  \draw[fill=white] (-\X, 0)
  node[color=\PA]{$\bm{a}$}
  +(-5pt, -5pt) rectangle +(5pt, 5pt);  
  \draw[fill=white] (0, 0) node{$\bm{b}$} +(-5pt, -5pt) rectangle +(5pt, 5pt);
  \draw[fill=white] (\X, 0) node{$\bm{c}$} +(-5pt, -5pt) rectangle +(5pt, 5pt);
  \draw[fill=white] (0, \Y) node[color=\PD]{$\bm{d}$} +(-5pt, -5pt) rectangle +(5pt, 5pt);
  
  \draw (-\X, 5pt) node[above, font=\small, color=\PA]{$\ADD_a^0$};
  \draw (-5pt, \Y) node[left, font=\small, color=\PD]{$\ADD_d^0$};

\end{tikzpicture}
}
    \hspace{1pt}
    \subfloat[Part C][\label{fig:addC}$b$ and
      $c$ receive, deliver, and forward $\alpha_{d}^{1}$ and
      $\alpha_d^3$ respectively.]{
\begin{tikzpicture}[scale=\SCALE]

  \draw (-\X + 5pt, 0) --
  node[above=-0.3em, right=-0.5em, above right, font=\SMSG]{$\textcolor{\PA}{\alpha_a^2} \RIGHT$}
  node[below=-0.3em, font=\SMSG]{$\LEFT \textcolor{\PD}{\alpha_d^3}$}
  (0 - 5pt, 0); %% b - a 

  \draw (0 +5pt, 0) --
  node[above=-0.3em, font=\SMSG]{$\LEFT \textcolor{\PD}{\alpha_d^4}$}  
  node[below=-0.3em, font=\SMSG]{$\textcolor{\PD}{\alpha_d^2} \RIGHT$}  
  (\X -5pt, 0); %% b - c

  \draw[opacity=0] (0, 0 - 5pt) --
  node[opacity=\OACK, above=-0.3em, sloped, font=\SMSG]{$\alpha_d^2 \RIGHT$}
  (0, \Y + 5pt); %% b - d
  \draw[->] (0, \Y + 5pt) --
  (0, 0 - 5pt);  %% d - b
  
  \draw[<-] (\X + 3pt, 0 - 5pt) --
  node[opacity=\OACK, below=-0.3em, sloped, font=\SMSG]{$\LEFT \alpha_d^6$}
  (0 + 5pt, \Y - 3pt); %% c - d


  
  \draw[fill=white] (-\X, 0)
  node[color=\PA]{$\bm{a}$}
  +(-5pt, -5pt) rectangle +(5pt, 5pt);  
  \draw[fill=white] (0, 0)
  node[color=\PC]{$\bm{b}$}
  +(-5pt, -5pt) rectangle +(5pt, 5pt);
  \draw[fill=white] (\X, 0)
  node[color=\PC]{$\bm{c}$}
  +(-5pt, -5pt) rectangle +(5pt, 5pt);
  \draw[fill=white] (0, \Y)
  node[color=\PC]{$\bm{d}$}
  +(-5pt, -5pt) rectangle +(5pt, 5pt);

  \draw ( 0, 5pt) node[above, font=\small, color=\PD]{$\alpha_d^1$}; % b
  \draw ( \X, 5pt) node[above, font=\small, color=\PD]{$\alpha_d^3$}; % c
  \draw (-\X, 5pt) node[above, font=\small, color=\PA]{$\alpha_a^0$}; % a
  \draw (-5pt, \Y) node[left, font=\small, color=\PD]{$\alpha_d^0$}; % d
  
\end{tikzpicture}
}
    \hspace{1pt}
    \subfloat[Part D][\label{fig:addD}$a$ and
      $b$ discarded their received messages. $c$ still improved with $\alpha_d^2$.]
    {
\begin{tikzpicture}[scale=\SCALE]

  \thickmuskip=0mu
  \medmuskip=0mu
  \thinmuskip=0mu
  
  \newcommand\X{50pt}
  \newcommand\Y{-50pt}

  \newcommand\ADD{\alpha}


  
  \draw (-\X + 5pt, 0) -- (0 - 5pt, 0); %% a - b

  \draw (0 +5pt, 0) --
  (\X -5pt, 0); %% b - c

  \draw (0, \Y + 5pt) --
  (0, 0 - 5pt);  %% d - b
  
  \draw (\X + 3pt, 0 - 5pt) --
  node[below=-0.3em, sloped, font=\tiny]{$\LEFT \textcolor{\PD}{\ADD_{d}^{5}}$}
  (0 + 5pt, \Y - 3pt); %% c - d


  
  \draw[fill=white] (-\X, 0)
  node[color=\PA]{$\bm{a}$}
  +(-5pt, -5pt) rectangle +(5pt, 5pt);  
  \draw[fill=white] (0, 0)
  node[color=\PC]{$\bm{b}$}
  +(-5pt, -5pt) rectangle +(5pt, 5pt);
  \draw[fill=white] (\X, 0)
  node[color=\PC]{$\bm{c}$}
  +(-5pt, -5pt) rectangle +(5pt, 5pt);
  \draw[fill=white] (0, \Y)
  node[color=\PC]{$\bm{d}$}
  +(-5pt, -5pt) rectangle +(5pt, 5pt);
  
  \draw ( 0, 5pt) node[above, font=\small, color=\PD]{$\ADD_d^1$}; % b
  \draw ( \X, 5pt) node[above, font=\small, color=\PD]{$\ADD_d^2$}; % c
  \draw (-\X, 5pt) node[above, font=\small, color=\PA]{$\ADD_a^0$}; % a
  \draw (-5pt, \Y) node[left, font=\small, color=\PD]{$\ADD_d^0$}; % d

  
\end{tikzpicture}
}
    \caption{\label{fig:add}Efficient consistent partitioning using
      \NAMEB. Partition~$P_a$ includes $a$ while Partition~$P_d$
      includes $b$, $c$, and $d$. \Process~$c$ and \Process~$d$ never
      acknowledge the existence of Source~$a$, for \Process~$b$ stops
      the propagation of the latter's notifications.}
  \end{center}
\end{figure*}

%%% Local Variables: 
%%% mode: latex
%%% TeX-master: "../paper"
%%% ispell-local-dictionary: "english"
%%% End: 



\begin{proof}
  When there are no sources, no \process receives, nor delivers hence
  forwards any message. \Processes do not belong to any partition.
  
  \noindent Whenever a \process $s$ becomes a source, it delivers its
  own message $d_p(\alpha_{s})$. Whatever its set of received
  messages, its own partition $P_s$ becomes and remains its closest
  partition $\alpha_{s} = \min D_p$ since $\forall p \neq s: |\Pi_{s
    s}| < |\Pi_{s p}|$.

  \noindent Such a source $s$ forwards $\alpha_{s}$ to its neighbors.
  Since communication links are reliable, neighboring \processes
  eventually receive such a notification $\forall \langle s, q \rangle
  \in E \iff \eventually r_q(\alpha_{s})$. Most importantly,
  whatever the order of received messages, every \process $q$ in this
  neighborhood such that there exists no closer source $s'$ than the
  received one $s$, delivers it, since $|\Pi_{sq}| < |\Pi_{s'q}|$.

  \noindent They forward it and their respective neighbors eventually
  receive it. By transitivity, the message originating from $s$
  reaches all \processes belonging to $P_s$ at least through their
  shortest paths: $\forall q' \in V, s, s' \in S: |\Pi_{s q'}| <
  |\Pi_{s' q'}| \implies \eventually d_{q'}(\alpha_{s})$.  Since there
  exists only one best sum of weights per \process that can never be
  retracted, \processes eventually reach consistent partitioning.
  %% /!\ not equivalence for there exists other implementations.
  %% \item [$CP \implies BEF$:] By contradiction, if a \process $q \in
  %%   \Pi_{sp} = [s, \ldots, q, q', \ldots, p]$ with $s, q, p \in P_s$
  %%   does not forward its received $\alpha_s^{|\Pi_{sq}|}$, then
  %%   following \processes from $q'$ to $p$ may mistake another
  %%   partition for their  because it needs the weight $W_{pq}$.
  %% \end{asparadesc}
\end{proof}

\begin{algorithm}
  \SetKwProg{Function}{func}{}{}

\small

\DontPrintSemicolon
\LinesNumbered

$O_p$, $W_p$\tcp*[r]{set of neighbors and weights}
$s \leftarrow \varnothing$ \tcp*[r]{best source of partition ($\sigma$)}
$d \leftarrow \infty$ \tcp*[r]{smallest distance to $s$ ($\sigma$ and $\mu$)}


\BlankLine

\Function{\textup{Add ( )} \tcp*[f]{$\alpha_p^0$}} {
  \textup{receiveAdd($\varnothing$, $p$, $0$)} \label{line:lowestbound} \tcp*[f]{$b_p(\alpha_p^0)$}
}

\BlankLine

\Function{\textup{receiveAdd($q$, $s'$, $d'$)} \tcp*[f]{$r_p(\alpha_{s'}^{d'})$ from $q$}} {
  \If (\tcp*[f]{($\phi$)}){$d' < d$} {
      $s \leftarrow s'$ \tcp*[r]{\smash{$d_p(\alpha_{s'}^{d'})$}}
      $d \leftarrow d'$ \;

      \ForEach(\tcp*[f]{\smash{$f_p(\alpha_{s'}^{d'})$}}) {$n \in O_p \setminus q$} {
          \textup{send$_n$($s', d' + W_{pn}$)} \label{line:accumulator}
          \tcp*[r]{\smash{$s_{pn}(\alpha_{s'}^{d'+w_{pn}})$}}
       }      
  }
}

%% \BlankLine

%% \Function{\textup{edgeUp($q$)} \tcp*[f]{new link to $q$}} {
%%   \lIf { $d < \infty$} {\textup{send$_q$($s, d + W_{pq}$)}}
%% }



  \caption{\label{algo:add}Adding a partition by \Process~$p$.}
\end{algorithm}

Algorithm~\ref{algo:add} shows the instructions that implement
forwarding of best to reach consistent partitioning when
\begin{inparaenum}[(i)]
\item weights are scalar values,
\item \processes only add new partitions to the system,
\item and \processes never crash nor leave the system.
\end{inparaenum} \TODO{Disambiguator and comparison with $\alpha < A$.}
\TODO{Why not towards parent?}
Figure~\ref{fig:add} illustrates its behavior on a system comprising 4
\processes $a$, $b$, $c$, and $d$. \Process~$a$ and \Process~$d$
become the sources of their partition. They \NAMEB a notification
\underline{a}dd message: $\alpha_a^0$ and $\alpha_d^0$. They
initialize their own state with the lowest possible bound $0$ (see
Line~\ref{line:lowestbound}), and send a message to each of their
neighbors by accumulating the corresponding edge weight (see
Line~\ref{line:accumulator}). In Figure~\ref{fig:addC}, \Process~$b$
receives $\alpha_{d}^{1}$. Since it improves its own partition
distance, it keeps it and forwards it to its neighbors. In
Figure~\ref{fig:addD}, \Process~$b$ discards $\alpha_{a}^{2}$, for it
does not improve its partition distance. \Processes $c$ and $d$ will
never acknowledge that Source~$a$ exists. Ultimately, \processes
discard last transiting messages. Despite the obvious lack of traffic
optimization, the system reaches consistent partitioning.

While only adding logical partitions to the distributed system is
straightforward, removing them introduces additional complexity caused
by concurrent operations.

\subsection{Dynamic consistent partitioning}
\label{subsec:dynamic}

%% At any time, a \process can become a source, hence adding a new
%% partition to the system. This partition eventually includes all
%% \processes that are closer from this new source than any other
%% else. \Processes naturally converge towards their respective best
%% partition by only piggybacking a monotonically increasing distance in
%% forwarded messages.


\begin{figure}[t]
  \newcommand{\SCALE}{0.8}

  \newcommand\X{38pt}
  \newcommand\Y{-40pt}

  \thickmuskip=0mu
  \medmuskip=0mu
  \thinmuskip=0mu

  \newcommand\ADD{\alpha}
  \newcommand\DEL{\delta}
    
  \newcommand{\LEFT}{\triangleleft}
  \newcommand{\RIGHT}{\triangleright}
  
  \begin{center}
    \subfloat[Part A][\label{fig:delA}$a$ deletes its partition. It
      notifies all \processes that belong to its
      partition.]{
\begin{tikzpicture}[scale=\SCALE]

  \draw[opacity=0](-2.45*\X, 0) -- (2.45*\X, 0); %% more space for caption
  

  
  \draw (-\X + 5pt, 0) --
  node[above=-0.3em,font=\tiny]{$\DEL_{a} \RIGHT$}
  (0 - 5pt, 0); %% b - a 

  \draw (0 +5pt, 0) --
  (\X -5pt, 0); %% b - c

  \draw [<-] (0, 0 - 5pt) --
  (0, \Y + 5pt);  %% b - d
  
  \draw [<-] (\X + 3pt, 0 - 5pt) --
  (0 + 5pt, \Y - 3pt); %% c - d


  
  \draw[fill=white] (-\X, 0) node{$\bm{a}$} +(-5pt, -5pt) rectangle +(5pt, 5pt);  
  \draw[fill=white] (0, 0) node[color=\PD]{$\bm{b}$} +(-5pt, -5pt) rectangle +(5pt, 5pt);
  \draw[fill=white] (\X, 0) node[color=\PD]{$\bm{c}$} +(-5pt, -5pt) rectangle +(5pt, 5pt);
  \draw[fill=white] (0, \Y) node[color=\PD]{$\bm{d}$} +(-5pt, -5pt) rectangle +(5pt, 5pt);
  
  \draw (-\X, 5pt) node[above, font=\small]{$\DEL_a$};
  \draw (0, 5pt) node[above, font=\small, color=\PD]{$\ADD_d^1$};
  \draw (\X, 5pt) node[above, font=\small, color=\PD]{$\ADD_d^2$};
  \draw (-5pt, \Y) node[left, font=\small, color=\PD]{$\ADD_d^0$};


\end{tikzpicture}
}
    \hspace{3pt}
    \subfloat[Part B][\label{fig:delB}$\delta$ stops as soon as it encounters
      another partition. $b$ answers with its partition.]{
\begin{tikzpicture}[scale=0.87]

  \thickmuskip=0mu
  \medmuskip=0mu
  \thinmuskip=0mu
  
  \newcommand\X{50pt}
  \newcommand\Y{-50pt}

  \newcommand\ADD{\alpha}
  \newcommand\DEL{\delta}



  \draw (-\X + 5pt, 0) --
  node[above=-0.3em,font=\tiny]{$\DEL_{a} \rightarrow$}
  node[below=-0.3em,font=\tiny]{$\leftarrow \ADD_{a}^{3}$}
  (0 - 5pt, 0); %% b - a 

  \draw (0 +5pt, 0) --
  node[above=-0.3em, font=\tiny]{$\leftarrow \textcolor{\PC}{\ADD_{c}^{1}}\cdot\DEL{c}$}
  node[below=-0.3em, font=\tiny]{$\textcolor{\PA}{\ADD_a^{2.5}} \rightarrow$}
  (\X -5pt, 0); %% b - c

  \draw[opacity=0] (0, 0 - 5pt) --
  node[opacity=1, above=-0.3em, font=\tiny, sloped]{$\ADD_a^{2.5} \rightarrow$}
  (0, \Y + 5pt); %% b - d
  \draw (0, \Y + 5pt) --
  node[above=-0.3em, font=\tiny, sloped]{$\ADD_c^2 \rightarrow$}
  (0, 0 - 5pt);  %% d - b
  
  \draw (\X + 3pt, 0 - 5pt) --
  node[above=-0.3em, font=\tiny, sloped]{$\ADD_{c}^{2} \rightarrow$}
  node[below=-0.3em, font=\tiny, sloped]{$\leftarrow \DEL_c$}
  (0 + 5pt, \Y - 3pt); %% c - d



  
  \draw[fill=white] (-\X, 0) node{$\bm{a}$} +(-5pt, -5pt) rectangle +(5pt, 5pt);  
  \draw[fill=white] (0, 0) node[color=\PA]{$\bm{b}$} +(-5pt, -5pt) rectangle +(5pt, 5pt);
  \draw[fill=white] (\X, 0) node{$\bm{c}$} +(-5pt, -5pt) rectangle +(5pt, 5pt);
  \draw[fill=white] (0, \Y) node[color=\PC]{$\bm{d}$} +(-5pt, -5pt) rectangle +(5pt, 5pt);
  
  % \draw (-\X+5pt, 5pt) node[above left]{$\DEL_a$};
  % \draw (\X+5pt, 5pt) node[above right]{$\DEL_c$};
  \draw (0, 5pt) node[above]{$\bm{a: 1.5}$};
  \draw (-5pt, \Y) node[left]{$\bm{c: 1}$};


\end{tikzpicture}
}
  \end{center}
  \caption{\label{fig:del}Efficient removal of a partition using
    scoped broadast. $a$ eventually acknowledges that it belongs to $P_d$ with $\alpha_d^3$ echoing.}
\end{figure}

%%% Local Variables: 
%%% mode: latex
%%% TeX-master: "../paper"
%%% ispell-local-dictionary: "english"
%%% End: 



At any time, a source can revoke its status of source by executing a
\texttt{Del} operation, hence deleting its partition from the
system. All \processes that belong to this partition must eventually
choose another partition to belong to.


\begin{figure*}[t]
  \newcommand{\SCALE}{0.8}

  \newcommand{\SMSG}{\tiny}
  \newcommand{\OACK}{0.5}
  
  \thickmuskip=0mu
  \medmuskip=0mu
  \thinmuskip=0mu

  
  \newcommand\X{41pt}
  \newcommand\Y{-40pt}
 
  \newcommand{\LEFT}{\triangleleft}
  \newcommand{\RIGHT}{\triangleright}
  
  \begin{center}
    \subfloat[Part A][\label{fig:problemA}Both $a$ and $c$ become sources.
      $w_{ab} = 2$; $w_{bc} = 1$.]{
\begin{tikzpicture}[scale=\SCALE]

  \draw [opacity=0] (-1.5*\X, 0) -- (1.5*\X, 0); %% spacing
  
  \draw (-\X + 5pt, 0) --
  node[above=-0.3em, font=\SMSG]{$\textcolor{\PA}{\alpha_{a}^{2}} \RIGHT$}
  (0 - 5pt, 0); %% b - a 

  \draw (0 +5pt, 0) --
  node[below=-0.3em,font=\SMSG]{~ ~ $\LEFT \textcolor{\PC}{\alpha_{c}^{1}}$}
  (\X -5pt, 0); %% b - c


  
  \draw[fill=white] (-\X, 0) node{\textcolor{\PA}{$\bm{a}$}} +(-5pt, -5pt) rectangle +(5pt, 5pt);  
  \draw[fill=white] (0, 0) node{$\bm{b}$} +(-5pt, -5pt) rectangle +(5pt, 5pt);
  \draw[fill=white] (\X, 0) node{\textcolor{\PC}{$\bm{c}$}} +(-5pt, -5pt) rectangle +(5pt, 5pt);
  
  \draw (-\X, -6pt) node[below, font=\SNODE]{$\textcolor{\PA}{\alpha_a^0}$};
  \draw ( \X, -6pt) node[below, font=\SNODE]{$\textcolor{\PC}{\alpha_c^0}\vphantom{\alpha_a^3}$};
  
\end{tikzpicture}
}
    \hspace{5pt}
    \subfloat[Part B][\label{fig:problemB}Both $a$ and $c$ delete their partition 
     while $b$ delivers and forwards $\alpha_a$.]
             {

\begin{tikzpicture}[scale=\SCALE]

  \thickmuskip=0mu
  \medmuskip=0mu
  \thinmuskip=0mu
  
  \newcommand\X{50pt}
  \newcommand\Y{-50pt}

  \newcommand\ADD{\alpha}
  \newcommand\DEL{\delta}


  
  \draw (-\X + 5pt, 0) --
  node[above=-0.3em,font=\tiny]{$\DEL_{a} \RIGHT$}
  (0 - 5pt, 0); %% b - a 

  \draw (0 +5pt, 0) --
  node[above=-0.3em,font=\tiny]{$\textcolor{\PA}{\ADD_{a}^{3}} \RIGHT$}
  node[below=-0.3em,font=\tiny]{$\LEFT \textcolor{\PC}{\ADD_{c}^{1}} \cdot \DEL_c$}
  (\X -5pt, 0); %% b - c


  
  \draw[fill=white] (-\X, 0) node{$\bm{a}$} +(-5pt, -5pt) rectangle +(5pt, 5pt);  
  \draw[fill=white] (0, 0) node{\textcolor{\PA}{$\bm{b}$}} +(-5pt, -5pt) rectangle +(5pt, 5pt);
  \draw[fill=white] (\X, 0) node{$\bm{c}$} +(-5pt, -5pt) rectangle +(5pt, 5pt);
  
  \draw (-\X, 5pt) node[above, font=\scriptsize]{$\textcolor{\PA}{\ADD_a}\rightarrow \DEL_a$};
  \draw (  0, 5pt) node[above, font=\scriptsize]{$\textcolor{\PA}{\ADD_a^2}$};
  \draw ( \X, 5pt) node[above, font=\scriptsize]{$\textcolor{\PC}{\ADD_c}\vphantom{\ADD_a^3} \rightarrow \DEL_c$};

  \begin{scope}[shift={(0, -1*\Y)}]
      \draw (-\X + 5pt, 0) --
      (0 - 5pt, 0); %% b - a 
      
      \draw (0 +5pt, 0) --
      node[above=-0.3em,font=\tiny]{$\textcolor{\PA}{\ADD_{a}^{3}} \RIGHT$}
      node[below=-0.3em,font=\tiny]{$\LEFT \textcolor{\PC}{\ADD_{c}^{1}} \cdot \DEL_c$}
      (\X -5pt, 0); %% b - c
      
      \draw[fill=white] (-\X, 0) node{\textcolor{\PA}{$\bm{a}$}} +(-5pt, -5pt) rectangle +(5pt, 5pt);  
      \draw[fill=white] (0, 0) node{\textcolor{\PA}{$\bm{b}$}} +(-5pt, -5pt) rectangle +(5pt, 5pt);
      \draw[fill=white] (\X, 0) node{$\bm{c}$} +(-5pt, -5pt) rectangle +(5pt, 5pt);
      
      \draw (-\X, 5pt) node[above, font=\scriptsize]{$\textcolor{\PA}{\ADD_a^0}$};
      \draw (  0, 5pt) node[above, font=\scriptsize]{$\textcolor{\PA}{\ADD_a^2}$};
      \draw ( \X, 5pt) node[above, font=\scriptsize]{$\textcolor{\PC}{\ADD_c}\vphantom{\ADD_a^3} \rightarrow \DEL_c$};

  \end{scope}

\end{tikzpicture}
}
    \hspace{5pt}
    \subfloat[Part C][\label{fig:problemC}$b$ blocks the only transiting
      $\delta_a$ while $b$ delivers and forwards $\alpha_c$.]
             {
\begin{tikzpicture}[scale=\SCALE]

  \thickmuskip=0mu
  \medmuskip=0mu
  \thinmuskip=0mu
  
  \newcommand\X{50pt}
  \newcommand\Y{-50pt}

  \newcommand\ADD{\alpha}
  \newcommand\DEL{\delta}


  
  \draw (-\X + 5pt, 0) --
  node[above=-0.3em, font=\tiny]{~ ~ ~ ~ ~ $\DEL_{a} \RIGHT$} %% b - a
  node[above=-0.3em, font=\tiny]{~ ~ ~ ~ ~ $\textcolor{\WRONG}{\text{\normalsize\xmark}} \hphantom{\RIGHT}$} %% b - a
  node[below=-0.3em, font=\tiny]{$\LEFT \textcolor{\PC}{\ADD_c^3}$} %% b - a
  (0 - 5pt, 0);

  \draw (0 +5pt, 0) --
  node[below=-0.3em, font=\tiny]{$\LEFT \DEL_c \vphantom{\ADD^1}$}
  (\X -5pt, 0); %% b - c


  
  \draw[fill=white] (-\X, 0) node{$\bm{a}$} +(-5pt, -5pt) rectangle +(5pt, 5pt);  
  \draw[fill=white] (0, 0) node{\textcolor{\PC}{$\bm{b}$}} +(-5pt, -5pt) rectangle +(5pt, 5pt);
  \draw[fill=white] (\X, 0) node{\textcolor{\PA}{$\bm{c}$}} +(-5pt, -5pt) rectangle +(5pt, 5pt);
  
  \draw (-\X, 5pt) node[above, font=\scriptsize]{$\textcolor{\PA}{\ADD_a}\rightarrow \DEL_a$};
  \draw (  0, 5pt) node[above, font=\scriptsize]{$\textcolor{\PA}{\ADD_a^2} \rightarrow \textcolor{\PC}{\ADD_c^1}$};
  \draw ( \X, 5pt) node[above, font=\scriptsize]{$\textcolor{\PC}{\ADD_c}\vphantom{\ADD_a^3} \rightarrow \DEL_c \rightarrow \textcolor{\PA}{\ADD_a^3}$};


  \begin{scope}[shift={(0, -1*\Y)}]
    \draw (-\X + 5pt, 0) --
    node[below=-0.3em, font=\tiny]{$\LEFT \textcolor{\PC}{\ADD_c^3}$} %% b - a
    (0 - 5pt, 0);
    
    \draw (0 +5pt, 0) --
    node[below=-0.3em, font=\tiny]{$\LEFT \DEL_c \vphantom{\ADD^1}$}
    (\X -5pt, 0); %% b - c
    
    \draw[fill=white] (-\X, 0) node{\textcolor{\PA}{$\bm{a}$}} +(-5pt, -5pt) rectangle +(5pt, 5pt);  
    \draw[fill=white] (0, 0) node{\textcolor{\PC}{$\bm{b}$}} +(-5pt, -5pt) rectangle +(5pt, 5pt);
    \draw[fill=white] (\X, 0) node{\textcolor{\PA}{$\bm{c}$}} +(-5pt, -5pt) rectangle +(5pt, 5pt);
    
    \draw (-\X, 5pt) node[above, font=\scriptsize]{$\textcolor{\PA}{\ADD_a^0}$};
    \draw (  0, 5pt) node[above, font=\scriptsize]{$\textcolor{\PA}{\ADD_a^2} \rightarrow \textcolor{\PC}{\ADD_c^1}$};
    \draw ( \X, 5pt) node[above, font=\scriptsize]{$\textcolor{\PC}{\ADD_c}\vphantom{\ADD_a^3} \rightarrow \DEL_c \rightarrow \textcolor{\PA}{\ADD_a^3}$};
    
  \end{scope}

\end{tikzpicture}
}
    \hspace{5pt}
    \subfloat[Part D][\label{fig:problemD}$\alpha_c^3$ reaches $a$ that delivers it.]
             {
\begin{tikzpicture}[scale=\SCALE]
  
  \draw (-\X + 5pt, 0) --
  (0 - 5pt, 0);

  \draw [->] (0 +5pt, 0) --
  node[below=-0.3em, font=\tiny]{$\vphantom{\alpha^1_c }$}
  (\X -5pt, 0); %% b - c


  
  \draw[fill=white] (-\X, 0) node{$\bm{a}$} +(-5pt, -5pt) rectangle +(5pt, 5pt);  
  \draw[fill=white] (0, 0) node{$\bm{b}$} +(-5pt, -5pt) rectangle +(5pt, 5pt);
  \draw[color=\WRONG, fill=white] (\X, 0) node{\textcolor{\PA}{$\bm{c}$}} +(-5pt, -5pt) rectangle +(5pt, 5pt);
  
  \draw (-\X, 5pt) node[above, font=\scriptsize]{$\ldots \rightarrow \textcolor{\PC}{\alpha_c} \rightarrow \delta_c$};
  \draw (  0, 5pt) node[above, font=\scriptsize]{$\ldots \rightarrow \delta_c$};
  \draw ( \X, 5pt) node[above, font=\scriptsize]{$\ldots \rightarrow \textcolor{\PA}{\alpha_a^3}$};


  
  %% \begin{scope}[shift={(0, -1*\Y)}]

  %%   \draw (-\X + 5pt, 0) --
  %%   (0 - 5pt, 0);
    
  %%   \draw (0 +5pt, 0) --
  %%   node[below=-0.3em, font=\tiny]{$\vphantom{\alpha^1_c }$}
  %%   (\X -5pt, 0); %% b - c

    
  %%   \draw[fill=white] (-\X, 0) node{\textcolor{\PA}{$\bm{a}$}} +(-5pt, -5pt) rectangle +(5pt, 5pt);  
  %%   \draw[fill=white] (0, 0) node{\textcolor{\PA}{$\bm{b}$}} +(-5pt, -5pt) rectangle +(5pt, 5pt);
  %%   \draw[fill=white] (\X, 0) node{\textcolor{\PA}{$\bm{c}$}} +(-5pt, -5pt) rectangle +(5pt, 5pt);
    
  %%   \draw (-\X, 5pt) node[above, font=\scriptsize]{$\textcolor{\PA}{\alpha_a^0}$};
  %%   \draw (  0, 5pt) node[above, font=\scriptsize]{$\ldots \rightarrow \delta_c \rightarrow \textcolor{\PA}{\alpha_a^2}$};
  %%   \draw ( \X, 5pt) node[above, font=\scriptsize]{$\ldots \rightarrow \textcolor{\PA}{\alpha_a^3}$};
  %% \end{scope}
  
\end{tikzpicture}
}
    \hspace{5pt}
    \subfloat[Part E][\label{fig:problemE}$\delta_c$ reaches $a$ that delivers it.]
             {
\begin{tikzpicture}[scale=\SCALE]

  \draw [opacity=0] (-1.5*\X, 0) -- (1.5*\X, 0); %% spacing
  
  \draw (-\X + 5pt, 0) -- (0 - 5pt, 0);

  \draw [->] (0 +5pt, 0) --
  node[opacity=\OACK, above=-0.3em, font=\SMSG]{~ ~ ~$\alpha_c^2 \RIGHT$}
  node[opacity=\OACK, below=-0.3em, font=\SMSG]{$\LEFT \alpha_a^4$ ~ ~ ~}
  (\X -5pt, 0); %% b - c


  
  \draw[fill=white] (-\X, 0) node{$\bm{a}$} +(-5pt, -5pt) rectangle +(5pt, 5pt);  
  \draw[fill=white] (0, 0) node{$\bm{b}$} +(-5pt, -5pt) rectangle +(5pt, 5pt);
  \draw[fill=white] (\X, 0) node{\textcolor{\PA}{$\bm{c}$}} +(-5pt, -5pt) rectangle +(5pt, 5pt);
  
  \draw (-\X, -6pt) node[below, font=\SNODE]{$\ldots \rightarrow \delta_c\vphantom{\alpha_a^3}$};
  \draw (  0,  6pt) node[above, font=\SNODE]{$\ldots \rightarrow \delta_c$};
  \draw ( \X, -6pt) node[below, font=\SNODE]{$\textcolor{\PC}{\alpha_c}\vphantom{\alpha_a^3} \rightarrow \delta_c \rightarrow \textcolor{\PA}{\alpha_a^3}$};

\end{tikzpicture}
}
    \hspace{5pt}
    \subfloat[Part F][\label{fig:problemF}$c$ stays forever in a stale partition: $P_a$.]
             {
\begin{tikzpicture}[scale=\SCALE]

  \draw [opacity=0] (-1.5*\X, 0) -- (1.5*\X, 0);
  
  \draw (-\X + 5pt, 0) --
  (0 - 5pt, 0);

  \draw [->] (0 +5pt, 0) --
  node[below=-0.3em, font=\tiny]{$\vphantom{\alpha^1_c }$}
  (\X -5pt, 0); %% b - c


  
  \draw[fill=white] (-\X, 0) node{$\bm{a}$} +(-5pt, -5pt) rectangle +(5pt, 5pt);  
  \draw[fill=white] (0, 0) node{$\bm{b}$} +(-5pt, -5pt) rectangle +(5pt, 5pt);
  \draw[color=\WRONG, fill=white] (\X, 0) node{\textcolor{\PA}{$\bm{c}$}} +(-5pt, -5pt) rectangle +(5pt, 5pt);
  
  \draw (-\X, 5pt) node[above, font=\scriptsize]{$\ldots \rightarrow \delta_c$};
  \draw (  0, 5pt) node[above, font=\scriptsize]{$\ldots \rightarrow \delta_c$};
  \draw ( \X, 5pt) node[above, font=\scriptsize]{$\ldots \rightarrow \textcolor{\PA}{\alpha_a^3}$};


  
  %% \begin{scope}[shift={(0, -1*\Y)}]

  %%   \draw (-\X + 5pt, 0) --
  %%   (0 - 5pt, 0);
    
  %%   \draw (0 +5pt, 0) --
  %%   node[below=-0.3em, font=\tiny]{$\vphantom{\alpha^1_c }$}
  %%   (\X -5pt, 0); %% b - c

    
  %%   \draw[fill=white] (-\X, 0) node{\textcolor{\PA}{$\bm{a}$}} +(-5pt, -5pt) rectangle +(5pt, 5pt);  
  %%   \draw[fill=white] (0, 0) node{\textcolor{\PA}{$\bm{b}$}} +(-5pt, -5pt) rectangle +(5pt, 5pt);
  %%   \draw[fill=white] (\X, 0) node{\textcolor{\PA}{$\bm{c}$}} +(-5pt, -5pt) rectangle +(5pt, 5pt);
    
  %%   \draw (-\X, 5pt) node[above, font=\scriptsize]{$\textcolor{\PA}{\alpha_a^0}$};
  %%   \draw (  0, 5pt) node[above, font=\scriptsize]{$\ldots \rightarrow \delta_c \rightarrow \textcolor{\PA}{\alpha_a^2}$};
  %%   \draw ( \X, 5pt) node[above, font=\scriptsize]{$\ldots \rightarrow \textcolor{\PA}{\alpha_a^3}$};
  %% \end{scope}
  
\end{tikzpicture}
}
  \end{center}
  \caption{\label{fig:problem}Even in the simplest scenarios, the
    naive propagation of $\alpha$ and $\delta$ messages may be
    insufficient to guarantee consistent partitioning. If $c$ had
    children, they would stay in the wrong partition too.}
\end{figure*}



A naive implementation resembles echos in acoustics:
\underline{d}elete notifications $\delta$ propagate as long as
receivers' current partition is the targeted one, and when delete
notifications reach the bordering \processes of the deleted partition,
it triggers a competition -- or an echo -- that goes backward to fill
the gap left open by removals.  For example, in Figure~\ref{fig:del},
two partitions initially exist: $P_a$ and $P_d$ that respectively
include $\{a\}$, and $\{b, c, d\}$. In Figure~\ref{fig:delA},
\Process~$a$ deletes its partition. It notifies all neighboring
\processes~--~here only \Process~$b$~-- that may belong to its
partition using \NAMEB. Upon receipt, \Process~$b$ discards the delete
notification $\delta_a$ since $\delta_a$ does not target the former's
partition $P_d$. \Process~$b$ sends its own best partition
$\alpha_d^3$ that may be the best for \Process~$a$. Eventually, every
\process belongs to Partition $P_d$. In this scenario, they reach
consistent partitioning.

Delete operations add a new notion of order between events, and most
importantly between message deliveries. Delete operations implicitly
state that all preceding events become obsolete, and that all messages
originating from these preceding events convey stale control
information.



\begin{figure*}[t]
  \newcommand{\SCALE}{0.8}

  \thickmuskip=0mu
  \medmuskip=0mu
  \thinmuskip=0mu

  \newcommand\X{42pt}
  \newcommand\Y{-40pt}

  \newcommand{\OACK}{0.5}
  \newcommand{\SMSG}{\tiny}
  
  \newcommand{\LEFT}{\triangleleft}
  \newcommand{\RIGHT}{\triangleright}
  
  \begin{center}
    \subfloat[Part A][\label{fig:undoproblemA}$a$ and $c$ become sources.
      $w_{ab} = 2$; $w_{bc} = 1$.]{
\begin{tikzpicture}[scale=\SCALE]

  %% \draw (-\X, 5pt) node[above, font=\scriptsize]{$\textcolor{\PA}{\alpha_a^0}$};
  %% \draw ( \X, 5pt) node[above, font=\scriptsize]{$\textcolor{\PC}{\alpha_c^0}\vphantom{\alpha_a^3}$};

  \begin{scope}[shift={(0, -1*\Y)}]

      \draw [opacity=0] (-1.5*\X, 0) -- (1.5*\X, 0); %% spacing
    
      \draw (-\X + 5pt, 0) --
      node[above=-0.3em,font=\SMSG]{$\textcolor{\PA}{\alpha_{a}^{2}} \RIGHT$}
      (0 - 5pt, 0); %% b - a 
      
      \draw (0 +5pt, 0) --
      node[below=-0.3em,font=\SMSG]{~ ~ ~ ~ ~$\LEFT \textcolor{\PC}{\alpha_{c}^{1}}$}
      (\X -5pt, 0); %% b - c
      
      \draw[fill=white] (-\X, 0) node{\textcolor{\PA}{$\bm{a}$}} +(-5pt, -5pt) rectangle +(5pt, 5pt);  
      \draw[fill=white] (0, 0) node{$\bm{b}$} +(-5pt, -5pt) rectangle +(5pt, 5pt);
      \draw[fill=white] (\X, 0) node{\textcolor{\PC}{$\bm{c}$}} +(-5pt, -5pt) rectangle +(5pt, 5pt);
      
      \draw (-\X, 5pt) node[above, font=\scriptsize]{$\textcolor{\PA}{\alpha_a^0}$};
      \draw ( \X, 5pt) node[above, font=\scriptsize]{$\textcolor{\PC}{\alpha_c^0}\vphantom{\alpha_a^3}$};
    
  \end{scope}
  
\end{tikzpicture}
}
    \hspace{5pt}
    \subfloat[Part B][\label{fig:undoproblemB}$c$ deletes its partition but $a$ does not.
      $b$ delivers and forwards $\alpha_a$.]
             {

\begin{tikzpicture}[scale=\SCALE]

  \thickmuskip=0mu
  \medmuskip=0mu
  \thinmuskip=0mu
  
  \newcommand\X{50pt}
  \newcommand\Y{-50pt}

  \newcommand\ADD{\alpha}
  \newcommand\DEL{\delta}


  
%%   \draw (-\X + 5pt, 0) --
%%   node[above=-0.3em,font=\tiny]{$\bm{\DEL_{a} \RIGHT}$}
%%   (0 - 5pt, 0); %% b - a 

%%   \draw (0 +5pt, 0) --
%%   node[above=-0.3em,font=\tiny]{$\textcolor{\PA}{\ADD_{a}^{3}} \RIGHT$}
%%   node[below=-0.3em,font=\tiny]{$\LEFT \textcolor{\PC}{\ADD_{c}^{1}} \cdot \DEL_c$}
%%   (\X -5pt, 0); %% b - c

%% 
  
%%   \draw[fill=white] (-\X, 0) node{$\bm{a}$} +(-5pt, -5pt) rectangle +(5pt, 5pt);  
%%   \draw[fill=white] (0, 0) node{\textcolor{\PA}{$\bm{b}$}} +(-5pt, -5pt) rectangle +(5pt, 5pt);
%%   \draw[fill=white] (\X, 0) node{$\bm{c}$} +(-5pt, -5pt) rectangle +(5pt, 5pt);
  
%%   \draw (-\X, 5pt) node[above, font=\scriptsize]{$\textcolor{\PA}{\ADD_a} \bm{\rightarrow \DEL_a}$};
%%   \draw (  0, 5pt) node[above, font=\scriptsize]{$\textcolor{\PA}{\ADD_a^2}$};
%%   \draw ( \X, 5pt) node[above, font=\scriptsize]{$\textcolor{\PC}{\ADD_c}\vphantom{\ADD_a^3} \rightarrow \DEL_c$};

  \begin{scope}[shift={(0, -1*\Y)}]
      \draw (-\X + 5pt, 0) --
      (0 - 5pt, 0); %% b - a 
      
      \draw (0 +5pt, 0) --
      node[above=-0.3em,font=\tiny]{$\textcolor{\PA}{\ADD_{a}^{3}} \RIGHT$}
      node[below=-0.3em,font=\tiny]{$\LEFT \textcolor{\PC}{\ADD_{c}^{1}} \cdot \DEL_c$}
      (\X -5pt, 0); %% b - c
      
      \draw[fill=white] (-\X, 0) node{\textcolor{\PA}{$\bm{a}$}} +(-5pt, -5pt) rectangle +(5pt, 5pt);  
      \draw[fill=white] (0, 0) node{\textcolor{\PA}{$\bm{b}$}} +(-5pt, -5pt) rectangle +(5pt, 5pt);
      \draw[fill=white] (\X, 0) node{$\bm{c}$} +(-5pt, -5pt) rectangle +(5pt, 5pt);
      
      \draw (-\X, 5pt) node[above, font=\scriptsize]{$\textcolor{\PA}{\ADD_a^0}$};
      \draw (  0, 5pt) node[above, font=\scriptsize]{$\textcolor{\PA}{\ADD_a^2}$};
      \draw ( \X, 5pt) node[above, font=\scriptsize]{$\textcolor{\PC}{\ADD_c}\vphantom{\ADD_a^3} \rightarrow \DEL_c$};

  \end{scope}

\end{tikzpicture}
}
    \hspace{5pt}
    \subfloat[Part C][\label{fig:undoproblemC}$b$ delivers + forwards $\alpha_c$.]
             {
\begin{tikzpicture}[scale=\SCALE]

    \draw [opacity=0] (-1.5*\X, 0) -- (1.5*\X, 0); %% spacing
    
    \draw (-\X + 5pt, 0) --
    node[below=-0.3em, font=\SMSG]{$\LEFT \textcolor{\PC}{\alpha_c^3}$} %% b - a
    (0 - 5pt, 0);
    
    \draw [<->] (0 +5pt, 0) --
    node[opacity=\OACK, above=-0.3em, font=\SMSG]{$\alpha_c^2 \RIGHT$~ ~ ~}
    node[below=-0.3em, font=\SMSG]{$\LEFT \delta_c \vphantom{\alpha^1}$~ ~ ~ }
    node[opacity=\OACK, below=-0.3em, font=\SMSG]{~ ~ ~ $\LEFT \alpha_a^4$}
    (\X -5pt, 0); %% b - c
    
    \draw[fill=white] (-\X, 0) node{\textcolor{\PA}{$\bm{a}$}} +(-5pt, -5pt) rectangle +(5pt, 5pt);  
    \draw[fill=white] (0, 0) node{\textcolor{\PC}{$\bm{b}$}} +(-5pt, -5pt) rectangle +(5pt, 5pt);
    \draw[fill=white] (\X, 0) node{\textcolor{\PA}{$\bm{c}$}} +(-5pt, -5pt) rectangle +(5pt, 5pt);
    
    \draw (-\X, -6pt) node[below, font=\SNODE]{$\textcolor{\PA}{\alpha_a^0}$};
    \draw (  0,  6pt) node[above, font=\SNODE]{$\textcolor{\PA}{\alpha_a^2} \rightarrow \textcolor{\PC}{\alpha_c^1}$};
    \draw ( \X, -6pt) node[below, font=\SNODE]{$\textcolor{\PC}{\alpha_c}\vphantom{\alpha_a^3} \rightarrow \delta_c \rightarrow \textcolor{\PA}{\alpha_a^3}$};
    

\end{tikzpicture}
}
    \hspace{5pt}
    \subfloat[Part D][\label{fig:undoproblemD}$b$ delivers + forwards $\delta_c$.]
             {
\begin{tikzpicture}[scale=\SCALE]

  \thickmuskip=0mu
  \medmuskip=0mu
  \thinmuskip=0mu
  
  \newcommand\X{50pt}
  \newcommand\Y{-50pt}

  \newcommand\ADD{\alpha}
  \newcommand\DEL{\delta}


  
%%   \draw (-\X + 5pt, 0) --
%%   (0 - 5pt, 0);

%%   \draw (0 +5pt, 0) --
%%   node[below=-0.3em, font=\tiny]{$\vphantom{\ADD^1_c }$}
%%   (\X -5pt, 0); %% b - c

%% 
  
%%   \draw[fill=white] (-\X, 0) node{$\bm{a}$} +(-5pt, -5pt) rectangle +(5pt, 5pt);  
%%   \draw[fill=white] (0, 0) node{$\bm{b}$} +(-5pt, -5pt) rectangle +(5pt, 5pt);
%%   \draw[color=\WRONG, fill=white] (\X, 0) node{\textcolor{\PA}{$\bm{c}$}} +(-5pt, -5pt) rectangle +(5pt, 5pt);
  
%%   \draw (-\X, 5pt) node[above, font=\scriptsize]{$\ldots \rightarrow \textcolor{\PC}{\ADD_c} \rightarrow \DEL_c$};
%%   \draw (  0, 5pt) node[above, font=\scriptsize]{$\ldots \rightarrow \DEL_c$};
%%   \draw ( \X, 5pt) node[above, font=\scriptsize]{$\ldots \rightarrow \textcolor{\PA}{\ADD_a^3}$};


  
  \begin{scope}[shift={(0, -1*\Y)}]

    \draw (-\X + 5pt, 0) --
    (0 - 5pt, 0);
    
    \draw (0 +5pt, 0) --
    node[below=-0.3em, font=\tiny]{$\vphantom{\ADD^1_c }$}
    (\X -5pt, 0); %% b - c

    
    \draw[fill=white] (-\X, 0) node{\textcolor{\PA}{$\bm{a}$}} +(-5pt, -5pt) rectangle +(5pt, 5pt);  
    \draw[fill=white] (0, 0) node{\textcolor{\PA}{$\bm{b}$}} +(-5pt, -5pt) rectangle +(5pt, 5pt);
    \draw[fill=white] (\X, 0) node{\textcolor{\PA}{$\bm{c}$}} +(-5pt, -5pt) rectangle +(5pt, 5pt);
    
    \draw (-\X, 5pt) node[above, font=\scriptsize]{$\textcolor{\PA}{\ADD_a^0}$};
    \draw (  0, 5pt) node[above, font=\scriptsize]{$\ldots \rightarrow \DEL_c \rightarrow \textcolor{\PA}{\ADD_a^2}$};
    \draw ( \X, 5pt) node[above, font=\scriptsize]{$\ldots \rightarrow \textcolor{\PA}{\ADD_a^3}$};
  \end{scope}
  
\end{tikzpicture}
}
    \hspace{5pt}
    \subfloat[Part E][\label{fig:undoproblemE}$a$ echoes back $\alpha_a^2$ to $b$.]
             {
\begin{tikzpicture}[scale=\SCALE]

%%   \draw (-\X + 5pt, 0) --
%%   node[above=-0.3em, font=\tiny]{~ ~ ~ ~ ~ $\delta_{a} \RIGHT$} %% b - a
%%   node[above=-0.3em, font=\tiny]{~ ~ ~ ~ ~ $\textcolor{\WRONG}{\text{\normalsize\xmark}} \hphantom{\RIGHT}$} %% b - a
%%   node[below=-0.3em, font=\tiny]{$\LEFT \textcolor{\PC}{\alpha_c^3}$} %% b - a
%%   (0 - 5pt, 0);

%%   \draw (0 +5pt, 0) --
%%   node[below=-0.3em, font=\tiny]{$\LEFT \delta_c \vphantom{\alpha^1}$}
%%   (\X -5pt, 0); %% b - c

%% 
  
%%   \draw[fill=white] (-\X, 0) node{$\bm{a}$} +(-5pt, -5pt) rectangle +(5pt, 5pt);  
%%   \draw[fill=white] (0, 0) node{\textcolor{\PC}{$\bm{b}$}} +(-5pt, -5pt) rectangle +(5pt, 5pt);
%%   \draw[fill=white] (\X, 0) node{\textcolor{\PA}{$\bm{c}$}} +(-5pt, -5pt) rectangle +(5pt, 5pt);
  
%%   \draw (-\X, 5pt) node[above, font=\scriptsize]{$\textcolor{\PA}{\alpha_a}\rightarrow \delta_a$};
%%   \draw (  0, 5pt) node[above, font=\scriptsize]{$\textcolor{\PA}{\alpha_a^2} \rightarrow \textcolor{\PC}{\alpha_c^1}$};
%%   \draw ( \X, 5pt) node[above, font=\scriptsize]{$\textcolor{\PC}{\alpha_c}\vphantom{\alpha_a^3} \rightarrow \delta_c \rightarrow \textcolor{\PA}{\alpha_a^3}$};


  \begin{scope}[shift={(0, -1*\Y)}]

    \draw [opacity=0] (-1.5*\X, 0) -- (1.5*\X, 0); %% spacing

    
    \draw (-\X + 5pt, 0) --
    node[above=-0.3em, font=\SMSG]{$\textcolor{\PA}{\alpha_a^2} \RIGHT$} %% b - a
    (0 - 5pt, 0);
    
    \draw [<->] (0 +5pt, 0) --
    node[opacity=\OACK, above=-0.3em, font=\SMSG]{~ ~ ~ ~ ~ ~$\alpha_c^2 \RIGHT$}
    node[opacity=\OACK, below=-0.3em, font=\SMSG]{$\LEFT \alpha_a^4$~ ~ ~ ~ ~ ~}
    (\X -5pt, 0); %% b - c
    
    \draw[fill=white] (-\X, 0) node{\textcolor{\PA}{$\bm{a}$}} +(-5pt, -5pt) rectangle +(5pt, 5pt);  
    \draw[fill=white] (0, 0) node{$\bm{b}$} +(-5pt, -5pt) rectangle +(5pt, 5pt);
    \draw[fill=white] (\X, 0) node{\textcolor{\PA}{$\bm{c}$}} +(-5pt, -5pt) rectangle +(5pt, 5pt);
    
    \draw (-\X, 5pt) node[above, font=\scriptsize]{$\textcolor{\PA}{\alpha_a^0}$};
    \draw (  0, 5pt) node[above, font=\scriptsize]{$\ldots \rightarrow \delta_c$};
    \draw ( \X, 5pt) node[above, font=\scriptsize]{$\textcolor{\PC}{\alpha_c}\vphantom{\alpha_a^3} \rightarrow \delta_c \rightarrow \textcolor{\PA}{\alpha_a^3}$};
    
  \end{scope}

\end{tikzpicture}
}
    \hspace{5pt}
    \subfloat[Part F][\label{fig:undoproblemF}$a$, $b$, $c$ belong to $P_a$.]
             {
\begin{tikzpicture}[scale=\SCALE]


  
  \draw [opacity=0] (-1.5*\X, 0) -- (1.5*\X, 0); %% spacing
  
  \draw [->](-\X + 5pt, 0) --
  node[opacity=\OACK, below=-0.3em, font=\SMSG]{$\LEFT \alpha_a^4$} 
  (0 - 5pt, 0);
  
  \draw [->] (0 +5pt, 0) --
  node[above=-0.3em, font=\SMSG]{$\textcolor{\PA}{\alpha^3_a} \RIGHT$}
  (\X -5pt, 0); %% b - c
  
  
  \draw[fill=white] (-\X, 0) node{\textcolor{\PA}{$\bm{a}$}} +(-5pt, -5pt) rectangle +(5pt, 5pt);  
  \draw[fill=white] (0, 0) node{\textcolor{\PA}{$\bm{b}$}} +(-5pt, -5pt) rectangle +(5pt, 5pt);
  \draw[fill=white] (\X, 0) node{\textcolor{\PA}{$\bm{c}$}} +(-5pt, -5pt) rectangle +(5pt, 5pt);
  
  \draw (-\X, -6pt) node[below, font=\SNODE]{$\textcolor{\PA}{\alpha_a^0}$};
  \draw (  0,  6pt) node[above, font=\SNODE]{$\ldots \rightarrow \textcolor{\PA}{\alpha_a^2}$};
  \draw ( \X, -6pt) node[below, font=\SNODE]{$\ldots \rightarrow \textcolor{\PA}{\alpha_a^3}$};

  
\end{tikzpicture}
}
  \end{center}
  \caption{\label{fig:undoproblem}From $c$'s perspective,
    Figure~\ref{fig:problemE} and Figure~\ref{fig:undoproblemE} are
    similar in terms of received messages, but the outcomes eventually
    differ. Yet, $c$ must act on Figure~\ref{fig:undoproblemE}, and
    acknowledge then propagate the \emph{possible} staleness of
    Partition $P_a$.}
\end{figure*}



\begin{definition}[Happens-before $\rightarrow$~\cite{lamport1978time}]
  The transitive, irreflexive, and antisymmetric happens-before
  relationship defines a strict partial order between events. Two
  messages are concurrent if none happens before the other.
\end{definition}

\begin{definition}[\label{def:lwo}Message staleness]
  Only the latest message of a \process matters. A message $m$ conveys
  \underline{s}tale control information if the \process that broadcast
  $m$ later broadcast another message $m'$: $\mathcal{S}(m)
  = \exists m': b_p(m) \rightarrow b_p(m')$.

  %% When a \process $p$ broadcasts two messages $b_p(m) \rightarrow
  %% b_p(m')$, no \process $q$ can deliver $m$ if it has delivered $m'$:
  %% $\nexists q \in V$ with $d_q(m') \rightarrow d_q(m)$, for $m$
  %% convey \emph{\underline{s}tale} control information: $S(m)$.
\end{definition}

Unfortunately, stale control information as stated in
Definition~\ref{def:lwo} may impair the propagation of both
\begin{inparaenum}[(i)]
\item notifications about actual sources, and
\item notifications about deleted partitions.
\end{inparaenum}
Figure~\ref{fig:problem} depicts a scenario comprising only three
\processes $a$, $b$, $c$ chained with FIFO links, \ie where \processes
receive the messages in the order of their sending ($s_{pq}(m)
\rightarrow s_{pq}(m') \implies r_q(m) \rightarrow r_q(m')$). In
Figure~\ref{fig:problemA}, both $a$ and $c$ become sources, sending
their respective notification message to $b$. In
Figure~\ref{fig:problemB}, both $a$ and $c$ delete their partition
while $b$ receives, delivers, and forwards $\alpha_a^2$. In
Figure~\ref{fig:problemC}, $b$ receives, delivers, and forwards
$\alpha_c^1$, for it improves its best partition. Then it receives and
discards $\delta_a$, for its best partition does not match the deleted
one. It produces an echo but since $\alpha_c^3$ is already transiting
towards $a$, we simplify it. Finally, Figure~\ref{fig:problemD} shows
that eventually, $c$ stays in the deleted partition $P_a$ for not
having received $\delta_a$ that $b$ blocked earlier. As a cascading
effect, not only $c$ wrongfully believes that it belongs to $a$'s
partition when $a$ already deleted it, but it could contribute to its
inconsistency by blocking farther but up-to-date messages from actual
sources. The system needs to purge all stale messages to guarantee
consistent partitioning. 

\begin{definition}[\label{def:purge}Message purge]
  A system \underline{p}urges a message $\TODO{\mathcal{P}}(m)$ when every
  \node $q$ that delivered $m$ eventually delivers another message
  $m'$ never followed by the delivery of the former message $m$:
  $d_q(m) \implies \exists m': \eventually (d_q(m)
  \rightarrow d_q(m') \wedge \neg (d_q(m') \rightarrow d_q(m)))$.
\end{definition}


\begin{theorem}[\label{theo:dcp}EFB$^+$: EFB $\wedge$ (Purge $\rightarrow$ EFB) $\implies$
    \underline{D}ynamic \underline{c}onsistent
    \underline{p}artitioning (DCP)]
%
  When a \process can \texttt{Add} and \texttt{Del}ete its partition,
  to ensure consistent partitioning, eventual forwarding of best
  (Theorem~\ref{theo:efb}) additionally requires
\begin{inparaenum}[(i)]
\item eventual purging of stale messages ($\mathcal{S}(m) \implies
  \mathcal{P}(m)$), then
\item the eventual forwarding of best (Theorem~\ref{theo:efb}).
\end{inparaenum}
\end{theorem}

\begin{proof}
  %% To reach consistent partitioning, the last message delivered by
  %% every process is not stale and comes from its respective best
  %% source: $d(m)$
  Assuming EFB as default propagation behavior where \processes
  deliver and forward the best sum of weights they received. We must
  prove that despite stale messages, the last delivery of every
  \process is the message that followed the shortest path to its
  closest source.
  \begin{asparadesc}
  \item [EFB without Purge:] A \process may receive messages in such
    an order (\eg in Figure~\ref{fig:problemC}: $d_b(\alpha_c^1)
    \rightarrow r_b(\delta_a) \rightarrow r_b(\delta_a)$) that it does
    not forward messages required by other \processes ($\delta_a$ for
    $c$). The last delivery of some nodes may be wrong (\eg in
    Figure~\ref{fig:problemD}: $\alpha_a^3$ instead of $\delta_a$).
    % \Processes never
    % reach consistent partitioning.
    %
    %% Figure~\ref{fig:proofA} depicts the generic case where a \process
    %% $b$ precludes itself and subsequent \processes such as $c$ of
    %% delivering the message from their respective best source
    %% $a$. \Process $b$ received and delivered a message from a
    %% partition that has since then been deleted but with a better sum
    %% of weights than newly received ones ($\alpha_d^y <
    %% \alpha_a^{x'}$). It forever discards such a message
    %% $\alpha_a^{x'}$.
  \item [Purge not followed by EFB:]
    %% also transiting messages
    Assuming that every \process removes control information as soon
    as it becomes stale ($S(\alpha_a) \implies d_b(\alpha_a^x)
    \rightarrow d_b(\delta_a)$) and cannot deliver stale messages
    anymore ($S(\alpha_a^x) \implies d_c(m) \rightarrow \ldots
    \rightarrow d_c(m' \not\in \alpha_a)$), staleness does not impair
    up-to-date messages propagation. However, even in such settings,
    depending on receipt order ($d_b(\alpha_a^{x}) \rightarrow
    r_b(\alpha_c^{y : y > x}) \rightarrow d_b(\delta_b)$), a \process
    may never receive again, hence deliver and forward a message that
    was received and discarded before ($\alpha_c^{y}$). When such a
    message contains its best up-to-date partition, a \process may
    never end up in its best partition.
    %    Therefore, \processes do not reach
    % consistent partitioning.
  \item [Purge then EFB:] Assuming the eventual and definitive removal
    of stale messages at each \process ($S(\alpha_a) \implies
    (d_b(\alpha_a) \implies \eventually d_b(\alpha_a) \rightarrow
    d_b(m') \not\rightarrow d_b(\alpha_a))$); and assuming that every
    removal eventually triggers the forwarding of best delivered
    messages ($d_c(\delta_x) \implies \eventually f_d(\alpha_y)$),
    \processes may deliver other stale messages that will be 
    eventually removed to never be delivered again. In the worst case,
    all \processes deliver and remove all stale messages to ultimately
    deliver and forward only up-to-date messages ($\eventually d_c(m')
    \implies \neg S(m')$). As in Theorem~\ref{theo:efb}, they
    eventually deliver and forward their best up-to-date message.
    
    %% upon receipt of such $\alpha_y$, either the \process already
    %% delivered its removal and it does 
    

    %% ($d_c(\delta_x) \implies \eventually f_d(\alpha_y) \wedge \neg
    %% S(\alpha_y)$)


    %% eventually every \process only delivers and forwards up-to-date
    %% messages, or no message at all 




    %% , every \process eventually delivers and forwards the best
    %% up-to-date delivered messages as in Theorem~\ref{theo:bef}.
    
    %% Removing \emph{forever} all such stale messages about
    %% deleted partitions would allow \processes to propagate their best
    %% up-to-date partition again, eventually reaching consistent
    %% partitioning together, as stated by Theorem~\ref{theo:bef}.
  \end{asparadesc}
  Using EFB$^+$, \processes reach consistent partitioning together
  even under dynamic partitioning settings where \nodes can
  \texttt{Add} and \texttt{Del}ete their partition.
\end{proof}

%% Figure removed as it's kind of redundant with preceding figure that
%% depicts the issue
%% \begin{figure} 
%%   \centering
\begin{tikzpicture}[scale=0.87]

  \thickmuskip=0mu
  \medmuskip=0mu
  \thinmuskip=0mu
  
  \newcommand\X{95pt}
  \newcommand\Y{-60pt}

  \newcommand\ADD{\alpha}
  \newcommand\DEL{\delta}

  \draw[opacity=0] (-1.2*\X, 0) -- (1.2*\X, 0);
  


  \draw[->] (-1.15*\X, 0) -- (-5pt + -1*\X, 0); \draw[dotted] (-1.25*\X, 0) -- (-1.15*\X, 0);
  \draw (5pt + \X, 0) -- (1.15*\X, 0); \draw[dotted,->] (1.15*\X, 0) -- (1.25*\X, 0);
  
  \draw[<->] (5pt - \X, 0.4*\Y) -- node[below, font=\small, align=center]{\textbf{I: propagation}\\\textbf{of $\alpha$ messages}} (-5pt, 0.4*\Y); %% I
  
  \draw[->] (-\X + 5pt, 0)
  node[below=-0.3em, below right, font=\tiny]{$\textcolor{\PA}{\ADD_a^{x'}} \rightarrow$}
  --
  (0 - 5pt, 0); %% a - b

  \draw[<->] (5pt, 0.4*\Y) -- node[below, align=center, font=\small]{\textbf{II: need purge}\\with I', III, IV} (-5pt + \X, 0.4*\Y); %% II
  \draw[->] (0 + 5pt, 0) -- (-5pt +  \X, 0); %% b - c


  
  \draw[fill=white] (-\X, 0) node[color=\PA]{$\bm{a}$} +(-5pt, -5pt) rectangle +(5pt, 5pt);  
  \draw[fill=white] (0, 0) node[color=\PD]{$\bm{b}$} +(-5pt, -5pt) rectangle +(5pt, 5pt);
  \draw[fill=white] (\X, 0) node{$\bm{c}$} +(-5pt, -5pt) rectangle +(5pt, 5pt);


  
  \draw (-\X, 5pt) node[above, color=\PA]{$\ADD_a^x$};
  \draw ( 0, 5pt) node[above]{$
    \xrightarrow[\textcolor{\PD}{\ADD_d^y}]{\mbox{\small{last}}}
    \xrightarrow[\DEL_d \rightarrow \textcolor{\PA}{\ADD_a^{x'}}]{\mbox{\small{expect}}}$};
  
  \draw (-5pt, 0pt) node [below=-0.3em, below left, font=\tiny]{$\textcolor{\WRONG}{\ADD_d^y < \ADD_a^{x'}}$};
  % \draw (-5pt, 0pt) node [below=-0.3em, below left, font=\tiny]{$\textcolor{\WRONG}{\ADD_d^{y} < \ADD_e^z}$};

  
  \draw (\X, 5pt) node[above]{$
    \xrightarrow[\textcolor{\PA}{\ADD_a^{x''}}]{\mbox{\small{expect}}}$};
%  \draw (0, -5pt) node[below, font=\scriptsize]{expect $\DEL_d \rightarrow \textcolor{\PA}{\ADD_a}$};


  
  \draw[->] (-\X,  0.5*\Y) -- (-\X, -5pt);
  \draw[dotted] (-\X,  0.5*\Y) -- (-\X, 0.9*\Y);

  \draw[->] ( 0,  0.5*\Y) -- ( 0, -5pt);
  \draw[dotted] ( 0,  0.5*\Y) -- ( 0, 0.9*\Y) node[below, font=\scriptsize]{some \process somewhere: $\textcolor{\PD}{\ADD_d} \rightarrow \DEL_d$};

  \draw[->] ( \X,  0.5*\Y) -- ( \X, -5pt);
  \draw[dotted] ( \X,  0.5*\Y) -- ( \X, 0.9*\Y);

  
\end{tikzpicture}

%%   \caption{\label{fig:proofA}Stale $\alpha$ messages may
%%       stop other $\alpha$ messages from reaching all \processes ($b$
%%       and $c$) that require it along the shortest path. Consistent
%%       partitioning requires eventual purging of stale messages (see
%%       Figure~\ref{fig:proofB}).}
%% \end{figure}

The main challenge consists in implementing a purging mechanism that
ensures the \emph{eventual and definitive} removal of stale control
information. One could guarantee consistent partitioning by always
propagating the $\delta$ messages corresponding to the $\alpha$
messages it propagated before. In Figure~\ref{fig:problem}, it means
that as soon as \Process~$b$ forwards $\alpha_a^2$, it assumes that
its neighbors \Process~$c$ and \Process~$d$ may need the notification
of removal $\delta_a$ if it exists. However, such a solution also
implies that \processes generate traffic not only related to their
current partition, but also related to partitions they belonged to in
the past. This would prove overly expensive in dynamic systems where
\processes join and leave the system, create and delete partitions, at
any time.  Instead, we propose to use the delivery order at each
\process to detect possible inconsistencies and solve them.
%% Together, \processes eventually remove all stale control information
%% (transiting messages and local states) of the system leaving room for
%% propagation of messages about up-to-date partitions.

%% 
\begin{figure*}
  \begin{center}
    \subfloat[Part A][\label{fig:proofA}Stale $\alpha$'s may stop up-to-date $\alpha$'s
    from reaching all processes that require it along the shortest path from $a$ to $c$.
    To solve this issue, we must guarantee
    the eventual removal of stale $\alpha$'s (see Figure~\ref{fig:proofB}).]
    {
\begin{tikzpicture}[scale=0.87]

  \thickmuskip=0mu
  \medmuskip=0mu
  \thinmuskip=0mu
  
  \newcommand\X{95pt}
  \newcommand\Y{-60pt}

  \newcommand\ADD{\alpha}
  \newcommand\DEL{\delta}

  \draw[opacity=0] (-1.2*\X, 0) -- (1.2*\X, 0);
  


  \draw[->] (-1.15*\X, 0) -- (-5pt + -1*\X, 0); \draw[dotted] (-1.25*\X, 0) -- (-1.15*\X, 0);
  \draw (5pt + \X, 0) -- (1.15*\X, 0); \draw[dotted,->] (1.15*\X, 0) -- (1.25*\X, 0);
  
  \draw[<->] (5pt - \X, 0.4*\Y) -- node[below, font=\small, align=center]{\textbf{I: propagation}\\\textbf{of $\alpha$ messages}} (-5pt, 0.4*\Y); %% I
  
  \draw[->] (-\X + 5pt, 0)
  node[below=-0.3em, below right, font=\tiny]{$\textcolor{\PA}{\ADD_a^{x'}} \rightarrow$}
  --
  (0 - 5pt, 0); %% a - b

  \draw[<->] (5pt, 0.4*\Y) -- node[below, align=center, font=\small]{\textbf{II: need purge}\\with I', III, IV} (-5pt + \X, 0.4*\Y); %% II
  \draw[->] (0 + 5pt, 0) -- (-5pt +  \X, 0); %% b - c


  
  \draw[fill=white] (-\X, 0) node[color=\PA]{$\bm{a}$} +(-5pt, -5pt) rectangle +(5pt, 5pt);  
  \draw[fill=white] (0, 0) node[color=\PD]{$\bm{b}$} +(-5pt, -5pt) rectangle +(5pt, 5pt);
  \draw[fill=white] (\X, 0) node{$\bm{c}$} +(-5pt, -5pt) rectangle +(5pt, 5pt);


  
  \draw (-\X, 5pt) node[above, color=\PA]{$\ADD_a^x$};
  \draw ( 0, 5pt) node[above]{$
    \xrightarrow[\textcolor{\PD}{\ADD_d^y}]{\mbox{\small{last}}}
    \xrightarrow[\DEL_d \rightarrow \textcolor{\PA}{\ADD_a^{x'}}]{\mbox{\small{expect}}}$};
  
  \draw (-5pt, 0pt) node [below=-0.3em, below left, font=\tiny]{$\textcolor{\WRONG}{\ADD_d^y < \ADD_a^{x'}}$};
  % \draw (-5pt, 0pt) node [below=-0.3em, below left, font=\tiny]{$\textcolor{\WRONG}{\ADD_d^{y} < \ADD_e^z}$};

  
  \draw (\X, 5pt) node[above]{$
    \xrightarrow[\textcolor{\PA}{\ADD_a^{x''}}]{\mbox{\small{expect}}}$};
%  \draw (0, -5pt) node[below, font=\scriptsize]{expect $\DEL_d \rightarrow \textcolor{\PA}{\ADD_a}$};


  
  \draw[->] (-\X,  0.5*\Y) -- (-\X, -5pt);
  \draw[dotted] (-\X,  0.5*\Y) -- (-\X, 0.9*\Y);

  \draw[->] ( 0,  0.5*\Y) -- ( 0, -5pt);
  \draw[dotted] ( 0,  0.5*\Y) -- ( 0, 0.9*\Y) node[below, font=\scriptsize]{some \process somewhere: $\textcolor{\PD}{\ADD_d} \rightarrow \DEL_d$};

  \draw[->] ( \X,  0.5*\Y) -- ( \X, -5pt);
  \draw[dotted] ( \X,  0.5*\Y) -- ( \X, 0.9*\Y);

  
\end{tikzpicture}
}
    \hspace{10pt}
    \subfloat[Part B][\label{fig:proofB}Stale $\alpha$'s may stop $\delta$'s from reaching
    processes with targeted $\alpha$'s. To ensure correctness, $b$ must either
    deliver $\delta_d$, $\delta_d^{0.5}$, or another $\alpha$, as well as downstream processes
    that delivered $\alpha$ coming from $b$ such as $g$.]
    {
\begin{tikzpicture}[scale=0.9]

  \thickmuskip=0mu
  \medmuskip=0mu
  \thinmuskip=0mu
  
  \newcommand\X{115pt}
  \newcommand\Y{-60pt}

  \newcommand\ADD{\alpha}
  \newcommand\DEL{\delta}
  \newcommand\DELDEL{\Delta}

  \draw[opacity=0] (-1.4*\X, 0) -- (2.4*\X, 0);
  


  \draw[dotted] (-1.5*\X, 0pt) -- (-1.35*\X, 0pt); %% X ->
  
  \draw[->] (-1.15*\X, 0) -- (-5pt + -1*\X, 0); \draw[dotted] (-1.25*\X, 0) -- (-1.15*\X, 0);
  \draw (5pt + 2*\X, 0) -- (2.15*\X, 0); \draw[dotted,->] (2.15*\X, 0) -- (2.25*\X, 0);
  
  \draw[<->] (5pt - \X, 0.4*\Y) --
  node[below, font=\small, align=center]{\textbf{I: propagation}\\\textbf{of $\delta$ messages}}
  (-5pt, 0.4*\Y); %% I as well
  
  \draw[->] (-\X + 5pt, 0)
  node[below=-0.3em, below right, font=\scriptsize]{$\DEL_X \rightarrow$}
  -- (0 - 5pt, 0)
  node [below=-0.15em, below left, font=\tiny]{$\textcolor{\WRONG}{a \neq c}$}
  ; %% d - e

  \draw[<->] (5pt , 0.4*\Y) --
  node[below, font=\small, align=center]{\textbf{II: detection}}
  (-5pt + \X, 0.4*\Y); %% IV
  
  \draw[->] (0 +5pt, 0)
  node[below=-0.3em, below right, font=\tiny]{$\textcolor{\PY}{\ADD_Y^{y'}} \rightarrow$}
  --
  (\X -5pt, 0)
  node [left= -0.15em, below=-0.3em, below left, font=\tiny]{$\textcolor{\WRONG}{\DEL_Y^{\vphantom{y'}} \not\rightarrow \ADD_Y^{y'}}$}
  ; %% e - f
  
  \draw[<->] (5pt +\X , 0.4*\Y) --
  node[below, font=\small, align=center]{\textbf{III: propagation}\\\textbf{of $\DELDEL$ messages}}
  (-5pt + 2*\X, 0.4*\Y); %% V
  
  \draw[->] ( \X +5pt, 0) -- (2*\X -5pt, 0); %% b - g

  \draw[dotted] (0.5*\X, 1.25*\Y) -- (0.2*\X, 1.1*\Y); %% Y ->
  \draw[dotted] (0.5*\X, 1.25*\Y) -- (0.8*\X, 1.1*\Y); %% Y ->



  \draw[fill=white] (-1.5*\X, 0) node{$\bm{X}$} +(-5pt, -5pt) rectangle +(5pt, 5pt);
  \draw[fill=white] (-\X, 0) node{$\bm{a}$} +(-5pt, -5pt) rectangle +(5pt, 5pt);  
  \draw[fill=white] (0, 0) node[color=\PY]{$\bm{b}$} +(-5pt, -5pt) rectangle +(5pt, 5pt);
  \draw[fill=white] (\X, 0) node[color=\PX]{$\bm{c}$} +(-5pt, -5pt) rectangle +(5pt, 5pt);
  \draw[fill=white] (2*\X, 0) node[color=\PX]{$\bm{d}$} +(-5pt, -5pt) rectangle +(5pt, 5pt);
  \draw[fill=white] (0.5*\X, 1.25*\Y)node{$\bm{Y}$}+(-5pt, -5pt) rectangle +(5pt, 5pt);



  \draw (-1.5*\X, 5pt) node[above]{$\textcolor{\PX}{\ADD_X} \rightarrow \DEL_X$};
  
  \draw (-\X, 5pt) node[above, font=\tiny]{$\textcolor{\PX}{\ADD_X} \rightarrow \DEL_X$};
  
  \draw ( 0, 5pt) node[above, font=\footnotesize]{$
    \xrightarrow[\textcolor{\PX}{\ADD_X^{\vphantom{x'}}}]{\mbox{\tiny{before}}}
    \xrightarrow[\textcolor{\PY}{\ADD_Y^{y\vphantom{'}}}]{\mbox{\tiny{last}}}
    \xrightarrow[\DEL_Y^{\vphantom{y'}} ]{\mbox{\tiny{expect}}}$};
  
  \draw ( \X, 5pt) node[above, font=\footnotesize]{$
    \xrightarrow[\textcolor{\PY}{\ADD_Y^{\vphantom{y'}}} \rightarrow
      \DEL_Y^{\vphantom{y'}}]{\mbox{\tiny{before}}}
    \xrightarrow[\textcolor{\PX}{\ADD_X^{x'}}]{\mbox{\tiny{last}}}
    \xrightarrow[\DELDEL_X^{\vphantom{y'}}]{\mbox{\tiny{expect}}}$};

  \draw (2*\X, 5pt) node[above, font=\footnotesize]{$
    \xrightarrow[\textcolor{\PX}{\ADD_X^{x''}}]{\mbox{\tiny{last}}}
    \xrightarrow[\DELDEL_X^{\vphantom{x'}}]{\mbox{\tiny{expect}}}$};

  \draw (0.5*\X, 1.25*\Y+5pt) node[above] {$\textcolor{\PY}{\ADD_Y} \rightarrow \DEL_Y$};
  


  \draw[->] (-\X,  0.5*\Y) -- (-\X, -5pt);
  \draw[dotted] (-\X,  0.5*\Y) -- (-\X, 0.9*\Y);

  \draw[->] ( 0,  0.5*\Y) -- ( 0, -5pt);
  \draw[dotted] ( 0,  0.5*\Y) -- ( 0, 0.9*\Y);

  \draw[->] ( \X,  0.5*\Y) -- ( \X, -5pt);
  \draw[dotted] ( \X,  0.5*\Y) -- ( \X, 0.9*\Y);

  \draw[->] (2*\X,  0.5*\Y) -- (2*\X, -5pt);
  \draw[dotted] (2*\X,  0.5*\Y) -- (2*\X, 0.9*\Y);
  
\end{tikzpicture}
}
    \caption{\label{fig:proof}Dynamic partitioning leads to correctness issues due to
      staleness and ordering of operations.}
  \end{center}
\end{figure*}



%% Figure~\ref{fig:proof} depicts the issues with staleness and message
%% orderings. In Figure~\ref{fig:proofA}, the shortest path from any
%% source to \Process $c$ is $[a, b, c]$. However, \Process $b$ still holds
%% a stale $\alpha_d^{0.5}$ without knowing. When it receives
%% $\alpha_a^1$, it discards it, for it assumes that downstream \processes
%% are more interested in $\alpha_d^{0.5}$. To reach consistent
%% partitioning, \Process $b$ first needs to purge its current partition
%% to later accept that of its current actual shortest path:
%% $\alpha_a^1$.

%% \noindent Figure~\ref{fig:proofB} shows that removing stale control
%% information is even more complex. The removal must reach all \processes
%% of the previous shortest path going from $d$ to $b''$. Label I' shows
%% the most obvious issue where \Process $e$ changed partition for a
%% better but stale $\alpha_f$. Since it can remember its previous
%% deliveries, it could still forward $\delta_d$ for the sake of
%% consistency. However, this would lead to every \process forwarding
%% every $\delta$ they ever delivered. Such protocol's overhead would
%% depend on past partitioning instead of current one. Label III shows
%% the issue when the blocking partition is already known to be stale at
%% \Process $b'$. \Process $b''$ eventually receives $\alpha_f$ from
%% $e$. However, it cannot deliver it, for it would break last win order.
%% \Process $b'$ may be in an inconsistent state. Label IV shows the
%% corollary issue: \Process $b''$ delivered a message from a \process that
%% may be inconsistent without knowing it.

\begin{theorem}[\label{theo:threephase}Three-phase purge]
  Three phases are sufficient to eventually purge the system from
  stale control information:
  \begin{inparaenum}[(i)]
  \item propagation of delete notifications,
  \item detection of possibly blocked deletion, and
  \item propagation of delete notifications that were possibly blocked.
  \end{inparaenum}
\end{theorem}

%% \begin{figure}
%%   \centering
\begin{tikzpicture}[scale=0.9]

  \thickmuskip=0mu
  \medmuskip=0mu
  \thinmuskip=0mu
  
  \newcommand\X{115pt}
  \newcommand\Y{-60pt}

  \newcommand\ADD{\alpha}
  \newcommand\DEL{\delta}
  \newcommand\DELDEL{\Delta}

  \draw[opacity=0] (-1.4*\X, 0) -- (2.4*\X, 0);
  


  \draw[dotted] (-1.5*\X, 0pt) -- (-1.35*\X, 0pt); %% X ->
  
  \draw[->] (-1.15*\X, 0) -- (-5pt + -1*\X, 0); \draw[dotted] (-1.25*\X, 0) -- (-1.15*\X, 0);
  \draw (5pt + 2*\X, 0) -- (2.15*\X, 0); \draw[dotted,->] (2.15*\X, 0) -- (2.25*\X, 0);
  
  \draw[<->] (5pt - \X, 0.4*\Y) --
  node[below, font=\small, align=center]{\textbf{I: propagation}\\\textbf{of $\delta$ messages}}
  (-5pt, 0.4*\Y); %% I as well
  
  \draw[->] (-\X + 5pt, 0)
  node[below=-0.3em, below right, font=\scriptsize]{$\DEL_X \rightarrow$}
  -- (0 - 5pt, 0)
  node [below=-0.15em, below left, font=\tiny]{$\textcolor{\WRONG}{a \neq c}$}
  ; %% d - e

  \draw[<->] (5pt , 0.4*\Y) --
  node[below, font=\small, align=center]{\textbf{II: detection}}
  (-5pt + \X, 0.4*\Y); %% IV
  
  \draw[->] (0 +5pt, 0)
  node[below=-0.3em, below right, font=\tiny]{$\textcolor{\PY}{\ADD_Y^{y'}} \rightarrow$}
  --
  (\X -5pt, 0)
  node [left= -0.15em, below=-0.3em, below left, font=\tiny]{$\textcolor{\WRONG}{\DEL_Y^{\vphantom{y'}} \not\rightarrow \ADD_Y^{y'}}$}
  ; %% e - f
  
  \draw[<->] (5pt +\X , 0.4*\Y) --
  node[below, font=\small, align=center]{\textbf{III: propagation}\\\textbf{of $\DELDEL$ messages}}
  (-5pt + 2*\X, 0.4*\Y); %% V
  
  \draw[->] ( \X +5pt, 0) -- (2*\X -5pt, 0); %% b - g

  \draw[dotted] (0.5*\X, 1.25*\Y) -- (0.2*\X, 1.1*\Y); %% Y ->
  \draw[dotted] (0.5*\X, 1.25*\Y) -- (0.8*\X, 1.1*\Y); %% Y ->



  \draw[fill=white] (-1.5*\X, 0) node{$\bm{X}$} +(-5pt, -5pt) rectangle +(5pt, 5pt);
  \draw[fill=white] (-\X, 0) node{$\bm{a}$} +(-5pt, -5pt) rectangle +(5pt, 5pt);  
  \draw[fill=white] (0, 0) node[color=\PY]{$\bm{b}$} +(-5pt, -5pt) rectangle +(5pt, 5pt);
  \draw[fill=white] (\X, 0) node[color=\PX]{$\bm{c}$} +(-5pt, -5pt) rectangle +(5pt, 5pt);
  \draw[fill=white] (2*\X, 0) node[color=\PX]{$\bm{d}$} +(-5pt, -5pt) rectangle +(5pt, 5pt);
  \draw[fill=white] (0.5*\X, 1.25*\Y)node{$\bm{Y}$}+(-5pt, -5pt) rectangle +(5pt, 5pt);



  \draw (-1.5*\X, 5pt) node[above]{$\textcolor{\PX}{\ADD_X} \rightarrow \DEL_X$};
  
  \draw (-\X, 5pt) node[above, font=\tiny]{$\textcolor{\PX}{\ADD_X} \rightarrow \DEL_X$};
  
  \draw ( 0, 5pt) node[above, font=\footnotesize]{$
    \xrightarrow[\textcolor{\PX}{\ADD_X^{\vphantom{x'}}}]{\mbox{\tiny{before}}}
    \xrightarrow[\textcolor{\PY}{\ADD_Y^{y\vphantom{'}}}]{\mbox{\tiny{last}}}
    \xrightarrow[\DEL_Y^{\vphantom{y'}} ]{\mbox{\tiny{expect}}}$};
  
  \draw ( \X, 5pt) node[above, font=\footnotesize]{$
    \xrightarrow[\textcolor{\PY}{\ADD_Y^{\vphantom{y'}}} \rightarrow
      \DEL_Y^{\vphantom{y'}}]{\mbox{\tiny{before}}}
    \xrightarrow[\textcolor{\PX}{\ADD_X^{x'}}]{\mbox{\tiny{last}}}
    \xrightarrow[\DELDEL_X^{\vphantom{y'}}]{\mbox{\tiny{expect}}}$};

  \draw (2*\X, 5pt) node[above, font=\footnotesize]{$
    \xrightarrow[\textcolor{\PX}{\ADD_X^{x''}}]{\mbox{\tiny{last}}}
    \xrightarrow[\DELDEL_X^{\vphantom{x'}}]{\mbox{\tiny{expect}}}$};

  \draw (0.5*\X, 1.25*\Y+5pt) node[above] {$\textcolor{\PY}{\ADD_Y} \rightarrow \DEL_Y$};
  


  \draw[->] (-\X,  0.5*\Y) -- (-\X, -5pt);
  \draw[dotted] (-\X,  0.5*\Y) -- (-\X, 0.9*\Y);

  \draw[->] ( 0,  0.5*\Y) -- ( 0, -5pt);
  \draw[dotted] ( 0,  0.5*\Y) -- ( 0, 0.9*\Y);

  \draw[->] ( \X,  0.5*\Y) -- ( \X, -5pt);
  \draw[dotted] ( \X,  0.5*\Y) -- ( \X, 0.9*\Y);

  \draw[->] (2*\X,  0.5*\Y) -- (2*\X, -5pt);
  \draw[dotted] (2*\X,  0.5*\Y) -- (2*\X, 0.9*\Y);
  
\end{tikzpicture}

%%   \caption{\label{fig:proofB}Stale $\alpha$ messages may stop $\delta$
%%     messages from reaching \processes with the corresponding
%%     partition. Consistent partitioning requires that all \processes
%%     propagate $\delta$ messages as often as possible (I); detect
%%     possible inconsistencies when parents' $\alpha$ messages
%%     contradict history or state (II); notify \processes of possible
%%     inconsistencies (III).}
%% \end{figure}

\begin{proof}
  We must prove that every \process that delivered a stale $\alpha_X$
  eventually delivers a better $\alpha_Y$, or delivers a removal
  notification of $\alpha_X$.  Figure~\ref{fig:proofB} summarizes the
  issue of purging in scenarios involving concurrent operations.
  
  (i) If every \process (such as $a$) starting from the source
  delivers and forwards a removal notification $\delta_X$ when their
  last delivery is $\alpha_X$, then every such \process eventually
  delivers the removal notification $\delta_X$ except \processes (such
  as $b$) that delivered $\alpha_X$ from a \process that delivered a
  message from another partition $\alpha_Y$ since then.
  
  These exceptions eventually receive and deliver $\alpha_Y$ since
  $\alpha_Y$ is better than $\alpha_X$ through this path, except if
  they already received and delivered the removal notification of
  $P_Y$ (such as $c$). These \processes may never receive hence
  deliver $\delta_X$, and may never receive hence deliver a better
  message than $\alpha_X$. They need an additional mechanism to
  eventually purge $\alpha_X$ that cannot rely on the eventual purging
  of $P_Y$ at \processes like $b$, to avoid deadlocks.
  
  (ii) Assuming that every \process keeps an history of its past
  deliveries, \processes (such as $c$) can detect the inconsistency,
  since they receive from their parent an already deleted partition
  $P_Y$. This notification means that the parent discards any
  $\delta_Z$ with $\alpha_Z^z < \alpha_Y^y$, and most importantly, if
  $\delta_X$ exists, it discards it. To ensure the purge of stale
  messages, a detecting \process must assume the worst case that such
  $\delta_X$ exists, and send another kind of message, noted $\Delta$,
  that notify the possible removal of $P_X$.
  
  (iii) A \process (such as $d$) whose last delivery is $\alpha_X$,
  but whose parent is neither inconsistent (like $b$) nor receiving
  $\delta_X$ (like $a$), eventually receives $\Delta$ from a detecting
  \process, either directly or transitively, for such a child \process
  ($d$) trusts the possible removal of $P_X$ by delivering and
  forwarding such $\Delta$. $\Delta$ suffers from identical blocking
  conditions (between $b$ and $c$) than $\delta$, leading to the same
  solutions of detection and propagation of $\Delta$. Eventually,
  every \process whose last delivery is $\alpha_X$ receives and
  delivers either a better $\alpha_Z$, or $\delta_X$, or $\Delta_X$.
  %
  %% Some \processes may have delivered messages about other partitions
  %% before or after $\alpha_X$. \Processes such as $b$ and $c$ delivered
  %% messages about $P_Y$.  For \processes such as $b$, the last delivery
  %% is $\alpha_Y$: $d_b(\_) \rightarrow \ldots \rightarrow d_b(\alpha_X)
  %% \rightarrow \ldots \rightarrow d_b(\alpha_Y)$; For \processes such
  %% as $c$, the last delivery is $\alpha_X$, and they delivered the
  %% addition and removal of $P_Y$: $d_c(\_) \rightarrow \ldots
  %% \rightarrow d_c(\alpha_Y) \rightarrow \ldots \rightarrow
  %% d_c(\delta_Y) \rightarrow \ldots \rightarrow d_c(\alpha_X)$. Both
  %% $a$ and $d$ delivered $\alpha_X$ last: $d_a(\_) \rightarrow \ldots
  %% \rightarrow d_a(\alpha_X)$. Their difference being their position in
  %% the chain: $d$ is a child of \processes that delivered messages
  %% about $P_Y$.
  %
  %% \begin{asparadesc}
  %% \item [($a$ or $c$ or $d$) to ($a$ or $c$ or $d$):] The source is
  %%   like $a$.  By propagating from \process to \process the
  %%   corresponding $\delta_X$ message, these \processes purge
  %%   $\alpha_X$.
  %% \item [$a$ to $b$:] $b$ stops the propagation of $\delta_X$, for the
  %%   latter does not target its current partition.
  %% \item [$b$:] They need to purge their own $P_Y$, we must \TODO{prove
  %%   that $[a, b, c, d]$ is solved and that it does not require to
  %%   purge $P_Y$.}
  %% \item [$b$ to $d$:] would deliver $d_(\alpha_y)$ hence becoming $b$.
  %% \item [$b$ to $c$:] block $d_c(\alpha_y)$, so $c$ stays in $c$. But
  %%   it detects $b$ is its parent, and possibly blocked any delta with
  %%   its value $r_b(\delta_z) \rightarrow d_b(\delta_z)$. Since not
  %%   sure, $\Delta$
  %
  %% \item [\processes with $\alpha_X^x \rightarrow \alpha_Y^y$ with last
  %%   $\alpha_Y^y$, for $\alpha_Y^y< \alpha_X^x$:] \Processes such as
  %%   \Process~$b$ that do not directly suffer from the non-delivery of
  %%   $\delta_X$. They need to purge their own $P_Y$ if need
  %%   be. Nonetheless, they deliver and forward $\alpha_Y$ that at least
  %%   one following \process will receive.
  %% \item [\processes with $\delta_Y \rightarrow \alpha_X$:] \Processes
  %%   such as \Process~$c$ that do not belong to the preceding category,
  %%   for they already delivered $\delta_Y$. Nevertheless, best eventual
  %%   forwarding guarantees that they receive $\alpha_Y$, which
  %%   contradicts their history or state. In such a case, the detecting
  %%   \processes purge their own $\alpha_X$. The detecting \processes
  %%   must also notify subsequent \processes that their current
  %%   partition \emph{may be} deleted. Indeed their parent may have
  %%   blocked the corresponding $\delta$ messages and subsequent
  %%   \processes must purge their possibly stale state, for the sake of
  %%   consistency.  We note such a notification $\Delta_X$ to emphasize
  %%   their proximity to $\delta_X$ messages.
  %%   % the uncertainty of a $\delta$ message.
  %% \item [\processes with last $\alpha_X$ receiving $\Delta_X$ from
  %%   their parent:] \Processes such as \Process~$d$ that have
  %%   $\alpha_X$ but preceding delivered messages are not
  %%   important. Such \processes must trust detecting \processes by
  %%   delivering and forwarding $\Delta_X$. It is worth noting that
  %%   these messages are subject to all aforementioned blocking
  %%   conditions, but are also solved by aforementioned mechanisms.
  %% \end{asparadesc}
  %% Propagation~I terminates: a \process does not deliver a $\delta$
  %% message if it has already delivered it. Propagation~III terminates: a
  %% \process does not deliver a looping message.  
  % ; and since \processes forward the messages they deliver, upstream
  % \processes eventually receive and may deliver messages detected as
  % issue. In Figure~\ref{fig:proofB}, \Process~$e$ eventually
  % receives $\delta_f$ from \Process~$b'$.
  %% Three-phase purge ensures that all \processes of the system
  %% eventually removes stale messages, leaving room for best eventual
  %% forwarding of up-to-date messages.
  %
  %  \TODO{cannot loop add from $e$ , undo from $b'$ because $b'$ sent
  % delete to $e$, it will solve the inconsistency.}
  %
  %% In the meantime, \processes that still belong to a partition can
  %% propagate their respective last $\alpha$ message, and reach
    %% consistent partitioning.
\end{proof}



\subsection{Lazy dynamic consistent partitioning}
\label{subsec:lazy}

In Edge infrastructures comprising tens of thousands of autonomous
systems~\cite{nur2018crossas}, only a small subset of autonomous
systems needs logical partitioning. At any time, an autonomous system
start logical partitioning when a \process from this network becomes a
source.  Then every \process from this network belongs to the
partition of its closest source not only from their own network but
that of adjacent networks that started logical partitioning as well.
At any time, an autonomous system stops logical partitioning when the
last source from this network revokes its status of
source. Eventually, all nodes from this network do not belong to any
partition and the generated traffic stops.

\begin{definition}[Edge infrastructure]
  An Edge infrastructure $\mathcal{G}$ has multiple autonomous systems
  interconnected with additional links $\mathcal{E} \in V_1 \times
  V_2$ where $V_1 \neq V_2$, hence $\mathcal{G} = \langle \{G_1,
  \ldots G_n\}, \mathcal{E} \rangle$.
\end{definition}

\begin{definition}[\TODO{Lazy DCP}]
  
\end{definition}

To provide lazy partitioning, \emph{gateway} \nodes that link
autonomous systems together must acknowledge if there exists a source
in their network.

Must differentiate local from global when they cross the gates.

% \NAME enables indexing the closest replica of each \process without
% flooding the whole network when a new replica appears or disappears.
% Nevertheless, every node must partake in the indexing of each
% content. When the networks accounts for billions of content, this
% raises scalability issues. Inspired by the Internet topology that
% comprises tens of thousands of interconnected autonomous systems
% serving billions of , we want to further
% lock-down traffic by leveraging the edges of networks as barriers, and
% enable on-demand indexing.


\TODO{Added invariant: global is always smaller than local when they
  exist.}


%%% Local Variables: 
%%% mode: latex
%%% TeX-master: "../paper"
%%% ispell-local-dictionary: "english"
%%% End: 
