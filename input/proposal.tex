
\section{Dynamic partitioning}
\label{sec:proposal}


\subsection{Complexity}
\label{subsec:complexity}

We focus on average-case and worst-case complexity. We divide our
analysis into space, time, and communication complexity.

\textbf{The communication complexity} concerns the size and number of
messages required to reach optimal partitioning. In the average-case,
a process $i$ chosen uniformly at random among all processes creates
a logical partition. Its messages $\alpha_i$ propagate through the
network until reaching processes that belong to another partition
closer to them. This splits partitions in half in average. Overall,
the $a^{th}$ new partition comprises
\smash{$\mathcal{O}(\frac{|V|}{2^{\lfloor \log_2 a \rfloor}})$}
processes. This decreases every new partition until reaching $0$
processes per new partition: even the chosen process already belongs
to its optimal partition. The average number of messages per process
is \smash{$\mathcal{O}(\frac{\overline{|O|}}{|2^{\lfloor \log_2 a
      \rfloor}})$}.  Deleting the $a^{th}$ partition generates the
exact same number of messages than the $a^{th}$ partition
creation. \TODO{But what about echos?} In the worst-case, every new
partition includes all but one process belonging to the previous
partition. The total number of messages after the $a^{th}$ new
partition is $O(\overline{|O|}\cdot a^2)$. As for the average-case,
the number of messages for the partition deletion is identical to the
number of messages of the corresponding partition creation.

%%% Local Variables: 
%%% mode: latex
%%% TeX-master: "../paper"
%%% ispell-local-dictionary: "english"
%%% End: 
