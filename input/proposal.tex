
\section{Scoped broadcast}
\label{sec:scoped}

In this section, we introduce model and notations to provide a general
definition of scoped broadcast in distributed systems.  Distributed
systems such as Edge infrastructures comprise heterogeneous machines
interconnected by communication links. We do not consider byzantine
processes.

\begin{definition}[Fog infrastructure]
  A fog infrastructure is a connected \underline{g}raph $G(V, E)$ of
  \underline{v}ertices $V$ and bidirectional \underline{e}dges $E
  \subseteq V \times V$.  A \underline{p}ath $\pi_{ij}$ from Process
  $i$ to Process $j$ is a sequence of vertices $[i, k_1, k_2, \ldots
    k_n, j]$ with $\forall m: 0\leq m \leq n, \langle \pi_{ij}[m],
  \pi_{ij}[m+1] \rangle \in E$.
\end{definition}

Processes can send messages to their neighboring processes. Processes
can reliably broadcast messages by forwarding delivered
messages~\cite{hadzilacos1994modular,nedelec2018causal,raynal2013distributed}. However,
uniform reliable broadcast states that \emph{all} correct processes
must deliver broadcast messages. This proves costly in large scale
systems comprising thousands of processes. Instead, we define
\emph{scoped broadcast} where messages reach only interested
processes, significantly reducing generated traffic.

\begin{definition}[\underline{S}coped broadcast (S-broadcast)]
  When Process $x$ scoped \underline{b}roadcasts $b_x(m)$ a
  \underline{m}essage $m$, every correct Process $y$ within a scope
  \underline{r}eceives $r_y(m)$ and \underline{d}elivers it
  $d_y(m)$. The scope depends on \underline{s}tates $\sigma$ of
  processes, \underline{m}etadata $\mu$ within received messages, and
  a \underline{p}redicate $\phi$ verified from process to process:
  $b_x(m) \wedge r_y(m) \implies \exists \pi_{xy} : \forall z \in
  \pi_{xy}, \phi(\mu_z, \sigma_z)$.
\end{definition}

This definition highlights the transitive relevance of messages and
encompasses more specific definitions of related
work~\cite{hsiao2005scoped,lue2006scoped,wang2015prodiluvian}.  A
process from Paris could scoped broadcast a message to all processes
in this city. A process from Paris could scoped broadcast a message to
all processes in this city plus neighboring cities by overloading
forwarded messages the first time they reach another city. Similarly
to reliable broadcast, scoped broadcast implementations expose
different trade-offs on space, time, and communication.

%% \begin{definition}[S-broadcast properties]
%%   S-broadcast guarantees 3 properties:
%%   \begin{inparaenum}[(i)]
%%   \item Validity: If a correct process broadcasts a message, it
%%     eventually delivers it;
%%   \item Scoped Agreement: If a process broadcasts a message, then all
%%     correct processes within the scope eventually delivers it;
%%   \item Scoped Integrity: A process delivers a message only if it was
%%     previously broadcast within the scope.
%%   \end{inparaenum}
%% \end{definition}

In this paper, we propose to dynamically adapt the scope of
S-broadcast using S-broadcast itself. At any time, a process can
become the source of a new logical partition by transitively modifying
the respective state of neighboring processes using path weights. When
a process S-broadcasts a message in this partition, it eventually
reaches all processes belonging to this partition.  While the first
partition gathers all processes of the system, the following
partitions reduce the membership size of their neighboring
partitions. In turns, the traffic generated to adapt scopes scales
with the number of partitions. 

\begin{definition}[\underline{C}onsistent \underline{p}artitioning (CP)]
  Let $S \subseteq V$ be the set of \underline{s}ources, $W_{xy} =
  W_{yx}$ be the symmetric \underline{w}eight between $x$ and $y$,
  $\Pi_{xz}$ be the shortest \underline{p}ath from $x$ to $z$ the
  weight of which $|\Pi_{xz}|$ is lower than any other path weight,
  the only consistent partitioning $\mathcal{P}$ is a set of logical
  partitions $P_{s\in S}$ such that each process belongs to a logical
  partition comprising its closest source: $\forall p \in P_{s_1},
  \nexists P_{s_2}$ such that $|\Pi_{s_1p}| > |\Pi_{s_2p}|$.
\end{definition}

%% \begin{theorem}[$SPED \implies CP$]
%%   \underline{S}hortest \underline{p}ath \underline{e}ventual
%%   \underline{d}elivery (SPED) where each process $k_n$ in the shortest
%%   path $\Pi_{ps}: [s, k_1,\ldots,k_n,p]$ eventually deliver the
%%   message forwarded by its predecessor $k_{n-1}$ leads to consistent
%%   partitioning.
%% \end{theorem}

%% \TODO{or:}

\begin{theorem}[\underline{B}est \underline{e}ventual \underline{d}elivery $(BED) \implies CP$]
  Every process eventually delivers, hence forwards, the notification
  message $\alpha_s$ of its best partition $P_s$, eventually leading
  to consistent partitioning.
\end{theorem}

\begin{proof} When a process $s_1$ becomes a source, it belongs to
its own partition, for there exists no better partition than its own:
$\forall p \in V: |\Pi_{s_1 s_1}| < |\Pi_{s_1 p}|$. It broadcasts such
message to its neighbors. Since communication channels are reliable,
neighboring processes eventually receive such notification. In
particular, whatever the order of received messages, every process
$p_1$ such that there exists no better partition $s_2$ than the
received one delivers and forwards it: $\forall s_2 \in S: |\Pi_{p
  s_1}| < |\Pi_{p s_2}|$. The notification emanating from $s_1$
transitively reaches all such processes through shortest paths. Since
there exists only one best value per process that can never be
retracted, processes eventually reach consistent partitioning.
\end{proof}

%% 
\begin{figure}[t]
  \newcommand{\SCALE}{0.8}

  \newcommand\X{38pt}
  \newcommand\Y{-40pt}

  \thickmuskip=0mu
  \medmuskip=0mu
  \thinmuskip=0mu

  \newcommand\ADD{\alpha}
  \newcommand\DEL{\delta}
    
  \newcommand{\LEFT}{\triangleleft}
  \newcommand{\RIGHT}{\triangleright}
  
  \begin{center}
    \subfloat[Part A][\label{fig:delA}$a$ deletes its partition. It
      notifies all \processes that belong to its
      partition.]{
\begin{tikzpicture}[scale=\SCALE]

  \draw[opacity=0](-2.45*\X, 0) -- (2.45*\X, 0); %% more space for caption
  

  
  \draw (-\X + 5pt, 0) --
  node[above=-0.3em,font=\tiny]{$\DEL_{a} \RIGHT$}
  (0 - 5pt, 0); %% b - a 

  \draw (0 +5pt, 0) --
  (\X -5pt, 0); %% b - c

  \draw [<-] (0, 0 - 5pt) --
  (0, \Y + 5pt);  %% b - d
  
  \draw [<-] (\X + 3pt, 0 - 5pt) --
  (0 + 5pt, \Y - 3pt); %% c - d


  
  \draw[fill=white] (-\X, 0) node{$\bm{a}$} +(-5pt, -5pt) rectangle +(5pt, 5pt);  
  \draw[fill=white] (0, 0) node[color=\PD]{$\bm{b}$} +(-5pt, -5pt) rectangle +(5pt, 5pt);
  \draw[fill=white] (\X, 0) node[color=\PD]{$\bm{c}$} +(-5pt, -5pt) rectangle +(5pt, 5pt);
  \draw[fill=white] (0, \Y) node[color=\PD]{$\bm{d}$} +(-5pt, -5pt) rectangle +(5pt, 5pt);
  
  \draw (-\X, 5pt) node[above, font=\small]{$\DEL_a$};
  \draw (0, 5pt) node[above, font=\small, color=\PD]{$\ADD_d^1$};
  \draw (\X, 5pt) node[above, font=\small, color=\PD]{$\ADD_d^2$};
  \draw (-5pt, \Y) node[left, font=\small, color=\PD]{$\ADD_d^0$};


\end{tikzpicture}
}
    \hspace{3pt}
    \subfloat[Part B][\label{fig:delB}$\delta$ stops as soon as it encounters
      another partition. $b$ answers with its partition.]{
\begin{tikzpicture}[scale=0.87]

  \thickmuskip=0mu
  \medmuskip=0mu
  \thinmuskip=0mu
  
  \newcommand\X{50pt}
  \newcommand\Y{-50pt}

  \newcommand\ADD{\alpha}
  \newcommand\DEL{\delta}



  \draw (-\X + 5pt, 0) --
  node[above=-0.3em,font=\tiny]{$\DEL_{a} \rightarrow$}
  node[below=-0.3em,font=\tiny]{$\leftarrow \ADD_{a}^{3}$}
  (0 - 5pt, 0); %% b - a 

  \draw (0 +5pt, 0) --
  node[above=-0.3em, font=\tiny]{$\leftarrow \textcolor{\PC}{\ADD_{c}^{1}}\cdot\DEL{c}$}
  node[below=-0.3em, font=\tiny]{$\textcolor{\PA}{\ADD_a^{2.5}} \rightarrow$}
  (\X -5pt, 0); %% b - c

  \draw[opacity=0] (0, 0 - 5pt) --
  node[opacity=1, above=-0.3em, font=\tiny, sloped]{$\ADD_a^{2.5} \rightarrow$}
  (0, \Y + 5pt); %% b - d
  \draw (0, \Y + 5pt) --
  node[above=-0.3em, font=\tiny, sloped]{$\ADD_c^2 \rightarrow$}
  (0, 0 - 5pt);  %% d - b
  
  \draw (\X + 3pt, 0 - 5pt) --
  node[above=-0.3em, font=\tiny, sloped]{$\ADD_{c}^{2} \rightarrow$}
  node[below=-0.3em, font=\tiny, sloped]{$\leftarrow \DEL_c$}
  (0 + 5pt, \Y - 3pt); %% c - d



  
  \draw[fill=white] (-\X, 0) node{$\bm{a}$} +(-5pt, -5pt) rectangle +(5pt, 5pt);  
  \draw[fill=white] (0, 0) node[color=\PA]{$\bm{b}$} +(-5pt, -5pt) rectangle +(5pt, 5pt);
  \draw[fill=white] (\X, 0) node{$\bm{c}$} +(-5pt, -5pt) rectangle +(5pt, 5pt);
  \draw[fill=white] (0, \Y) node[color=\PC]{$\bm{d}$} +(-5pt, -5pt) rectangle +(5pt, 5pt);
  
  % \draw (-\X+5pt, 5pt) node[above left]{$\DEL_a$};
  % \draw (\X+5pt, 5pt) node[above right]{$\DEL_c$};
  \draw (0, 5pt) node[above]{$\bm{a: 1.5}$};
  \draw (-5pt, \Y) node[left]{$\bm{c: 1}$};


\end{tikzpicture}
}
  \end{center}
  \caption{\label{fig:del}Efficient removal of a partition using
    scoped broadast. $a$ eventually acknowledges that it belongs to $P_d$ with $\alpha_d^3$ echoing.}
\end{figure}

%%% Local Variables: 
%%% mode: latex
%%% TeX-master: "../paper"
%%% ispell-local-dictionary: "english"
%%% End: 

 %% figure positioning

%% \begin{definition}[\label{def:transitivedelivery}Transitive delivery]
%%   Transitive delivery ensures the system eventually converges to
%%   optimal partitioning:
%%   \begin{inparaenum}[(i)]
%%   \item (\textbf{termination}) processes do not need to propagate
%%     operations they do not deliver: $\forall \langle i, j \rangle \in
%%     E, \neg d_i(\alpha_{s}^w)\implies \neg
%%     d_j(\alpha_{s}^{w+w_{ij}})$;
%%   \item (\textbf{propagation}) operations delivered by a process may
%%     also benefit neighbor processes: $\forall \langle i, j \rangle \in
%%     E, d_i(\alpha_{s}^{w_1}) \implies
%%     d_j(\alpha_{s}^{w_1+w_{ij}}) \vee
%%     d_j(\alpha_{*}^{w_2 < w_1 + w_{ij}})$.
%%   \end{inparaenum}
%% \end{definition}

\begin{algorithm}
  \SetKwProg{Function}{func}{}{}

\small

\DontPrintSemicolon
\LinesNumbered

$O_p$, $W_p$\tcp*[r]{set of neighbors and weights}
$s \leftarrow \varnothing$ \tcp*[r]{best source of partition ($\sigma$)}
$d \leftarrow \infty$ \tcp*[r]{smallest distance to $s$ ($\sigma$ and $\mu$)}


\BlankLine

\Function{\textup{Add ( )} \tcp*[f]{$\alpha_p^0$}} {
  \textup{receiveAdd($\varnothing$, $p$, $0$)} \label{line:lowestbound} \tcp*[f]{$b_p(\alpha_p^0)$}
}

\BlankLine

\Function{\textup{receiveAdd($q$, $s'$, $d'$)} \tcp*[f]{$r_p(\alpha_{s'}^{d'})$ from $q$}} {
  \If (\tcp*[f]{($\phi$)}){$d' < d$} {
      $s \leftarrow s'$ \tcp*[r]{\smash{$d_p(\alpha_{s'}^{d'})$}}
      $d \leftarrow d'$ \;

      \ForEach(\tcp*[f]{\smash{$f_p(\alpha_{s'}^{d'})$}}) {$n \in O_p \setminus q$} {
          \textup{send$_n$($s', d' + W_{pn}$)} \label{line:accumulator}
          \tcp*[r]{\smash{$s_{pn}(\alpha_{s'}^{d'+w_{pn}})$}}
       }      
  }
}

%% \BlankLine

%% \Function{\textup{edgeUp($q$)} \tcp*[f]{new link to $q$}} {
%%   \lIf { $d < \infty$} {\textup{send$_q$($s, d + W_{pq}$)}}
%% }



  \caption{\label{algo:add}Adding a partition by Process $p$.}
\end{algorithm}

Algorithm~\ref{algo:add} shows the instructions that implement
eventually consistent partitioning when
\begin{inparaenum}[(i)]
\item processes only have local knowledge about their direct
  neighborhoods,
\item weights are scalar values,
\item and processes only add new partitions to the system.
\end{inparaenum}
At any time, a process running this protocol can add a logical
partition to its distributed system.  Figure~\ref{fig:add} illustrates
its behavior on a system comprising 4 processes $a$, $b$, $c$, and
$d$. Two processes $a$ and $c$ concurrently create a partition by
S-broadcasting their respective \underline{a}dd message: $\alpha_a^0$
and $\alpha_c^0$. They initialize their own state with the lowest
possible bound $0$ (see Line~\ref{line:lowestbound}), and send a
message to each of their neighbors by accumulating the corresponding
edge weight (see Line~\ref{line:accumulator}). In
Figure~\ref{fig:addC}, Process $b$ receives $\alpha_{c}^{1}$. Since it
improves its own partition distance, it keeps it and forwards it to
its neighbors. Process $b$ discards $\alpha_{b}^{1.5}$, for it does
not improve its partition distance. Processes $c$ and $d$ will never
hear of Process $a$'s partition.  In Figure~\ref{fig:addD}, processes
discard last transiting messages. The system converged to the
consistent partitioning. However, since processes operate using local
knowledge only, generated traffic may be sub-optimal, for processes
may receive, deliver, and forward messages emanating from a same
source simply because it did not come from the shortest path yet.

%% Transitive relationships ensure that each
%% process gets its closest source while accumulation of weights ensures
%% that messages propagation terminates.

%% Interestingly, while the receipt order of messages does not impact on
%% local control information ($r_b(\alpha_a^{1.5}) \cdot
%% r_b(\alpha_a^{1}) \Leftrightarrow r_b(\alpha_a^{1}) \cdot
%% r_b(\alpha_a^{1.5}) \implies \TODO{d_b(\alpha_a^{1})}$), it impacts on
%% control information broadcast in the network ($r_b(\alpha_a^{1.5})
%% \cdot r_b(\alpha_a^{1}) \implies b_b(\alpha_{b}^{1.5}) \cdot
%% b_b(\alpha_{b}^{1})$ while $r_b(\alpha_a^{1}) \cdot
%% r_b(\alpha_a^{1.5}) \implies b_b(\alpha_a^{1})$).

While only adding logical partitions to the distributed system is
straightforward, removing them introduces additional complexity:
messages of old partitions may stop the notification of best sources,
for former weights are better than the newly received one. 





\section{Adaptive scoped broadcast}
\label{sec:adaptive}

In this section, we introduce \NAME (stands for \underline{A}daptive
\underline{S}coped broad\underline{cast}), a wait-free reactive
protocol for dynamic logical partitioning in dynamic distributed
systems the cost of which actually depends on its usage. At any time,
processes can add \emph{and} remove partitions from the system. When
the system becomes quiescent, processes eventually converge to their
respective partition and do not require further communication
afterward.

Each process works autonomously and asynchronously, in a peer-to-peer
fashion. Each process broadcasts each change to its neighborhood, for
the state of its neighbors may depend on this change as well. Yet, the
traffic generated to to adapt the scope of scoped broadcast remains
bounded, for it uses the principles of scoped broadcast
itself. Messages that carry changes travel through the network
depending on partition of processes, stopping as soon as they
encounter uninterested processes. We demonstrate that \NAME guarantees
consistent partitioning despite different order in message deliveries
from one process to another.

%% This section starts by describing \NAME's functioning by detailing its
%% operation to avoid inconsistent states due to ordering and staleness
%% of delivered messages. This section ends with a complexity analysis of
%% proposed protocols.



%% \subsection{Intra-autonomous system partitioning}


\begin{figure*}
  \begin{center}
    \subfloat[Part A][\label{fig:proofA}Stale $\alpha$'s may stop up-to-date $\alpha$'s
    from reaching all processes that require it along the shortest path from $a$ to $c$.
    To solve this issue, we must guarantee
    the eventual removal of stale $\alpha$'s (see Figure~\ref{fig:proofB}).]
    {
\begin{tikzpicture}[scale=0.87]

  \thickmuskip=0mu
  \medmuskip=0mu
  \thinmuskip=0mu
  
  \newcommand\X{95pt}
  \newcommand\Y{-60pt}

  \newcommand\ADD{\alpha}
  \newcommand\DEL{\delta}

  \draw[opacity=0] (-1.2*\X, 0) -- (1.2*\X, 0);
  


  \draw[->] (-1.15*\X, 0) -- (-5pt + -1*\X, 0); \draw[dotted] (-1.25*\X, 0) -- (-1.15*\X, 0);
  \draw (5pt + \X, 0) -- (1.15*\X, 0); \draw[dotted,->] (1.15*\X, 0) -- (1.25*\X, 0);
  
  \draw[<->] (5pt - \X, 0.4*\Y) -- node[below, font=\small, align=center]{\textbf{I: propagation}\\\textbf{of $\alpha$ messages}} (-5pt, 0.4*\Y); %% I
  
  \draw[->] (-\X + 5pt, 0)
  node[below=-0.3em, below right, font=\tiny]{$\textcolor{\PA}{\ADD_a^{x'}} \rightarrow$}
  --
  (0 - 5pt, 0); %% a - b

  \draw[<->] (5pt, 0.4*\Y) -- node[below, align=center, font=\small]{\textbf{II: need purge}\\with I', III, IV} (-5pt + \X, 0.4*\Y); %% II
  \draw[->] (0 + 5pt, 0) -- (-5pt +  \X, 0); %% b - c


  
  \draw[fill=white] (-\X, 0) node[color=\PA]{$\bm{a}$} +(-5pt, -5pt) rectangle +(5pt, 5pt);  
  \draw[fill=white] (0, 0) node[color=\PD]{$\bm{b}$} +(-5pt, -5pt) rectangle +(5pt, 5pt);
  \draw[fill=white] (\X, 0) node{$\bm{c}$} +(-5pt, -5pt) rectangle +(5pt, 5pt);


  
  \draw (-\X, 5pt) node[above, color=\PA]{$\ADD_a^x$};
  \draw ( 0, 5pt) node[above]{$
    \xrightarrow[\textcolor{\PD}{\ADD_d^y}]{\mbox{\small{last}}}
    \xrightarrow[\DEL_d \rightarrow \textcolor{\PA}{\ADD_a^{x'}}]{\mbox{\small{expect}}}$};
  
  \draw (-5pt, 0pt) node [below=-0.3em, below left, font=\tiny]{$\textcolor{\WRONG}{\ADD_d^y < \ADD_a^{x'}}$};
  % \draw (-5pt, 0pt) node [below=-0.3em, below left, font=\tiny]{$\textcolor{\WRONG}{\ADD_d^{y} < \ADD_e^z}$};

  
  \draw (\X, 5pt) node[above]{$
    \xrightarrow[\textcolor{\PA}{\ADD_a^{x''}}]{\mbox{\small{expect}}}$};
%  \draw (0, -5pt) node[below, font=\scriptsize]{expect $\DEL_d \rightarrow \textcolor{\PA}{\ADD_a}$};


  
  \draw[->] (-\X,  0.5*\Y) -- (-\X, -5pt);
  \draw[dotted] (-\X,  0.5*\Y) -- (-\X, 0.9*\Y);

  \draw[->] ( 0,  0.5*\Y) -- ( 0, -5pt);
  \draw[dotted] ( 0,  0.5*\Y) -- ( 0, 0.9*\Y) node[below, font=\scriptsize]{some \process somewhere: $\textcolor{\PD}{\ADD_d} \rightarrow \DEL_d$};

  \draw[->] ( \X,  0.5*\Y) -- ( \X, -5pt);
  \draw[dotted] ( \X,  0.5*\Y) -- ( \X, 0.9*\Y);

  
\end{tikzpicture}
}
    \hspace{10pt}
    \subfloat[Part B][\label{fig:proofB}Stale $\alpha$'s may stop $\delta$'s from reaching
    processes with targeted $\alpha$'s. To ensure correctness, $b$ must either
    deliver $\delta_d$, $\delta_d^{0.5}$, or another $\alpha$, as well as downstream processes
    that delivered $\alpha$ coming from $b$ such as $g$.]
    {
\begin{tikzpicture}[scale=0.9]

  \thickmuskip=0mu
  \medmuskip=0mu
  \thinmuskip=0mu
  
  \newcommand\X{115pt}
  \newcommand\Y{-60pt}

  \newcommand\ADD{\alpha}
  \newcommand\DEL{\delta}
  \newcommand\DELDEL{\Delta}

  \draw[opacity=0] (-1.4*\X, 0) -- (2.4*\X, 0);
  


  \draw[dotted] (-1.5*\X, 0pt) -- (-1.35*\X, 0pt); %% X ->
  
  \draw[->] (-1.15*\X, 0) -- (-5pt + -1*\X, 0); \draw[dotted] (-1.25*\X, 0) -- (-1.15*\X, 0);
  \draw (5pt + 2*\X, 0) -- (2.15*\X, 0); \draw[dotted,->] (2.15*\X, 0) -- (2.25*\X, 0);
  
  \draw[<->] (5pt - \X, 0.4*\Y) --
  node[below, font=\small, align=center]{\textbf{I: propagation}\\\textbf{of $\delta$ messages}}
  (-5pt, 0.4*\Y); %% I as well
  
  \draw[->] (-\X + 5pt, 0)
  node[below=-0.3em, below right, font=\scriptsize]{$\DEL_X \rightarrow$}
  -- (0 - 5pt, 0)
  node [below=-0.15em, below left, font=\tiny]{$\textcolor{\WRONG}{a \neq c}$}
  ; %% d - e

  \draw[<->] (5pt , 0.4*\Y) --
  node[below, font=\small, align=center]{\textbf{II: detection}}
  (-5pt + \X, 0.4*\Y); %% IV
  
  \draw[->] (0 +5pt, 0)
  node[below=-0.3em, below right, font=\tiny]{$\textcolor{\PY}{\ADD_Y^{y'}} \rightarrow$}
  --
  (\X -5pt, 0)
  node [left= -0.15em, below=-0.3em, below left, font=\tiny]{$\textcolor{\WRONG}{\DEL_Y^{\vphantom{y'}} \not\rightarrow \ADD_Y^{y'}}$}
  ; %% e - f
  
  \draw[<->] (5pt +\X , 0.4*\Y) --
  node[below, font=\small, align=center]{\textbf{III: propagation}\\\textbf{of $\DELDEL$ messages}}
  (-5pt + 2*\X, 0.4*\Y); %% V
  
  \draw[->] ( \X +5pt, 0) -- (2*\X -5pt, 0); %% b - g

  \draw[dotted] (0.5*\X, 1.25*\Y) -- (0.2*\X, 1.1*\Y); %% Y ->
  \draw[dotted] (0.5*\X, 1.25*\Y) -- (0.8*\X, 1.1*\Y); %% Y ->



  \draw[fill=white] (-1.5*\X, 0) node{$\bm{X}$} +(-5pt, -5pt) rectangle +(5pt, 5pt);
  \draw[fill=white] (-\X, 0) node{$\bm{a}$} +(-5pt, -5pt) rectangle +(5pt, 5pt);  
  \draw[fill=white] (0, 0) node[color=\PY]{$\bm{b}$} +(-5pt, -5pt) rectangle +(5pt, 5pt);
  \draw[fill=white] (\X, 0) node[color=\PX]{$\bm{c}$} +(-5pt, -5pt) rectangle +(5pt, 5pt);
  \draw[fill=white] (2*\X, 0) node[color=\PX]{$\bm{d}$} +(-5pt, -5pt) rectangle +(5pt, 5pt);
  \draw[fill=white] (0.5*\X, 1.25*\Y)node{$\bm{Y}$}+(-5pt, -5pt) rectangle +(5pt, 5pt);



  \draw (-1.5*\X, 5pt) node[above]{$\textcolor{\PX}{\ADD_X} \rightarrow \DEL_X$};
  
  \draw (-\X, 5pt) node[above, font=\tiny]{$\textcolor{\PX}{\ADD_X} \rightarrow \DEL_X$};
  
  \draw ( 0, 5pt) node[above, font=\footnotesize]{$
    \xrightarrow[\textcolor{\PX}{\ADD_X^{\vphantom{x'}}}]{\mbox{\tiny{before}}}
    \xrightarrow[\textcolor{\PY}{\ADD_Y^{y\vphantom{'}}}]{\mbox{\tiny{last}}}
    \xrightarrow[\DEL_Y^{\vphantom{y'}} ]{\mbox{\tiny{expect}}}$};
  
  \draw ( \X, 5pt) node[above, font=\footnotesize]{$
    \xrightarrow[\textcolor{\PY}{\ADD_Y^{\vphantom{y'}}} \rightarrow
      \DEL_Y^{\vphantom{y'}}]{\mbox{\tiny{before}}}
    \xrightarrow[\textcolor{\PX}{\ADD_X^{x'}}]{\mbox{\tiny{last}}}
    \xrightarrow[\DELDEL_X^{\vphantom{y'}}]{\mbox{\tiny{expect}}}$};

  \draw (2*\X, 5pt) node[above, font=\footnotesize]{$
    \xrightarrow[\textcolor{\PX}{\ADD_X^{x''}}]{\mbox{\tiny{last}}}
    \xrightarrow[\DELDEL_X^{\vphantom{x'}}]{\mbox{\tiny{expect}}}$};

  \draw (0.5*\X, 1.25*\Y+5pt) node[above] {$\textcolor{\PY}{\ADD_Y} \rightarrow \DEL_Y$};
  


  \draw[->] (-\X,  0.5*\Y) -- (-\X, -5pt);
  \draw[dotted] (-\X,  0.5*\Y) -- (-\X, 0.9*\Y);

  \draw[->] ( 0,  0.5*\Y) -- ( 0, -5pt);
  \draw[dotted] ( 0,  0.5*\Y) -- ( 0, 0.9*\Y);

  \draw[->] ( \X,  0.5*\Y) -- ( \X, -5pt);
  \draw[dotted] ( \X,  0.5*\Y) -- ( \X, 0.9*\Y);

  \draw[->] (2*\X,  0.5*\Y) -- (2*\X, -5pt);
  \draw[dotted] (2*\X,  0.5*\Y) -- (2*\X, 0.9*\Y);
  
\end{tikzpicture}
}
    \caption{\label{fig:proof}Dynamic partitioning leads to correctness issues due to
      staleness and ordering of operations.}
  \end{center}
\end{figure*}



%% In this section, we aim at solving Problem Statement~\ref{prob:intra}
%% by proposing \NAME, a reactive logical partitioning protocol for
%% dynamic partitioning in dynamic networks.

\subsection{Principle}
At any time, a process can become a source, hence adding a new
partition to the system. This partition eventually includes all
processes that are closer from this new source than any other else. We
described such protocol in Section~\ref{sec:scoped}. Processes
naturally converge towards their respective best partition by only
piggybacking a monotonically increasing distance in forwarded
messages. % \TODO{Traffic of each partition is contained to the
%  partition.}

Then, at any time, a source can revoke its self-appointed status of
source, hence deleting its partition from the system. All processes
that belong to this partition must eventually choose another partition
to belong to. Unfortunately, stale control information about deleted
partitions may impair the propagation of both
\begin{inparaenum}[(i)]
\item notifications about actual sources, and
\item notifications about other deleted partitions.
\end{inparaenum}
The correctness of the partitioning depends on receipt orders at each
processes.

\begin{definition}[Happens-before $\rightarrow$~\cite{lamport1978time}]
  The transitive, irreflexive, and antisymmetric happens-before
  relationship defines a strict partial order between events. Two
  messages are concurrent if none happens before the other.
\end{definition}

We define a last win order to forbid the delivery of stale information
emanating from a same process. In other terms, the \emph{best}
notification message is always the most recent known of each process.
 
\begin{definition}[Last win order]
  When a process $p$ broadcasts two messages $b_p(m) \rightarrow
  b_p(m')$, no process $q$ can deliver $m$ if it has delivered $m'$:
  $\not\exists q \in V$ with $d_q(m') \rightarrow d_q(m)$.
\end{definition}

Figure~\ref{fig:proof} depicts the issues with staleness and message
orderings. In Figure~\ref{fig:proofA}, the shortest path from any
source to Process $c$ is $[a, b, c]$. However, Process $b$ still holds
a stale $\alpha_d^{0.5}$ without knowing. When it receives
$\alpha_a^1$, it discards it, for it assumes that downstream processes
are more interested in $\alpha_d^{0.5}$. To reach consistent
partitioning, Process $b$ first needs to purge its current partition
to later accept that of its current actual shortest path:
$\alpha_a^1$.

\noindent Figure~\ref{fig:proofB} shows that removing stale control
information is even more complex. The removal must reach all processes
of the previous shortest path going from $d$ to $b''$. Label I' shows
the most obvious issue where Process $e$ changed partition for a
better but stale $\alpha_f$. Since it can remember its previous
deliveries, it could still forward $\delta_d$ for the sake of
consistency. However, this would lead to every process forwarding
every $\delta$ they ever delivered. Such protocol's overhead would
depend on past partitioning instead of current one. Label III shows
the issue when the blocking partition is already known to be stale at
Process $b'$. Process $b''$ eventually receives $\alpha_f$ from
$e$. However, it cannot deliver it, for it would break last win order.
Process $b'$ may be in an inconsistent state. Label IV shows the
corollary issue: Process $b''$ delivered a message from a process that
may be inconsistent without knowing it.

\begin{theorem}[\label{theo:dcp}Dynamic consistent partitioning]
  Eventual consistent partitioning in presence of dynamic sources
  requires
  \begin{inparaenum}[(i)]
  \item eventual purging of stale notifications, and
  \item eventual delivery of bests partitions.
  \end{inparaenum}
\end{theorem}

\begin{proof}
  \TODO{Meow meow meow.}
  \TODO{Label I: default propagation to what seems the border of the partition but isn't.}
  \TODO{Label II: gates eventually break because of I', III, IV.}
  \TODO{Label I': as Label I.}
  \TODO{Label III: detecting possible inconsistent partitioning.}
  \TODO{Label IV: propagation of this issue to all and only concerned parties.}
\end{proof}



\subsection{Implementation}

\begin{algorithm}
  
\SetKwProg{Function}{func}{}{}

\small

\DontPrintSemicolon
\LinesNumbered

$O_p$, $W_p$ \tcp*[r]{set of neighbors and weights}
$V \leftarrow \varnothing$ \tcp*[r]{vector of versions}
$c \leftarrow 0$ \tcp*[r]{local counter}
$best \leftarrow \varnothing$ \tcp*[r]{best $\alpha$ so far}
%% $bests$ \tcp*[r]{last delivered $\alpha$ of $p$ and neighbors}

\BlankLine
\BlankLine

\Function{\textup{Add ( )} \tcp*[f]{$\alpha_{p, c}^{0, \varnothing} $}} {
  $c \leftarrow c + 1$ \;
  \textup{receiveAdd($p$, $p$, $c$, $0$, $\varnothing$)}
}

\Function{\textup{Del ( )} \tcp*[f]{$\delta_{p, c}$}} {
  $c \leftarrow c + 1$ \;
  \textup{receiveDel($p$, $p$, $c$)}
}

\BlankLine
\BlankLine

\Function{\textup{receiveAdd($q$, $s'$, $c'$, $d'$, $\pi'$)}
\tcp*[f]{\smash{$r_p(\alpha_{s', c'}^{d', \pi'})$}}  \textup{\texttt{from}} $q$} {
  \If {$p \not\in \pi'$ \label{line:notloopingA}} {
      \uIf {$c' \geq V[s']$ \label{line:detectA}} {
          $V[s'] \leftarrow c'$ \;
          \If {$best \leq_\alpha \langle s', c', d', \pi'\rangle$} {
              $best \leftarrow \langle s', c', d', \pi'\rangle$ \;
              \ForEach{$n \in O_p$}
                  {send$_n$($s'$, $c'$, $d' + W_{pq}$, $\pi' \cup p$)}
           }
      } \uElseIf {$q = best.\pi[|best.\pi| - 1] \wedge q \in O_p$ \label{line:detectB}} {
          \textup{receiveDel($q$, $best$)} \tcp*[r]{undo}
      }
      
   }
}

\Function{\textup{receiveDel($q$, $s'$, $c'$, $\pi'$)}
\tcp*[f]{\tiny \smash{$r_p (\delta_{s', c'})$ \textup{\texttt{or}} $r_p (\delta_{s', c'}^{\pi'})$}}} {
  \If {$p \not\in \pi'$ \label{line:notloopingB}} {
     $V[s'] \leftarrow \max(V[s'], c')$ \;
     \uIf {$\langle s', c', \pi' \rangle \overset{\alpha}{=} best $} {
         $best \leftarrow \varnothing$ \;
        \ForEach{$n \in O_p$} {
            \lIf{$\pi' = \varnothing$} {send$_q$($s'$, $c'$)}
            \lElse{send$_n$($s'$, $c'$, $\pi' \cup p$)}}
     } \uElseIf {$best \neq \varnothing \wedge q \in O_p$} {
         \textbf{let} $\langle s, c, d, \pi \rangle \leftarrow best$ \;
         \textup{send$_q$($s, c, d + W_{pq}, \pi \cup p$)} \tcp*[r]{compete}
     }
  }
}



  \caption{\label{algo:adddelundo}Dynamic partitioning by Process $p$.}
\end{algorithm}

Algorithm~\ref{algo:adddelundo} provides the instructions that
implement dynamic consistent partitioning:
\begin{inparaenum}[(i)]
\item \label{algo:most} most of the time, it propagates $\alpha$'s and
  $\delta$'s when the partitions allow it;
\item \label{algo:sometimes} sometimes, it detects possible
  inconsistent partitioning and
\item \label{algo:solves}solves it using propagation trees (as opposed
  to propagation graphs);
\item \label{algo:competition} when required, it triggers competitions
  among neighboring partitions.
\end{inparaenum}
While (\ref{algo:most})-(\ref{algo:solves}) tackle the eventual purging of
stale notifications, (\ref{algo:competition}) tackle the eventual
delivery of the best up-to-date partitions.

\paragraph{Dynamic partitions.}
We extend Algorithm~\ref{algo:add} to enable each source to revoke its
self-appointed status of source and become a simple process again. In
that regard, we enforce the delivery order of messages at each
process.

To implement last win order, each process maintains a vector of
versions that associates to each known source, or has-been source, its
known local counter. It enables processes to deliver only up-to-date
$\alpha$'s that may improve their best known partition (see
Line~\ref{line:detectA}), but also to detect possible inconsistent
partitioning (see Line~\ref{line:detectB}) as labeled by III in
Figure~\ref{fig:proofB}. From a global perspective, it ensures
monotonic increase of knowledge despite the non-monotonic number of
sources. In turns, it contributes to convergence and termination, for
corresponding $\alpha$ and $\delta$ cannot follow each other in an
infinite loop.

Implementing the propagation of deletes is no different from the
propagation of adds as hinted in Figure~\ref{fig:proof} with Labels I
and I'. Messages and comparisons include the local counter of the
source. When a process receives a better up-to-date partition, it
delivers and forwards it. When a process receives a delete
notification about its partition, it delivers and forwards it. It is
worth noting that such $\delta$'s apply to the whole partition. They
travel in the propagation graph of their partition using all
communication links, allowing processes to change partition as soon as
possible.

Vectors of versions also allow each process to detect possible last
win order violations. Upon possible inconsistency, the detecting
process must remove all $\alpha$ messages originating from it, as
labeled by IV in Figure~\ref{fig:proofB}. Contrarily to delete
operations, this applies to a sub-partition of the current partition,
and cannot rely on local counters to
operate. Algorithm~\ref{algo:adddelundo} implements this behavior by
piggybacking paths in messages. Paths guarantee that such $\delta$
messages apply to all and only $\alpha$'s that followed an identical
path. $\delta$'s propagate in the propagation tree rooted at the
detecting process. Local states of processes remain unchanged, but
paths in messages monotonically grow. In turns, $\alpha$ and $\delta$
cannot follow each other in an infinite loop, for a process knows when
it already received a forwarded a message (see
Lines~\ref{line:notloopingA} and \ref{line:notloopingB}).

\noindent It is worth noting that scoped broadcast and the
piggybacking of paths synergies well: Paths tend to be smaller as the
number of sources in the system increases.

Finally, as stated in Theorem~\ref{theo:dcp}, dynamic consistent
partitioning not only requires the removal of all stale partitions,
but also the retrieval of the best up-to-date partitions. In that
regard, $\delta$'s have dual use: since they already reach borders of
partitions when they remove stale control informations, they also
trigger a competition when reaching such bordering processes (see
Line~\ref{line:compete}). This simply consists in sending the current
partition through the communication link from which the process
received the $\delta$. Upon receipt of this answer, processes act
normally by propagating their changes when they improve. These adds
fill the gaps left open by deletes.

\begin{algorithm}
  \SetKwProg{Function}{func}{}{}

\SetInd{0.2em}{1em}

\small

\DontPrintSemicolon
\LinesNumbered

% \begin{multicols}{2}
\Function(\tcp*[f]{new link to $q$}){\textup{edgeUp($q$)}}  {
    \lIf {$A_{\pi}^{d} \neq \alpha_\varnothing^\infty$} {
         send$_q$($\alpha_{\pi}^{d + W_{pq}}$)
         }
}

% \BlankLine

\Function(\tcp*[f]{link to $q$ removed}){\textup{edgeDown($q$)}} {
  \lIf {\textup{isParent($q$)}} {
       receiveDel($q$, $\delta_{p, V[p]+1}$)
  }
}

% \end{multicols}

% \BlankLine

  \caption{\label{algo:edges}Dynamic partitioning by Process $p$ in dynamic networks.}
\end{algorithm}

\paragraph{Dynamic network.}
Algorithm~\ref{algo:edges} provides the instructions of \NAME that
enable dynamic consistent partitioning in dynamic networks where
processes can join or leave the system without
notification. \TODO{More.}

%% Adding new communication links to the network may create shortcuts
%% between processes. Both processes must send their current best
%% partition to each other. Upon receipt, they act normally: if a process
%% finds out that the received partition is closer than its current one,
%% it delivers it which in turns also triggers another competition
%% amongst neighbors due to forwarding.

%% \noindent Joining the network is equivalent to add as many
%% communication links as necessary between the joining process and its
%% new neighbors.

%% When removing a communication link between two processes does not
%% break any active path, because neither distances of processes depend
%% on the other, then nothing needs to be done. \NAME has no overhead.
%% Unfortunately, when a process' distance depends on the other process,
%% the protocol becomes much more complex. Indeed, this requires to
%% \TODO{undo} all add messages originated from this process. A message
%% must convey the fact that
%% \begin{inparaenum}[(i)]
%% \item an edge at a particular process has been removed, and
%% \item the distance that has been delivered by a process comes from
%%   this particular process.
%% \end{inparaenum}

\paragraph{Another trade-off.}
We propose an implementation that further decreases generated traffic
and improves on anonymity~\cite{whitaker2002forwarding}.  \NAME makes
extensive use of paths to enable deleting messages without
incrementing local counters. Messages carry their respective path and
processes detect when such messages have been looping when they carry
their identity (see Lines~\ref{line:notloopingA} and
\ref{line:notloopingB} of Algorithm~\ref{algo:adddelundo}). Since
membership is all that matters, we propose to trade vectors of
identities for Bloom filters~\cite{almeida2007scalable}. Processes
know with high probability if they already received and forwarded each
message without knowing the identity of all processes that received
it.  False positive probability only incurs premature stopping of
broadcast messages and does not invalidate the delete of specific
messages.



%% \subsection{Inter-autonomous system partitioning}

%% \begin{figure}
%%   \centering\begin{tikzpicture}[scale=0.92]

  \thickmuskip=0mu
  \medmuskip=0mu
  \thinmuskip=0mu
  
  \newcommand\X{50pt}
  \newcommand\Y{-50pt}

  \newcommand\ADD{\alpha}



  \draw (-1.5*\X, 0 - 5pt)
  node[below, circle, draw=black, fill=white, scale=0.35]{\large{0}}
  --  
  (-1.5*\X, \Y + 5pt)
  node[above, circle, draw=black, fill=white, scale=0.35]{\large{0}}; %% a - e

  \draw (-1.5*\X + 5pt, 0)
  node[right, circle, draw=black, fill=white, scale=0.35]{\large{1}}
  --
  (-1*\X, 0);

  \draw[dotted] (-1*\X, 0) -- (-0.5*\X, 0);
  
  \draw (-0.5*\X, 0)
  --  
  (0 - 5pt, 0)
  node[left, circle, draw=black, fill=white, scale=0.35]{\large{2}}; %% a - b

  \draw (0 +5pt, 0)
  node[right, circle, draw=black, fill=white, scale=0.35]{\large{1}}
  --
  (\X -5pt, 0)
  node[left, circle, draw=black, fill=white, scale=0.35]{\large{1}}; %% b - c
  
  \draw (0, 0 - 5pt)
  node[below, circle, draw=black, fill=white, scale=0.35]{\large{0}}
  --
  (0, \Y + 5pt)
  node[above, circle, draw=black, fill=white, scale=0.35]{\large{0}}; %% b - d

  \draw (\X + 3pt, 0 - 5pt)
  node[below left, circle, draw=black, fill=white, scale=0.35]{\large{0}}
  --
  (0 + 5pt, \Y - 3pt)
  node[above right, circle, draw=black, fill=white, scale=0.35]{\large{0}}; %% c - d



  \draw[dashed] (-0.75*\X, -0.5*\Y) -- (-0.75*\X, 1.5*\Y)
  node[above left]{\textit{AS 1}}
  node[above right]{\textit{AS 2}};

  \draw[fill=white] (-1.5*\X, 0) node[color=\PA]{$\bm{a}$} +(-5pt, -5pt) rectangle +(5pt, 5pt);
  \draw [left] (-1.5*\X - 5pt, 0) node{\textbf{e: 1}};
  \draw[fill=white] (0, 0) node[color=\PA]{$\bm{b}$} +(-5pt, -5pt) rectangle +(5pt, 5pt);
  \draw [above] ( 0, 5pt ) node{\textbf{e: 2}};
  \draw[fill=white] (\X, 0) node{$\bm{c}$} +(-5pt, -5pt) rectangle +(5pt, 5pt);
  \draw[fill=white] (0, \Y) node{$\bm{d}$} +(-5pt, -5pt) rectangle +(5pt, 5pt);

  \draw[fill=white] (-1.5*\X, \Y) node[color=\PA]{$\bm{e}$} +(-5pt, -5pt) rectangle +(5pt, 5pt);
  \draw [left] (-1.5*\X - 5pt, \Y) node{\textbf{e: 0}};
  
  
  % \draw (-\X, 5pt) node[above, color=\PA]{$\ADD_a^0$};
  % \draw (\X, 5pt) node[above, color=\PC]{$\ADD_c^0$};


\end{tikzpicture}

%%   \caption{\label{fig:AS}Inter autonomous systems
%%     partitioning. \TODO{More about relative ordering of links at each
%%       process}}
%% \end{figure}

%% In this section, we aim at solving Problem Statement~\ref{prob:inter}
%% by proposing \NAMEB, an extension of \NAME, that enables dynamic
%% logical partitioning in inter-autonomous systems. It leverages
%% hierarchical properties of these networks to further improve on scoped
%% flooding.

%% Autonomous systems are geo-distributed networks. Heterogeneous
%% communication links. Hierarchy of communication links. Per-object
%% broadcast.

%% We leave the link handling to membership protocols~\REF. The ordering
%% of links can be done based on latency. For instance, from 0 to 100 ms,
%% rank 0, from 100 to 1s, rank 2 \ldots

%% \begin{algorithm}
%%   \SetKwProg{Function}{func}{}{}

\small

\DontPrintSemicolon
\LinesNumbered

$\mathcal{A} \leftarrow \varnothing$ \tcp*[r]{map each object to its \TODO{\NAME}}
$\mathcal{O}_p$ \tcp*[r]{sorted list of sets of links}

\BlankLine
\BlankLine

\Function{\textup{create($o$)}} {
    initObject($o$) \;
    \lForEach {$a \in \mathcal{A}[o]$} {$a$.Add( )}
}

\Function{\textup{remove($o$)}} {
    \lForEach {$a \in \mathcal{A}[o]$} {$a$.Del( )}
}

\BlankLine
\BlankLine

\Function{\textup{receive($q$, $o$, $m$) }} {
    initObject($o$) \;

    \textbf{let} $i$ \textbf{with} $q \in \mathcal{O}^i_p$ \textbf{or} $0$
    \tcp*[r]{$i^{th}$ set of links}

    \For{$j$ \textbf{\textup{with}} $i \leq j < |\mathcal{A}[o]|$}  {
       \textbf{let} $a \leftarrow \mathcal{A}[o][j]$ \;
       \lIf {$m \in \alpha$} {$a$.receivedAdd($q$, $m$)}
       \lIf {$m \in \delta$} {$a$.receivedDel($q$, $m$)}
    }
}

\BlankLine
\BlankLine

\Function{\textup{initObject($o$)} \tcp*[f]{utility}} {
    \If {$o \not\in \mathcal{A}$} {
        $\mathcal{A}[o] \leftarrow \varnothing$ \;
        \ForEach {$\mathcal{O}_p^r \in \mathcal{O}_p$} {
           $\mathcal{A}[o] \leftarrow \mathcal{A}[o] \cup \{ \TODO{A^4}(\mathcal{O}_p^r) \} $
        }
    }

}

% \Function{\textup{newObject($o$)}} {

% \Function{\textup{receive($q$, $m$)}} {
%    \If {$m \in \alpha$} {
%       \textbf{let} $best \leftarrow a.best$ \;
%       $a$.receiveAdd($q$, $m$) \;
%       \If {$best = a.best$} {
%       \TODO{from higher $q$ to min $q$} \;
%       broadcast($q$, $a.best$)
%       }
%    }
% }

% \Function{\textup{broadcast($q$, $m$)}} {

%   \textbf{let} $i$ \textbf{\textup{with}} $q \in \mathcal{O}^i_p$ \textbf{or} $0$
%   \tcp*[r]{$i^{th}$ set of links}

%   \ForEach{$q \in O_p^j$ \textbf{\textup{with}} $i \leq j < |\mathcal{O}_p|$}  {
%         \textup{sendTo$_q$($m +_\alpha w_{pq}$)}
%   }
  
% }

% }
%%   \caption{\label{algo:aaaa}\NAMEB running at Process $p$ for dynamic
%%     partitioning in inter-autonomous systems.}
%% \end{algorithm}

%% Figure~\ref{fig:AS} shows an example of inter-AS logical partitioning
%% where each process ranks its communication links, and each process
%% forwards messages to its neighbors starting from the rank of the link
%% that triggered the receipt, to its highest rank neighbors. In this
%% example, Process $e$ adds a partition. It forwards the generated
%% message to all its neighbors as if it received it from its lowest rank
%% links. When Process $a$ receives $\alpha_e^1$ from its rank-$0$ link,
%% it forwards it to rank-$0$ links and rank-$1$ links.  Process $b$
%% stops the propagation, for all its links have lower ranks than the one
%% from which it received the message $\alpha_e^2$. Ultimately, the
%% partition includes Process $e$, Process $a$, and Process $b$. Process
%% $c$ and Process $d$ remain unaware of the new \TODO{object}.

%% To ease the reasoning about inter-AS dynamic partitioning in dynamic
%% networks, \NAME runs an independent broadcast protocol for each
%% object, and for each rank of links.  \TODO{Not so easy\ldots}
%% \TODO{Write and describe algo.} \TODO{Overhead of separating things,
%%   slightly more memory, but traffic wise? slower ?}  \TODO{In the
%%   example, not only Process $b$ keeps $e_1$ as its best partition, but
%%   it keeps it for links of rank 0, and links of rank 1.}  When there
%% are no ranking in links, processes solely rely on
%% Algorithm~\ref{algo:adddelundo} for optimal partitioning.





\subsection{Complexity}

We focus on average-case and worst-case complexity. We divide our
analysis into space, time, and communication complexity.

\paragraph{Space.}
Each process needs to maintain a vector of versions the size of which
depends on the number of sources. In the worst case, the size of this
vector increases linearly with the number of distinct sources that
ever existed in the system. For the sake of last win order, processes
can never purge this vector from any entry. \TODO{Even when the
  process leaves the system.} Fortunately, scoped broadcast limits
message propagation to interested processes. A process at a side of
the system may never hear of a source at the other side of the
system. In turns, it may never have to maintain an entry for this
source. \TODO{This is application-dependent?}

\noindent Each process must also maintain the partition it belongs to,
along with its control informations. Among other, it must maintain a
path the size of which depends on the membership size of the
partition. In the worst case, this path includes the identity of every
process of the system that currently belong to the system. Eventually,
the size of this path increases linearly with the diameter of the
current partition. \TODO{assuming weight equals 1 \ldots}

\paragraph{\TODO{Time.}}
In terms of processing time, whatever the operation, it increases
linearly with the number of neighbors of the process. It scales well,
for the number of neighbors is commonly order of magnitude lower than
the number of processes in the system. For instance, random networks
only require $O(\log(|V|))$ neighbors to work properly \REF; and
processes in autonomous systems have in average \TODO{X} neighbors
\REF. Lines~\ref{line:notloopingA} and \ref{line:notloopingB} can
prove costly depending on the data structure encoding paths. Using
Bloom filters however, this is \TODO{efficient}: it depends on the
number of buckets, and the number of hash functions.

\noindent In terms of convergence time, every $\alpha$ needs to travel
at least its shortest path before termination. Every $\delta$ needs to
reach the border of its partition, triggering an $\alpha$ that fills
the gaps left open by the removal. \TODO{More.}

\paragraph{\TODO{Communication.}}
In terms of number of messages required to reach optimal
partitioning. In the average-case, a process $i$ chosen uniformly at
random among all processes creates a logical partition. Its messages
$\alpha_i$ propagate through the network until reaching processes that
belong to another partition closer to them. This splits partitions in
half in average. Overall, the $a^{th}$ new partition comprises
\smash{$\mathcal{O}(\frac{|V|}{2^{\lfloor \log_2 a \rfloor}})$}
processes. This decreases every new partition until reaching $0$
processes per new partition: even the chosen process already belongs
to its optimal partition. The average number of messages per process
is \smash{$\mathcal{O}(\frac{\overline{|O|}}{2^{\lfloor \log_2 a
      \rfloor}})$}. \TODO{Multiple receipt and multiple delivery imply
  more messages (receipt bounded by $|O|$ as well).} Deleting the
$a^{th}$ partition generates the exact same number of messages than
the $a^{th}$ partition creation. \TODO{But what about echos?} In the
worst-case, every new partition includes all but one process belonging
to the previous partition. The total number of messages after the
$a^{th}$ new partition is $\mathcal{O}(\overline{|O|}\cdot a^2)$. As
for the average-case, the number of messages for the partition
deletion is identical to the number of messages of the corresponding
partition creation.

\noindent \TODO{In terms of size of messages.}


%%% Local Variables: 
%%% mode: latex
%%% TeX-master: "../paper"
%%% ispell-local-dictionary: "english"
%%% End: 
