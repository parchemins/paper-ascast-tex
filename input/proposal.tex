
\section{\NAME for Dynamic Partitioning}
\label{sec:proposal}

In this section, we introduce \NAME (\TODO{stands for}) and its
\underline{ex}tension \NAMEB, reactive protocols for logical
partitioning in autonomous systems the cost of which actually depends
on their usage: when the system becomes quiescent, processes
eventually converge to their respective partition and do not require
further communication afterward.

Each process is stateless, works autonomously and asynchronously, in a
peer-to-peer fashion. Each process broadcasts each change to its
neighborhood, for the state of its neighbors may depend on this change
as well. Yet, generated traffic remains \TODO{low} using scoped
flooding. Messages that carry changes travel through the network
depending on partition of processes, stopping as soon as they
encounter uninterested processes. We demonstrate that \NAME guarantees
consistent partitioning despite different order in message deliveries
from one process to another. While \NAME is a general purpose logical
partitioning protocol for dynamic networks, \NAMEB further reduces
generated traffic in the context of inter-autonomous systems by
leveraging the hierarchical properties of such networks. 

This section starts by describing \NAME's functioning by detailing its
operation to avoid inconsistent states due to ordering and staleness
of delivered messages. Then, it describes its extension \NAMEB that
builds multiple broadcast layers ordered using local knowledge only.
Finally, it ends with a complexity analysis of proposed protocols. 



\subsection{Intra-autonomous system partitioning}

\begin{algorithm}
  
\SetKwProg{Function}{func}{}{}

\small

\DontPrintSemicolon
\LinesNumbered

$O_p$, $W_p$ \tcp*[r]{set of neighbors and weights}
$V \leftarrow \varnothing$ \tcp*[r]{vector of versions}
$c \leftarrow 0$ \tcp*[r]{local counter}
$best \leftarrow \varnothing$ \tcp*[r]{best $\alpha$ so far}
%% $bests$ \tcp*[r]{last delivered $\alpha$ of $p$ and neighbors}

\BlankLine
\BlankLine

\Function{\textup{Add ( )} \tcp*[f]{$\alpha_{p, c}^{0, \varnothing} $}} {
  $c \leftarrow c + 1$ \;
  \textup{receiveAdd($p$, $p$, $c$, $0$, $\varnothing$)}
}

\Function{\textup{Del ( )} \tcp*[f]{$\delta_{p, c}$}} {
  $c \leftarrow c + 1$ \;
  \textup{receiveDel($p$, $p$, $c$)}
}

\BlankLine
\BlankLine

\Function{\textup{receiveAdd($q$, $s'$, $c'$, $d'$, $\pi'$)}
\tcp*[f]{\smash{$r_p(\alpha_{s', c'}^{d', \pi'})$}}  \textup{\texttt{from}} $q$} {
  \If {$p \not\in \pi'$ \label{line:notloopingA}} {
      \uIf {$c' \geq V[s']$ \label{line:detectA}} {
          $V[s'] \leftarrow c'$ \;
          \If {$best \leq_\alpha \langle s', c', d', \pi'\rangle$} {
              $best \leftarrow \langle s', c', d', \pi'\rangle$ \;
              \ForEach{$n \in O_p$}
                  {send$_n$($s'$, $c'$, $d' + W_{pq}$, $\pi' \cup p$)}
           }
      } \uElseIf {$q = best.\pi[|best.\pi| - 1] \wedge q \in O_p$ \label{line:detectB}} {
          \textup{receiveDel($q$, $best$)} \tcp*[r]{undo}
      }
      
   }
}

\Function{\textup{receiveDel($q$, $s'$, $c'$, $\pi'$)}
\tcp*[f]{\tiny \smash{$r_p (\delta_{s', c'})$ \textup{\texttt{or}} $r_p (\delta_{s', c'}^{\pi'})$}}} {
  \If {$p \not\in \pi'$ \label{line:notloopingB}} {
     $V[s'] \leftarrow \max(V[s'], c')$ \;
     \uIf {$\langle s', c', \pi' \rangle \overset{\alpha}{=} best $} {
         $best \leftarrow \varnothing$ \;
        \ForEach{$n \in O_p$} {
            \lIf{$\pi' = \varnothing$} {send$_q$($s'$, $c'$)}
            \lElse{send$_n$($s'$, $c'$, $\pi' \cup p$)}}
     } \uElseIf {$best \neq \varnothing \wedge q \in O_p$} {
         \textbf{let} $\langle s, c, d, \pi \rangle \leftarrow best$ \;
         \textup{send$_q$($s, c, d + W_{pq}, \pi \cup p$)} \tcp*[r]{compete}
     }
  }
}



  \caption{\label{algo:adddelundo}Dynamic partitioning by Process $p$.}
\end{algorithm}


\begin{figure*}
  \begin{center}
    \subfloat[Part A][\label{fig:proofA}Stale $\alpha$'s may stop up-to-date $\alpha$'s
    from reaching all processes that require it along the shortest path from $a$ to $c$.
    To solve this issue, we must guarantee
    the eventual removal of stale $\alpha$'s (see Figure~\ref{fig:proofB}).]
    {
\begin{tikzpicture}[scale=0.87]

  \thickmuskip=0mu
  \medmuskip=0mu
  \thinmuskip=0mu
  
  \newcommand\X{95pt}
  \newcommand\Y{-60pt}

  \newcommand\ADD{\alpha}
  \newcommand\DEL{\delta}

  \draw[opacity=0] (-1.2*\X, 0) -- (1.2*\X, 0);
  


  \draw[->] (-1.15*\X, 0) -- (-5pt + -1*\X, 0); \draw[dotted] (-1.25*\X, 0) -- (-1.15*\X, 0);
  \draw (5pt + \X, 0) -- (1.15*\X, 0); \draw[dotted,->] (1.15*\X, 0) -- (1.25*\X, 0);
  
  \draw[<->] (5pt - \X, 0.4*\Y) -- node[below, font=\small, align=center]{\textbf{I: propagation}\\\textbf{of $\alpha$ messages}} (-5pt, 0.4*\Y); %% I
  
  \draw[->] (-\X + 5pt, 0)
  node[below=-0.3em, below right, font=\tiny]{$\textcolor{\PA}{\ADD_a^{x'}} \rightarrow$}
  --
  (0 - 5pt, 0); %% a - b

  \draw[<->] (5pt, 0.4*\Y) -- node[below, align=center, font=\small]{\textbf{II: need purge}\\with I', III, IV} (-5pt + \X, 0.4*\Y); %% II
  \draw[->] (0 + 5pt, 0) -- (-5pt +  \X, 0); %% b - c


  
  \draw[fill=white] (-\X, 0) node[color=\PA]{$\bm{a}$} +(-5pt, -5pt) rectangle +(5pt, 5pt);  
  \draw[fill=white] (0, 0) node[color=\PD]{$\bm{b}$} +(-5pt, -5pt) rectangle +(5pt, 5pt);
  \draw[fill=white] (\X, 0) node{$\bm{c}$} +(-5pt, -5pt) rectangle +(5pt, 5pt);


  
  \draw (-\X, 5pt) node[above, color=\PA]{$\ADD_a^x$};
  \draw ( 0, 5pt) node[above]{$
    \xrightarrow[\textcolor{\PD}{\ADD_d^y}]{\mbox{\small{last}}}
    \xrightarrow[\DEL_d \rightarrow \textcolor{\PA}{\ADD_a^{x'}}]{\mbox{\small{expect}}}$};
  
  \draw (-5pt, 0pt) node [below=-0.3em, below left, font=\tiny]{$\textcolor{\WRONG}{\ADD_d^y < \ADD_a^{x'}}$};
  % \draw (-5pt, 0pt) node [below=-0.3em, below left, font=\tiny]{$\textcolor{\WRONG}{\ADD_d^{y} < \ADD_e^z}$};

  
  \draw (\X, 5pt) node[above]{$
    \xrightarrow[\textcolor{\PA}{\ADD_a^{x''}}]{\mbox{\small{expect}}}$};
%  \draw (0, -5pt) node[below, font=\scriptsize]{expect $\DEL_d \rightarrow \textcolor{\PA}{\ADD_a}$};


  
  \draw[->] (-\X,  0.5*\Y) -- (-\X, -5pt);
  \draw[dotted] (-\X,  0.5*\Y) -- (-\X, 0.9*\Y);

  \draw[->] ( 0,  0.5*\Y) -- ( 0, -5pt);
  \draw[dotted] ( 0,  0.5*\Y) -- ( 0, 0.9*\Y) node[below, font=\scriptsize]{some \process somewhere: $\textcolor{\PD}{\ADD_d} \rightarrow \DEL_d$};

  \draw[->] ( \X,  0.5*\Y) -- ( \X, -5pt);
  \draw[dotted] ( \X,  0.5*\Y) -- ( \X, 0.9*\Y);

  
\end{tikzpicture}
}
    \hspace{10pt}
    \subfloat[Part B][\label{fig:proofB}Stale $\alpha$'s may stop $\delta$'s from reaching
    processes with targeted $\alpha$'s. To ensure correctness, $b$ must either
    deliver $\delta_d$, $\delta_d^{0.5}$, or another $\alpha$, as well as downstream processes
    that delivered $\alpha$ coming from $b$ such as $g$.]
    {
\begin{tikzpicture}[scale=0.9]

  \thickmuskip=0mu
  \medmuskip=0mu
  \thinmuskip=0mu
  
  \newcommand\X{115pt}
  \newcommand\Y{-60pt}

  \newcommand\ADD{\alpha}
  \newcommand\DEL{\delta}
  \newcommand\DELDEL{\Delta}

  \draw[opacity=0] (-1.4*\X, 0) -- (2.4*\X, 0);
  


  \draw[dotted] (-1.5*\X, 0pt) -- (-1.35*\X, 0pt); %% X ->
  
  \draw[->] (-1.15*\X, 0) -- (-5pt + -1*\X, 0); \draw[dotted] (-1.25*\X, 0) -- (-1.15*\X, 0);
  \draw (5pt + 2*\X, 0) -- (2.15*\X, 0); \draw[dotted,->] (2.15*\X, 0) -- (2.25*\X, 0);
  
  \draw[<->] (5pt - \X, 0.4*\Y) --
  node[below, font=\small, align=center]{\textbf{I: propagation}\\\textbf{of $\delta$ messages}}
  (-5pt, 0.4*\Y); %% I as well
  
  \draw[->] (-\X + 5pt, 0)
  node[below=-0.3em, below right, font=\scriptsize]{$\DEL_X \rightarrow$}
  -- (0 - 5pt, 0)
  node [below=-0.15em, below left, font=\tiny]{$\textcolor{\WRONG}{a \neq c}$}
  ; %% d - e

  \draw[<->] (5pt , 0.4*\Y) --
  node[below, font=\small, align=center]{\textbf{II: detection}}
  (-5pt + \X, 0.4*\Y); %% IV
  
  \draw[->] (0 +5pt, 0)
  node[below=-0.3em, below right, font=\tiny]{$\textcolor{\PY}{\ADD_Y^{y'}} \rightarrow$}
  --
  (\X -5pt, 0)
  node [left= -0.15em, below=-0.3em, below left, font=\tiny]{$\textcolor{\WRONG}{\DEL_Y^{\vphantom{y'}} \not\rightarrow \ADD_Y^{y'}}$}
  ; %% e - f
  
  \draw[<->] (5pt +\X , 0.4*\Y) --
  node[below, font=\small, align=center]{\textbf{III: propagation}\\\textbf{of $\DELDEL$ messages}}
  (-5pt + 2*\X, 0.4*\Y); %% V
  
  \draw[->] ( \X +5pt, 0) -- (2*\X -5pt, 0); %% b - g

  \draw[dotted] (0.5*\X, 1.25*\Y) -- (0.2*\X, 1.1*\Y); %% Y ->
  \draw[dotted] (0.5*\X, 1.25*\Y) -- (0.8*\X, 1.1*\Y); %% Y ->



  \draw[fill=white] (-1.5*\X, 0) node{$\bm{X}$} +(-5pt, -5pt) rectangle +(5pt, 5pt);
  \draw[fill=white] (-\X, 0) node{$\bm{a}$} +(-5pt, -5pt) rectangle +(5pt, 5pt);  
  \draw[fill=white] (0, 0) node[color=\PY]{$\bm{b}$} +(-5pt, -5pt) rectangle +(5pt, 5pt);
  \draw[fill=white] (\X, 0) node[color=\PX]{$\bm{c}$} +(-5pt, -5pt) rectangle +(5pt, 5pt);
  \draw[fill=white] (2*\X, 0) node[color=\PX]{$\bm{d}$} +(-5pt, -5pt) rectangle +(5pt, 5pt);
  \draw[fill=white] (0.5*\X, 1.25*\Y)node{$\bm{Y}$}+(-5pt, -5pt) rectangle +(5pt, 5pt);



  \draw (-1.5*\X, 5pt) node[above]{$\textcolor{\PX}{\ADD_X} \rightarrow \DEL_X$};
  
  \draw (-\X, 5pt) node[above, font=\tiny]{$\textcolor{\PX}{\ADD_X} \rightarrow \DEL_X$};
  
  \draw ( 0, 5pt) node[above, font=\footnotesize]{$
    \xrightarrow[\textcolor{\PX}{\ADD_X^{\vphantom{x'}}}]{\mbox{\tiny{before}}}
    \xrightarrow[\textcolor{\PY}{\ADD_Y^{y\vphantom{'}}}]{\mbox{\tiny{last}}}
    \xrightarrow[\DEL_Y^{\vphantom{y'}} ]{\mbox{\tiny{expect}}}$};
  
  \draw ( \X, 5pt) node[above, font=\footnotesize]{$
    \xrightarrow[\textcolor{\PY}{\ADD_Y^{\vphantom{y'}}} \rightarrow
      \DEL_Y^{\vphantom{y'}}]{\mbox{\tiny{before}}}
    \xrightarrow[\textcolor{\PX}{\ADD_X^{x'}}]{\mbox{\tiny{last}}}
    \xrightarrow[\DELDEL_X^{\vphantom{y'}}]{\mbox{\tiny{expect}}}$};

  \draw (2*\X, 5pt) node[above, font=\footnotesize]{$
    \xrightarrow[\textcolor{\PX}{\ADD_X^{x''}}]{\mbox{\tiny{last}}}
    \xrightarrow[\DELDEL_X^{\vphantom{x'}}]{\mbox{\tiny{expect}}}$};

  \draw (0.5*\X, 1.25*\Y+5pt) node[above] {$\textcolor{\PY}{\ADD_Y} \rightarrow \DEL_Y$};
  


  \draw[->] (-\X,  0.5*\Y) -- (-\X, -5pt);
  \draw[dotted] (-\X,  0.5*\Y) -- (-\X, 0.9*\Y);

  \draw[->] ( 0,  0.5*\Y) -- ( 0, -5pt);
  \draw[dotted] ( 0,  0.5*\Y) -- ( 0, 0.9*\Y);

  \draw[->] ( \X,  0.5*\Y) -- ( \X, -5pt);
  \draw[dotted] ( \X,  0.5*\Y) -- ( \X, 0.9*\Y);

  \draw[->] (2*\X,  0.5*\Y) -- (2*\X, -5pt);
  \draw[dotted] (2*\X,  0.5*\Y) -- (2*\X, 0.9*\Y);
  
\end{tikzpicture}
}
    \caption{\label{fig:proof}Dynamic partitioning leads to correctness issues due to
      staleness and ordering of operations.}
  \end{center}
\end{figure*}



In this section, we aim at solving Problem Statement~\ref{prob:intra}
by proposing \NAME, a reactive logical partitioning protocol for
dynamic partitioning in dynamic networks.

\paragraph{Dynamic partitions.}
At any time, a process can become a source, hence adding a new
partition to the system. This partition eventually includes all
processes that are closer from this new source than any other else. We
described such protocol in Section~\ref{sec:background}. Processes
naturally converge towards their respective best partition by only
piggybacking a monotonically increasing distance in forwarded
messages. % \TODO{Traffic of each partition is contained to the
%  partition.}

Then, at any time, a source can revoke its self-appointed status of
source, hence deleting its partition from the system. All processes
that belong to this partition must eventually choose another partition
to belong to. Since the number of partitions does not monotonically
increase in the system any longer, each process requires a vector of
versions that monotonically increases over delivered operations.

This allows processes to quickly discard stale messages saving
bandwidth. For instance, a process that received a message originated
at Process $a$ with a version $2$ knows that any message originated at
Process $a$ with a version $1$ are stale system-wide; it must not
forward it. As consequence, this also ensure termination, for
corresponding add and delete messages cannot follow each other in
infinite loops.
% In terms of traffic, this only requires each message
% to piggyback a source identifier and version number.

\noindent Deletes must trigger competition amongst neighboring
partitions. These \TODO{add} messages operate normally and fill gaps
left open by deletes. 

\TODO{Paths and \NAME synergize well, for paths tend to be smaller as the
number of sources increase in the system.}



\paragraph{Dynamic networks.}
Adding new communication links to the network may create shortcuts
between processes. Both processes must send their current best
partition to each other. Upon receipt, they act normally: if a process
finds out that the received partition is closer than its current one,
it delivers it which in turns also triggers another competition
amongst neighbors due to forwarding.

\noindent Joining the network is equivalent to add as many
communication links as necessary between the joining process and its
new neighbors.

\begin{algorithm}
  \SetKwProg{Function}{func}{}{}

\SetInd{0.2em}{1em}

\small

\DontPrintSemicolon
\LinesNumbered

% \begin{multicols}{2}
\Function(\tcp*[f]{new link to $q$}){\textup{edgeUp($q$)}}  {
    \lIf {$A_{\pi}^{d} \neq \alpha_\varnothing^\infty$} {
         send$_q$($\alpha_{\pi}^{d + W_{pq}}$)
         }
}

% \BlankLine

\Function(\tcp*[f]{link to $q$ removed}){\textup{edgeDown($q$)}} {
  \lIf {\textup{isParent($q$)}} {
       receiveDel($q$, $\delta_{p, V[p]+1}$)
  }
}

% \end{multicols}

% \BlankLine

  \caption{\label{algo:edges}Dynamic partitioning by Process $p$ in dynamic networks.}
\end{algorithm}

When removing a communication link between two processes does not
break any active path, because neither distances of processes depend
on the other, then nothing needs to be done. \NAME has no overhead.
Unfortunately, when a process' distance depends on the other process,
the protocol becomes much more complex. Indeed, this requires to
\TODO{undo} all add messages originated from this process. A message
must convey the fact that
\begin{inparaenum}[(i)]
\item an edge at a particular process has been removed, and
\item the distance that has been delivered by a process comes from
  this particular process.
\end{inparaenum}

\noindent \NAME handles this as a particular case of partition within
current partition: it contains to affected regions the traffic
generated to patch affected regions. In order to know if it is
affected by the \TODO{undo} operation, using distances already
piggybacked in messages is not sufficient.  Each process must know the
path taken by delivered message. Since we do not beforehand which
communication links will crash and which messages will be affected by
undos, each message has to carry its path along forwarding. This also
allows processes to remove loops, for they cannot rely on version
number (that remains unchanged) for this operation.

\noindent Instead of version number, one could use such paths to remove loops
due to \TODO{add and del}, however this may be much more costly in
terms of generated traffic. Version numbers guarantee that only one
delete is delivered and all prior inserts are then discarded, while
using paths, processes may deliver multiple inserts and deletes before
reaching termination. \TODO{maybe clearer with figures.} As
consequence, \NAME still makes use of vector of versions.

\noindent Although costly, piggybacking paths with logical partitioning synergies
well, for the size of these paths decreases quickly as the number of
partitions grows.



\paragraph{A last optimization.}
\NAME makes extensive use of paths to enable deleting messages without
incrementing local counters. Messages carry their respective path and
processes detect when such messages have been looping when they carry
their identity (see Lines~\ref{line:notloopingA} and
\ref{line:notloopingB} of Algorithm~\ref{algo:adddelundo}). Since
membership is all that matters, we propose to trade vectors of
identities for Bloom filters~\cite{almeida2007scalable} to improve on
generated traffic and
anonymity~\cite{whitaker2002forwarding}. Processes know with high
probability if they already received and forwarded each message
without knowing the identity of all processes that received it.  False
positive probability only incurs premature stopping of broadcast
messages and does not invalidate the delete of specific messages.



\subsection{Inter-autonomous system partitioning}

\begin{figure}
  \centering\begin{tikzpicture}[scale=0.92]

  \thickmuskip=0mu
  \medmuskip=0mu
  \thinmuskip=0mu
  
  \newcommand\X{50pt}
  \newcommand\Y{-50pt}

  \newcommand\ADD{\alpha}



  \draw (-1.5*\X, 0 - 5pt)
  node[below, circle, draw=black, fill=white, scale=0.35]{\large{0}}
  --  
  (-1.5*\X, \Y + 5pt)
  node[above, circle, draw=black, fill=white, scale=0.35]{\large{0}}; %% a - e

  \draw (-1.5*\X + 5pt, 0)
  node[right, circle, draw=black, fill=white, scale=0.35]{\large{1}}
  --
  (-1*\X, 0);

  \draw[dotted] (-1*\X, 0) -- (-0.5*\X, 0);
  
  \draw (-0.5*\X, 0)
  --  
  (0 - 5pt, 0)
  node[left, circle, draw=black, fill=white, scale=0.35]{\large{2}}; %% a - b

  \draw (0 +5pt, 0)
  node[right, circle, draw=black, fill=white, scale=0.35]{\large{1}}
  --
  (\X -5pt, 0)
  node[left, circle, draw=black, fill=white, scale=0.35]{\large{1}}; %% b - c
  
  \draw (0, 0 - 5pt)
  node[below, circle, draw=black, fill=white, scale=0.35]{\large{0}}
  --
  (0, \Y + 5pt)
  node[above, circle, draw=black, fill=white, scale=0.35]{\large{0}}; %% b - d

  \draw (\X + 3pt, 0 - 5pt)
  node[below left, circle, draw=black, fill=white, scale=0.35]{\large{0}}
  --
  (0 + 5pt, \Y - 3pt)
  node[above right, circle, draw=black, fill=white, scale=0.35]{\large{0}}; %% c - d



  \draw[dashed] (-0.75*\X, -0.5*\Y) -- (-0.75*\X, 1.5*\Y)
  node[above left]{\textit{AS 1}}
  node[above right]{\textit{AS 2}};

  \draw[fill=white] (-1.5*\X, 0) node[color=\PA]{$\bm{a}$} +(-5pt, -5pt) rectangle +(5pt, 5pt);
  \draw [left] (-1.5*\X - 5pt, 0) node{\textbf{e: 1}};
  \draw[fill=white] (0, 0) node[color=\PA]{$\bm{b}$} +(-5pt, -5pt) rectangle +(5pt, 5pt);
  \draw [above] ( 0, 5pt ) node{\textbf{e: 2}};
  \draw[fill=white] (\X, 0) node{$\bm{c}$} +(-5pt, -5pt) rectangle +(5pt, 5pt);
  \draw[fill=white] (0, \Y) node{$\bm{d}$} +(-5pt, -5pt) rectangle +(5pt, 5pt);

  \draw[fill=white] (-1.5*\X, \Y) node[color=\PA]{$\bm{e}$} +(-5pt, -5pt) rectangle +(5pt, 5pt);
  \draw [left] (-1.5*\X - 5pt, \Y) node{\textbf{e: 0}};
  
  
  % \draw (-\X, 5pt) node[above, color=\PA]{$\ADD_a^0$};
  % \draw (\X, 5pt) node[above, color=\PC]{$\ADD_c^0$};


\end{tikzpicture}

  \caption{\label{fig:AS}Inter autonomous systems
    partitioning. \TODO{More about relative ordering of links at each
      process}}
\end{figure}

In this section, we aim at solving Problem Statement~\ref{prob:inter}
by proposing \NAMEB, an extension of \NAME, that enables dynamic
logical partitioning in inter-autonomous systems. It leverages
hierarchical properties of these networks to further improve on scoped
flooding.

Autonomous systems are geo-distributed networks. Heterogeneous
communication links. Hierarchy of communication links. Per-object
broadcast.

We leave the link handling to membership protocols~\REF. The ordering
of links can be done based on latency. For instance, from 0 to 100 ms,
rank 0, from 100 to 1s, rank 2 \ldots

\begin{algorithm}
  \SetKwProg{Function}{func}{}{}

\small

\DontPrintSemicolon
\LinesNumbered

$\mathcal{A} \leftarrow \varnothing$ \tcp*[r]{map each object to its \TODO{\NAME}}
$\mathcal{O}_p$ \tcp*[r]{sorted list of sets of links}

\BlankLine
\BlankLine

\Function{\textup{create($o$)}} {
    initObject($o$) \;
    \lForEach {$a \in \mathcal{A}[o]$} {$a$.Add( )}
}

\Function{\textup{remove($o$)}} {
    \lForEach {$a \in \mathcal{A}[o]$} {$a$.Del( )}
}

\BlankLine
\BlankLine

\Function{\textup{receive($q$, $o$, $m$) }} {
    initObject($o$) \;

    \textbf{let} $i$ \textbf{with} $q \in \mathcal{O}^i_p$ \textbf{or} $0$
    \tcp*[r]{$i^{th}$ set of links}

    \For{$j$ \textbf{\textup{with}} $i \leq j < |\mathcal{A}[o]|$}  {
       \textbf{let} $a \leftarrow \mathcal{A}[o][j]$ \;
       \lIf {$m \in \alpha$} {$a$.receivedAdd($q$, $m$)}
       \lIf {$m \in \delta$} {$a$.receivedDel($q$, $m$)}
    }
}

\BlankLine
\BlankLine

\Function{\textup{initObject($o$)} \tcp*[f]{utility}} {
    \If {$o \not\in \mathcal{A}$} {
        $\mathcal{A}[o] \leftarrow \varnothing$ \;
        \ForEach {$\mathcal{O}_p^r \in \mathcal{O}_p$} {
           $\mathcal{A}[o] \leftarrow \mathcal{A}[o] \cup \{ \TODO{A^4}(\mathcal{O}_p^r) \} $
        }
    }

}

% \Function{\textup{newObject($o$)}} {

% \Function{\textup{receive($q$, $m$)}} {
%    \If {$m \in \alpha$} {
%       \textbf{let} $best \leftarrow a.best$ \;
%       $a$.receiveAdd($q$, $m$) \;
%       \If {$best = a.best$} {
%       \TODO{from higher $q$ to min $q$} \;
%       broadcast($q$, $a.best$)
%       }
%    }
% }

% \Function{\textup{broadcast($q$, $m$)}} {

%   \textbf{let} $i$ \textbf{\textup{with}} $q \in \mathcal{O}^i_p$ \textbf{or} $0$
%   \tcp*[r]{$i^{th}$ set of links}

%   \ForEach{$q \in O_p^j$ \textbf{\textup{with}} $i \leq j < |\mathcal{O}_p|$}  {
%         \textup{sendTo$_q$($m +_\alpha w_{pq}$)}
%   }
  
% }

% }
  \caption{\label{algo:aaaa}\NAMEB running at Process $p$ for dynamic
    partitioning in inter-autonomous systems.}
\end{algorithm}

Figure~\ref{fig:AS} shows an example of inter-AS logical partitioning
where each process ranks its communication links, and each process
forwards messages to its neighbors starting from the rank of the link
that triggered the receipt, to its highest rank neighbors. In this
example, Process $e$ adds a partition. It forwards the generated
message to all its neighbors as if it received it from its lowest rank
links. When Process $a$ receives $\alpha_e^1$ from its rank-$0$ link,
it forwards it to rank-$0$ links and rank-$1$ links.  Process $b$
stops the propagation, for all its links have lower ranks than the one
from which it received the message $\alpha_e^2$. Ultimately, the
partition includes Process $e$, Process $a$, and Process $b$. Process
$c$ and Process $d$ remain unaware of the new \TODO{object}.

To ease the reasoning about inter-AS dynamic partitioning in dynamic
networks, \NAME runs an independent broadcast protocol for each
object, and for each rank of links.  \TODO{Not so easy\ldots}
\TODO{Write and describe algo.} \TODO{Overhead of separating things,
  slightly more memory, but traffic wise? slower ?}  \TODO{In the
  example, not only Process $b$ keeps $e_1$ as its best partition, but
  it keeps it for links of rank 0, and links of rank 1.}  When there
are no ranking in links, processes solely rely on
Algorithm~\ref{algo:adddelundo} for optimal partitioning.



\subsection{Complexity}
\label{subsec:complexity}

\TODO{To rework, for there are more components now. Maybe do this
  along the description of the approach.}

We focus on average-case and worst-case complexity. We divide our
analysis into space, time, and communication complexity.

\textbf{The communication complexity} concerns the size and number of
messages required to reach optimal partitioning. In the average-case,
a process $i$ chosen uniformly at random among all processes creates a
logical partition. Its messages $\alpha_i$ propagate through the
network until reaching processes that belong to another partition
closer to them. This splits partitions in half in average. Overall,
the $a^{th}$ new partition comprises
\smash{$\mathcal{O}(\frac{|V|}{2^{\lfloor \log_2 a \rfloor}})$}
processes. This decreases every new partition until reaching $0$
processes per new partition: even the chosen process already belongs
to its optimal partition. The average number of messages per process
is \smash{$\mathcal{O}(\frac{\overline{|O|}}{2^{\lfloor \log_2 a
      \rfloor}})$}. \TODO{Multiple receipt and multiple delivery imply
  more messages (receipt bounded by $|O|$ as well).} Deleting the
$a^{th}$ partition generates the exact same number of messages than
the $a^{th}$ partition creation. \TODO{But what about echos?} In the
worst-case, every new partition includes all but one process belonging
to the previous partition. The total number of messages after the
$a^{th}$ new partition is $\mathcal{O}(\overline{|O|}\cdot a^2)$. As
for the average-case, the number of messages for the partition
deletion is identical to the number of messages of the corresponding
partition creation.

%%% Local Variables: 
%%% mode: latex
%%% TeX-master: "../paper"
%%% ispell-local-dictionary: "english"
%%% End: 
