
\section{Conclusion}
\label{sec:conclusion}

In this paper, we proposed a general purpose definition of scoped
broadcast, and an extension to automatically adapt the scope of scoped
broadcast in contexts where each process can decide whether or not it
decide to be a source. We proposed an implementation called \NAME.

As future work, we plan on enhancing the
\underline{I}nter\underline{P}lanetary \underline{F}ile
\underline{S}ystem (IPFS)~\cite{henningsen2020mapping} with dynamic
logical partitioning capabilities in order to improve its caching
policy of popular contents at marginal cost.

\noindent We propose an implementation that further decreases
generated traffic and improves on
anonymity~\cite{whitaker2002forwarding}.  \NAME makes extensive use of
paths to enable deleting messages without incrementing local
counters. Messages carry their respective path and processes detect
when such messages have been looping when they carry their identity
(see Lines~\ref{line:notloopingA} and \ref{line:notloopingB} of
Algorithm~\ref{algo:adddelundo}). Since membership is all that
matters, we propose to trade vectors of identities for Bloom
filters~\cite{almeida2007scalable}. Processes know with high
probability if they already received and forwarded each message
without knowing the identity of all processes that received it.  False
positive probability only incurs premature stopping of broadcast
messages and does not invalidate the delete of specific
messages. \TODO{Rework.}

\noindent We also plan on further improving scoped broadcast between
multiple autonomous systems. Indeed, processes could leverage the
hierarchical properties of such topology~\cite{nur2018geography} to
avoid flooding the whole systems with control information about all
partitions.

%%% Local Variables: 
%%% mode: latex
%%% TeX-master: "../paper"
%%% ispell-local-dictionary: "english"
%%% End: 
