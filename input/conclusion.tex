\section{Conclusion}
\label{sec:conclusion}

With the advent of the Edge and IoT era, major distributed services
proposed by our community should be revised to mitigate as much as
possible traffic between \processes.  In this paper, we addressed the
challenge of content indexing as a dynamic logical partitioning
problem where partitions grow and shrink to reflect content locations,
and infrastructure changes.  Using an efficient communication
primitive called scoped broadcast, each \process composing the
infrastructure eventually knows the \process hosting the closest replica
of a specific content.  The challenge resides in handling concurrent
operations that may impair the propagation of messages, and in turn,
lead to inconsistent partitioning.
%
We highlighted the properties that solve this problem and proposed an
implementation called \NAME for \underline{A}daptive
\underline{S}coped broad\underline{cast}.  Simulations confirmed that
\processes quickly reach consistent partitioning together while the
generated traffic remains locked down to partitions.

As future work, we plan to leverage the hierarchical properties of
interconnected autonomous systems~\cite{nur2018geography} to further
limit the propagation of indexes within interested systems only.  We
expect that such an improvement would greatly benefit the autonomous
systems, and particularly those hosting a large number of contents.
Indeed, \NAME ensures that every \process eventually acknowledges its
closest content replica. This feature unfortunately becomes an issue
when each \process is only interested by a small portion of
content.

We also plan to evaluate our proposal within a concrete storage system
such as \underline{I}nter\underline{P}lanetary \underline{F}ile
\underline{S}ystem (IPFS)~\cite{ipfs}. This would assess the relevance
of \NAME in real systems subject to high dynamics, and compare it
against its current DHT-based indexing system that does not include
distance in its operation. More generally, we claim that \NAME and its
extension can constitute novel building blocks for geo-distributed
services.  For instance, \NAME could complement content delivery
infrastructures~\cite{triukose2011measuring} by efficiently sharing
between \processes attached to the same CDN server the locations of
web objects that have already been downloaded. This would further
improve the containment of web traffic in our networks, and in turns,
reduce the overall traffic footprint.
%
%% More generally, it can be used in other services in order to share information while mitigating the traffic.
%% MWhile we use the distance to the closest copy in this paper, It is
%% notewothy that other scopes can be envisioned according to the
%% use-cases.  For instance, the location of a city in order to broadcast
%% message to all nodes in the city.  This requires nodes to store and
%% maintain their city location in their local state. Nodes stop
%% delivering and forwarding their received messages when they come from
%% a different city. Such a predicate can be complexified in order to
%% broadcast messages to all nodes in the city plus neighboring
%% cities. This requires nodes to overload forwarded messages the first
%% time they reach another city. The predicate checks if messages already
%% reached two distinct cities before delivery.  The predicate can even
%% be piggypack within the message and so be different for each
%% message.\TODO{Complete the use-cases}


%%% Local Variables: 
%%% mode: latex
%%% TeX-master: "../paper"
%%% ispell-local-dictionary: "english"
%%% End: 
