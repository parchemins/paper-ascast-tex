
\section{Conclusion}
\label{sec:conclusion}

With the advent of the edge and IoT era, major distributed services
proposed by our community should be revised in order to mitigate as
much as possible trafic between \processes.  In this paper, we tackled
the problem of content indexing, a key service for storage systems.
\TODO{pas vraiment du content indexing}
%
More precisely, we proposed to address it as a dynamic logical partitioning of the infrastructure
that should be maintained according to replicas creations/removals and
infrastructure changes. Through an efficient scoped broadcast, each
\process composing the infrastructure knows eventually where the best
replica for a particular content is. The major difficulty is to handle
concurrent operations, each having a different scope and ensure the
correct convergence of the system. We defined multiple properties that highlight this complexity.

We proposed an implementation of this protocol called \NAME and
evaluated it on top of Peersim. The simulations confirmed the
correctness of our protocol as well as the confinement of the traffic
only where it is needed.

% 
From the technical viewpoint, this new service can easily be
integrated in storage services such as
\underline{I}nter\underline{P}lanetary \underline{F}ile
\underline{S}ystem (IPFS)~\cite{henningsen2020mapping}.  Such an integration would enabe us to
to perform additional experiments and observe the gain with 
respect to the DHT-based approach that is currently used.

%
%% More generally, it can be used in other services in order to share information while mitigating the traffic.
%% MWhile we use the distance to the closest copy in this paper, It is
%% notewothy that other scopes can be envisioned according to the
%% use-cases.  For instance, the location of a city in order to broadcast
%% message to all nodes in the city.  This requires nodes to store and
%% maintain their city location in their local state. Nodes stop
%% delivering and forwarding their received messages when they come from
%% a different city. Such a predicate can be complexified in order to
%% broadcast messages to all nodes in the city plus neighboring
%% cities. This requires nodes to overload forwarded messages the first
%% time they reach another city. The predicate checks if messages already
%% reached two distinct cities before delivery.  The predicate can even
%% be piggypack within the message and so be different for each
%% message.\TODO{Complete the use-cases}


From the scientific viewpoint, we are currently investigating how our
proposal can be extended in orer to take into account multiple
autonomous systems. Indeed, \processes could leverage the hierarchical
properties of the Internet topology~\cite{nur2018geography} to avoid
flooding the whole systems with control information about all
partitions.
\TODO{should we highlight this at the end of the experiment}



%%% Local Variables: 
%%% mode: latex
%%% TeX-master: "../paper"
%%% ispell-local-dictionary: "english"
%%% End: 
