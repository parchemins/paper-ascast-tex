\section{Conclusion}
\label{sec:conclusion}

With the advent of the Edge and IoT era, major distributed services
proposed by our community should be revised in order to mitigate as
much as possible traffic between \processes.  In this paper, we
tackled the problem of content indexing, a key service for storage
systems.
%
We proposed to address this problem as a dynamic logical partitioning
where partitions grow and shrink to reflect content and replicas
locations, as well as infrastructure changes. Using an efficient
communication primitive called scoped broadcast, each \process
composing the infrastructure eventually knows the \process hosting the
best replica of specific content.  The challenge resides in handling
concurrent operations that may impair the propagation of messages, and
in turn, lead to inconsistent partitioning.
%% The major difficulty lies in concurrent operations, each having a
%% different scope, and ensure the correct convergence of the system.  We
%% defined, in particular, multiple properties that highlight this
%% complexity.
%
We highlighted the properties that solve this problem and proposed an
implementation called \NAME.  Our simulations confirmed that
\processes quickly reach consistent partitioning together, and that
the generated traffic remains locked down to partitions.
%% correctness of our protocol as well as the confinement of
%% the traffic only where it is needed.

As future work, we plan on extending the logical partitioning protocol
to enable lazy retrieval of partitions. Indeed, using \NAME, every
\process partakes in the propagation of indexes. We would like to
leverage the hierarchical properties of interconnected autonomous
systems to limit the propagation of such indexes only to interested
systems.
%
%% As ongoing and future work, we are currently studying how it might be
%% possible to limit the scope of a partition to one autonomous system at
%% most. Indeed, the gain of propagating changes to \processes belonging to
%% another autonomous system might be negligible as the latency penalty
%% is mainly due to inter autonomous systems' links (\eg trans-continental
%% links).  Exactly, we are investigating how such a change can be
%% formalized and its impact on the current protocol.
%
%autonomous systems. Indeed, \processes could leverage the hierarchical
%properties of the Internet topology~\cite{nur2018geography} to avoid
%flooding the whole systems with control information about all
%partitions.
%\TODO{should we highlight this at the end of the experiment}
%

\noindent We also would like to evaluate \NAME within a concrete
storage system such as \underline{I}nter\underline{P}lanetary
\underline{F}ile \underline{S}ystem
(IPFS)~\cite{henningsen2020mapping}. This would allow us to assess the
relevance of \NAME in a real system subject to high dynamicity, and
compare it against its current DHT-based indexing system that does not
include distance in its operation.

%% Such an integration would enable us to to perform additional
%% experiments and observe the gain with respect to the DHT-based
%% approach that is currently used.

%% In addition to this action, it would make sense to evaluate
%% the use of \NAME within a concrete storage systems, such as
%% \underline{I}nter\underline{P}lanetary \underline{F}ile
%% \underline{S}ystem (IPFS)~\cite{henningsen2020mapping}.  Such an
%% integration would enable us to to perform additional experiments and
%% observe the gain with respect to the DHT-based approach that is
%% currently used.

%
%% More generally, it can be used in other services in order to share information while mitigating the traffic.
%% MWhile we use the distance to the closest copy in this paper, It is
%% notewothy that other scopes can be envisioned according to the
%% use-cases.  For instance, the location of a city in order to broadcast
%% message to all nodes in the city.  This requires nodes to store and
%% maintain their city location in their local state. Nodes stop
%% delivering and forwarding their received messages when they come from
%% a different city. Such a predicate can be complexified in order to
%% broadcast messages to all nodes in the city plus neighboring
%% cities. This requires nodes to overload forwarded messages the first
%% time they reach another city. The predicate checks if messages already
%% reached two distinct cities before delivery.  The predicate can even
%% be piggypack within the message and so be different for each
%% message.\TODO{Complete the use-cases}





%%% Local Variables: 
%%% mode: latex
%%% TeX-master: "../paper"
%%% ispell-local-dictionary: "english"
%%% End: 
