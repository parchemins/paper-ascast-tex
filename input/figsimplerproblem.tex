
\begin{figure*}
  \newcommand{\SCALE}{0.9}

  \newcommand{\LEFT}{\triangleleft}
  \newcommand{\RIGHT}{\triangleright}
  
  \begin{center}
    \subfloat[Part A][\label{fig:problemA}Both $a$ and $c$ become sources.
      $w_{ab} = 2$; $w_{bc} = 1$.]{
\begin{tikzpicture}[scale=\SCALE]

  \draw [opacity=0] (-1.5*\X, 0) -- (1.5*\X, 0); %% spacing
  
  \draw (-\X + 5pt, 0) --
  node[above=-0.3em, font=\SMSG]{$\textcolor{\PA}{\alpha_{a}^{2}} \RIGHT$}
  (0 - 5pt, 0); %% b - a 

  \draw (0 +5pt, 0) --
  node[below=-0.3em,font=\SMSG]{~ ~ $\LEFT \textcolor{\PC}{\alpha_{c}^{1}}$}
  (\X -5pt, 0); %% b - c


  
  \draw[fill=white] (-\X, 0) node{\textcolor{\PA}{$\bm{a}$}} +(-5pt, -5pt) rectangle +(5pt, 5pt);  
  \draw[fill=white] (0, 0) node{$\bm{b}$} +(-5pt, -5pt) rectangle +(5pt, 5pt);
  \draw[fill=white] (\X, 0) node{\textcolor{\PC}{$\bm{c}$}} +(-5pt, -5pt) rectangle +(5pt, 5pt);
  
  \draw (-\X, -6pt) node[below, font=\SNODE]{$\textcolor{\PA}{\alpha_a^0}$};
  \draw ( \X, -6pt) node[below, font=\SNODE]{$\textcolor{\PC}{\alpha_c^0}\vphantom{\alpha_a^3}$};
  
\end{tikzpicture}
}
    \hspace{0pt}
    \subfloat[Part B][\label{fig:problemB}Both $a$ and $c$ delete their partition 
     while $b$ delivers and forwards $\alpha_a$.]
             {

\begin{tikzpicture}[scale=\SCALE]

  \thickmuskip=0mu
  \medmuskip=0mu
  \thinmuskip=0mu
  
  \newcommand\X{50pt}
  \newcommand\Y{-50pt}

  \newcommand\ADD{\alpha}
  \newcommand\DEL{\delta}


  
  \draw (-\X + 5pt, 0) --
  node[above=-0.3em,font=\tiny]{$\DEL_{a} \RIGHT$}
  (0 - 5pt, 0); %% b - a 

  \draw (0 +5pt, 0) --
  node[above=-0.3em,font=\tiny]{$\textcolor{\PA}{\ADD_{a}^{3}} \RIGHT$}
  node[below=-0.3em,font=\tiny]{$\LEFT \textcolor{\PC}{\ADD_{c}^{1}} \cdot \DEL_c$}
  (\X -5pt, 0); %% b - c


  
  \draw[fill=white] (-\X, 0) node{$\bm{a}$} +(-5pt, -5pt) rectangle +(5pt, 5pt);  
  \draw[fill=white] (0, 0) node{\textcolor{\PA}{$\bm{b}$}} +(-5pt, -5pt) rectangle +(5pt, 5pt);
  \draw[fill=white] (\X, 0) node{$\bm{c}$} +(-5pt, -5pt) rectangle +(5pt, 5pt);
  
  \draw (-\X, 5pt) node[above, font=\scriptsize]{$\textcolor{\PA}{\ADD_a}\rightarrow \DEL_a$};
  \draw (  0, 5pt) node[above, font=\scriptsize]{$\textcolor{\PA}{\ADD_a^2}$};
  \draw ( \X, 5pt) node[above, font=\scriptsize]{$\textcolor{\PC}{\ADD_c}\vphantom{\ADD_a^3} \rightarrow \DEL_c$};

  \begin{scope}[shift={(0, -1*\Y)}]
      \draw (-\X + 5pt, 0) --
      (0 - 5pt, 0); %% b - a 
      
      \draw (0 +5pt, 0) --
      node[above=-0.3em,font=\tiny]{$\textcolor{\PA}{\ADD_{a}^{3}} \RIGHT$}
      node[below=-0.3em,font=\tiny]{$\LEFT \textcolor{\PC}{\ADD_{c}^{1}} \cdot \DEL_c$}
      (\X -5pt, 0); %% b - c
      
      \draw[fill=white] (-\X, 0) node{\textcolor{\PA}{$\bm{a}$}} +(-5pt, -5pt) rectangle +(5pt, 5pt);  
      \draw[fill=white] (0, 0) node{\textcolor{\PA}{$\bm{b}$}} +(-5pt, -5pt) rectangle +(5pt, 5pt);
      \draw[fill=white] (\X, 0) node{$\bm{c}$} +(-5pt, -5pt) rectangle +(5pt, 5pt);
      
      \draw (-\X, 5pt) node[above, font=\scriptsize]{$\textcolor{\PA}{\ADD_a^0}$};
      \draw (  0, 5pt) node[above, font=\scriptsize]{$\textcolor{\PA}{\ADD_a^2}$};
      \draw ( \X, 5pt) node[above, font=\scriptsize]{$\textcolor{\PC}{\ADD_c}\vphantom{\ADD_a^3} \rightarrow \DEL_c$};

  \end{scope}

\end{tikzpicture}
}
    \hspace{0pt}
    \subfloat[Part C][\label{fig:problemC}$b$ blocks the only transiting
      $\delta_a$ while $b$ delivers and forwards $\alpha_c$.]
             {
\begin{tikzpicture}[scale=\SCALE]

  \thickmuskip=0mu
  \medmuskip=0mu
  \thinmuskip=0mu
  
  \newcommand\X{50pt}
  \newcommand\Y{-50pt}

  \newcommand\ADD{\alpha}
  \newcommand\DEL{\delta}


  
  \draw (-\X + 5pt, 0) --
  node[above=-0.3em, font=\tiny]{~ ~ ~ ~ ~ $\DEL_{a} \RIGHT$} %% b - a
  node[above=-0.3em, font=\tiny]{~ ~ ~ ~ ~ $\textcolor{\WRONG}{\text{\normalsize\xmark}} \hphantom{\RIGHT}$} %% b - a
  node[below=-0.3em, font=\tiny]{$\LEFT \textcolor{\PC}{\ADD_c^3}$} %% b - a
  (0 - 5pt, 0);

  \draw (0 +5pt, 0) --
  node[below=-0.3em, font=\tiny]{$\LEFT \DEL_c \vphantom{\ADD^1}$}
  (\X -5pt, 0); %% b - c


  
  \draw[fill=white] (-\X, 0) node{$\bm{a}$} +(-5pt, -5pt) rectangle +(5pt, 5pt);  
  \draw[fill=white] (0, 0) node{\textcolor{\PC}{$\bm{b}$}} +(-5pt, -5pt) rectangle +(5pt, 5pt);
  \draw[fill=white] (\X, 0) node{\textcolor{\PA}{$\bm{c}$}} +(-5pt, -5pt) rectangle +(5pt, 5pt);
  
  \draw (-\X, 5pt) node[above, font=\scriptsize]{$\textcolor{\PA}{\ADD_a}\rightarrow \DEL_a$};
  \draw (  0, 5pt) node[above, font=\scriptsize]{$\textcolor{\PA}{\ADD_a^2} \rightarrow \textcolor{\PC}{\ADD_c^1}$};
  \draw ( \X, 5pt) node[above, font=\scriptsize]{$\textcolor{\PC}{\ADD_c}\vphantom{\ADD_a^3} \rightarrow \DEL_c \rightarrow \textcolor{\PA}{\ADD_a^3}$};


  \begin{scope}[shift={(0, -1*\Y)}]
    \draw (-\X + 5pt, 0) --
    node[below=-0.3em, font=\tiny]{$\LEFT \textcolor{\PC}{\ADD_c^3}$} %% b - a
    (0 - 5pt, 0);
    
    \draw (0 +5pt, 0) --
    node[below=-0.3em, font=\tiny]{$\LEFT \DEL_c \vphantom{\ADD^1}$}
    (\X -5pt, 0); %% b - c
    
    \draw[fill=white] (-\X, 0) node{\textcolor{\PA}{$\bm{a}$}} +(-5pt, -5pt) rectangle +(5pt, 5pt);  
    \draw[fill=white] (0, 0) node{\textcolor{\PC}{$\bm{b}$}} +(-5pt, -5pt) rectangle +(5pt, 5pt);
    \draw[fill=white] (\X, 0) node{\textcolor{\PA}{$\bm{c}$}} +(-5pt, -5pt) rectangle +(5pt, 5pt);
    
    \draw (-\X, 5pt) node[above, font=\scriptsize]{$\textcolor{\PA}{\ADD_a^0}$};
    \draw (  0, 5pt) node[above, font=\scriptsize]{$\textcolor{\PA}{\ADD_a^2} \rightarrow \textcolor{\PC}{\ADD_c^1}$};
    \draw ( \X, 5pt) node[above, font=\scriptsize]{$\textcolor{\PC}{\ADD_c}\vphantom{\ADD_a^3} \rightarrow \DEL_c \rightarrow \textcolor{\PA}{\ADD_a^3}$};
    
  \end{scope}

\end{tikzpicture}
}
    \hspace{0pt}
    \subfloat[Part D][\label{fig:problemD}$b$ delivers and forwards $\delta_c$. $c$
    stays in the deleted partition $P_a$.]
             {
\begin{tikzpicture}[scale=\SCALE]
  
  \draw (-\X + 5pt, 0) --
  (0 - 5pt, 0);

  \draw [->] (0 +5pt, 0) --
  node[below=-0.3em, font=\tiny]{$\vphantom{\alpha^1_c }$}
  (\X -5pt, 0); %% b - c


  
  \draw[fill=white] (-\X, 0) node{$\bm{a}$} +(-5pt, -5pt) rectangle +(5pt, 5pt);  
  \draw[fill=white] (0, 0) node{$\bm{b}$} +(-5pt, -5pt) rectangle +(5pt, 5pt);
  \draw[color=\WRONG, fill=white] (\X, 0) node{\textcolor{\PA}{$\bm{c}$}} +(-5pt, -5pt) rectangle +(5pt, 5pt);
  
  \draw (-\X, 5pt) node[above, font=\scriptsize]{$\ldots \rightarrow \textcolor{\PC}{\alpha_c} \rightarrow \delta_c$};
  \draw (  0, 5pt) node[above, font=\scriptsize]{$\ldots \rightarrow \delta_c$};
  \draw ( \X, 5pt) node[above, font=\scriptsize]{$\ldots \rightarrow \textcolor{\PA}{\alpha_a^3}$};


  
  %% \begin{scope}[shift={(0, -1*\Y)}]

  %%   \draw (-\X + 5pt, 0) --
  %%   (0 - 5pt, 0);
    
  %%   \draw (0 +5pt, 0) --
  %%   node[below=-0.3em, font=\tiny]{$\vphantom{\alpha^1_c }$}
  %%   (\X -5pt, 0); %% b - c

    
  %%   \draw[fill=white] (-\X, 0) node{\textcolor{\PA}{$\bm{a}$}} +(-5pt, -5pt) rectangle +(5pt, 5pt);  
  %%   \draw[fill=white] (0, 0) node{\textcolor{\PA}{$\bm{b}$}} +(-5pt, -5pt) rectangle +(5pt, 5pt);
  %%   \draw[fill=white] (\X, 0) node{\textcolor{\PA}{$\bm{c}$}} +(-5pt, -5pt) rectangle +(5pt, 5pt);
    
  %%   \draw (-\X, 5pt) node[above, font=\scriptsize]{$\textcolor{\PA}{\alpha_a^0}$};
  %%   \draw (  0, 5pt) node[above, font=\scriptsize]{$\ldots \rightarrow \delta_c \rightarrow \textcolor{\PA}{\alpha_a^2}$};
  %%   \draw ( \X, 5pt) node[above, font=\scriptsize]{$\ldots \rightarrow \textcolor{\PA}{\alpha_a^3}$};
  %% \end{scope}
  
\end{tikzpicture}
}
  \end{center}
  \caption{\label{fig:problem}Even in the simplest scenarios, naive
    propagation of $\alpha$ and $\delta$ messages may be insufficient
    to guarantee consistent partitioning. In the bottom scenario, if
    $c$ had children, they would stay in the wrong partition
    too.
    %
    % Despite both scenarios look the same from $c$'s perspective, $c$
    % must acknowledge $P_a$'s \emph{possible} deletion and propagate
    % it.
  }
\end{figure*}

