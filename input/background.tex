
\section{Background and motivation}
\label{sec:background}

Fog infrastructures are distributed systems comprising heterogeneous
machines interconnected by communication links. We do not consider
byzantine processes.

\begin{definition}[Fog infrastructure]
  A fog infrastructure is a \underline{g}raph $G(V, E)$ of
  \underline{v}ertices $V$ and bidirectional \underline{e}dges $E \in
  V \times V$. Edges have symmetric \underline{w}eights $w_{ij} =
  w_{ji}$ with $i, j \in V$.
\end{definition}

\begin{definition}[Logical partition]
  A logical \underline{p}artition $P_s$ originated at a
  \underline{s}ource process $s \in V$ is a set of interconnected
  processes that do not belong to another logical partition. The set
  of sources is $S \subseteq V$.
\end{definition}

Processes belong to logical partitions depending on path weights. When
weights are about latency or hop numbers, the optimal partitioning
consists in processing the multiple destinations shortest path at each
process \REF.

\begin{definition}[Shortest path]
  A \underline{p}ath $\pi_{ij}$ from Process $i$ to Process $j$ is a
  sequence of vertices $[i, k_1, k_2, \ldots k_n, j]$ with $\forall
  0\leq m \leq n, \langle \pi_{ij}[m], \pi_{ij}[m+1] \rangle \in
  E$. Shortest \underline{p}aths $\Pi_{ij}$ are sets of paths such
  that there exists no lower sum of weights $|\Pi_{ij}|$ from $i$ to
  $j$.
\end{definition}

\begin{definition}[Optimal partitioning]
  The optimal partitioning $\mathcal{P}$ is a set of logical
  partitions such that each process belongs to a logical partition
  comprising its closest source process.  $\forall p \in P_{s_1\in S},
  \nexists P_{s_2 \in S}$ such that $|\Pi_{s_1p}| > |\Pi_{s_2p}|$.
\end{definition}

Processes use reliable asynchronous message passing to reach optimal
partitioning together. Process $s$ generates a message
$\alpha_s^0$. Upon \underline{r}eceipt $r_i(\alpha_s^{w})$, Process
$i$ knows that Process $s$ created a partition at a path length of
$w$. Process $i$ may choose to \underline{d}eliver $d_i(\alpha_s^{w})$
if it improves its partitioning. Process $i$ may choose to propagate
its change through \underline{b}roadcast $b_i(\alpha_s^w)$.  Without
global knowledge, processes only rely on their own behavior to deduce
the behavior of their neighbor processes.


\begin{figure}[t]
  \newcommand{\SCALE}{0.8}

  \newcommand\X{38pt}
  \newcommand\Y{-40pt}

  \thickmuskip=0mu
  \medmuskip=0mu
  \thinmuskip=0mu

  \newcommand\ADD{\alpha}
  \newcommand\DEL{\delta}
    
  \newcommand{\LEFT}{\triangleleft}
  \newcommand{\RIGHT}{\triangleright}
  
  \begin{center}
    \subfloat[Part A][\label{fig:delA}$a$ deletes its partition. It
      notifies all \processes that belong to its
      partition.]{
\begin{tikzpicture}[scale=\SCALE]

  \draw[opacity=0](-2.45*\X, 0) -- (2.45*\X, 0); %% more space for caption
  

  
  \draw (-\X + 5pt, 0) --
  node[above=-0.3em,font=\tiny]{$\DEL_{a} \RIGHT$}
  (0 - 5pt, 0); %% b - a 

  \draw (0 +5pt, 0) --
  (\X -5pt, 0); %% b - c

  \draw [<-] (0, 0 - 5pt) --
  (0, \Y + 5pt);  %% b - d
  
  \draw [<-] (\X + 3pt, 0 - 5pt) --
  (0 + 5pt, \Y - 3pt); %% c - d


  
  \draw[fill=white] (-\X, 0) node{$\bm{a}$} +(-5pt, -5pt) rectangle +(5pt, 5pt);  
  \draw[fill=white] (0, 0) node[color=\PD]{$\bm{b}$} +(-5pt, -5pt) rectangle +(5pt, 5pt);
  \draw[fill=white] (\X, 0) node[color=\PD]{$\bm{c}$} +(-5pt, -5pt) rectangle +(5pt, 5pt);
  \draw[fill=white] (0, \Y) node[color=\PD]{$\bm{d}$} +(-5pt, -5pt) rectangle +(5pt, 5pt);
  
  \draw (-\X, 5pt) node[above, font=\small]{$\DEL_a$};
  \draw (0, 5pt) node[above, font=\small, color=\PD]{$\ADD_d^1$};
  \draw (\X, 5pt) node[above, font=\small, color=\PD]{$\ADD_d^2$};
  \draw (-5pt, \Y) node[left, font=\small, color=\PD]{$\ADD_d^0$};


\end{tikzpicture}
}
    \hspace{3pt}
    \subfloat[Part B][\label{fig:delB}$\delta$ stops as soon as it encounters
      another partition. $b$ answers with its partition.]{
\begin{tikzpicture}[scale=0.87]

  \thickmuskip=0mu
  \medmuskip=0mu
  \thinmuskip=0mu
  
  \newcommand\X{50pt}
  \newcommand\Y{-50pt}

  \newcommand\ADD{\alpha}
  \newcommand\DEL{\delta}



  \draw (-\X + 5pt, 0) --
  node[above=-0.3em,font=\tiny]{$\DEL_{a} \rightarrow$}
  node[below=-0.3em,font=\tiny]{$\leftarrow \ADD_{a}^{3}$}
  (0 - 5pt, 0); %% b - a 

  \draw (0 +5pt, 0) --
  node[above=-0.3em, font=\tiny]{$\leftarrow \textcolor{\PC}{\ADD_{c}^{1}}\cdot\DEL{c}$}
  node[below=-0.3em, font=\tiny]{$\textcolor{\PA}{\ADD_a^{2.5}} \rightarrow$}
  (\X -5pt, 0); %% b - c

  \draw[opacity=0] (0, 0 - 5pt) --
  node[opacity=1, above=-0.3em, font=\tiny, sloped]{$\ADD_a^{2.5} \rightarrow$}
  (0, \Y + 5pt); %% b - d
  \draw (0, \Y + 5pt) --
  node[above=-0.3em, font=\tiny, sloped]{$\ADD_c^2 \rightarrow$}
  (0, 0 - 5pt);  %% d - b
  
  \draw (\X + 3pt, 0 - 5pt) --
  node[above=-0.3em, font=\tiny, sloped]{$\ADD_{c}^{2} \rightarrow$}
  node[below=-0.3em, font=\tiny, sloped]{$\leftarrow \DEL_c$}
  (0 + 5pt, \Y - 3pt); %% c - d



  
  \draw[fill=white] (-\X, 0) node{$\bm{a}$} +(-5pt, -5pt) rectangle +(5pt, 5pt);  
  \draw[fill=white] (0, 0) node[color=\PA]{$\bm{b}$} +(-5pt, -5pt) rectangle +(5pt, 5pt);
  \draw[fill=white] (\X, 0) node{$\bm{c}$} +(-5pt, -5pt) rectangle +(5pt, 5pt);
  \draw[fill=white] (0, \Y) node[color=\PC]{$\bm{d}$} +(-5pt, -5pt) rectangle +(5pt, 5pt);
  
  % \draw (-\X+5pt, 5pt) node[above left]{$\DEL_a$};
  % \draw (\X+5pt, 5pt) node[above right]{$\DEL_c$};
  \draw (0, 5pt) node[above]{$\bm{a: 1.5}$};
  \draw (-5pt, \Y) node[left]{$\bm{c: 1}$};


\end{tikzpicture}
}
  \end{center}
  \caption{\label{fig:del}Efficient removal of a partition using
    scoped broadast. $a$ eventually acknowledges that it belongs to $P_d$ with $\alpha_d^3$ echoing.}
\end{figure}

%%% Local Variables: 
%%% mode: latex
%%% TeX-master: "../paper"
%%% ispell-local-dictionary: "english"
%%% End: 

 %% figure positioning

\begin{definition}[\label{def:transitivedelivery}Transitive delivery]
  Transitive delivery ensures the system eventually converges to
  optimal partitioning:
  \begin{inparaenum}[(i)]
  \item (\textbf{termination}) processes do not need to propagate
    operations they do not deliver: $\forall \langle i, j \rangle \in
    E, \neg d_i(\alpha_{s}^w)\implies \neg
    d_j(\alpha_{s}^{w+w_{ij}})$;
  \item (\textbf{propagation}) operations delivered by a process may
    also benefit neighbor processes: $\forall \langle i, j \rangle \in
    E, d_i(\alpha_{s}^{w_1}) \implies
    d_j(\alpha_{s}^{w_1+w_{ij}}) \vee
    d_j(\alpha_{*}^{w_2 < w_1 + w_{ij}})$.
  \end{inparaenum}
\end{definition}

\begin{algorithm}
  \SetKwProg{Function}{func}{}{}

\small

\DontPrintSemicolon
\LinesNumbered

$O_p$, $W_p$\tcp*[r]{set of neighbors and weights}
$s \leftarrow \varnothing$ \tcp*[r]{best source of partition ($\sigma$)}
$d \leftarrow \infty$ \tcp*[r]{smallest distance to $s$ ($\sigma$ and $\mu$)}


\BlankLine

\Function{\textup{Add ( )} \tcp*[f]{$\alpha_p^0$}} {
  \textup{receiveAdd($\varnothing$, $p$, $0$)} \label{line:lowestbound} \tcp*[f]{$b_p(\alpha_p^0)$}
}

\BlankLine

\Function{\textup{receiveAdd($q$, $s'$, $d'$)} \tcp*[f]{$r_p(\alpha_{s'}^{d'})$ from $q$}} {
  \If (\tcp*[f]{($\phi$)}){$d' < d$} {
      $s \leftarrow s'$ \tcp*[r]{\smash{$d_p(\alpha_{s'}^{d'})$}}
      $d \leftarrow d'$ \;

      \ForEach(\tcp*[f]{\smash{$f_p(\alpha_{s'}^{d'})$}}) {$n \in O_p \setminus q$} {
          \textup{send$_n$($s', d' + W_{pn}$)} \label{line:accumulator}
          \tcp*[r]{\smash{$s_{pn}(\alpha_{s'}^{d'+w_{pn}})$}}
       }      
  }
}

%% \BlankLine

%% \Function{\textup{edgeUp($q$)} \tcp*[f]{new link to $q$}} {
%%   \lIf { $d < \infty$} {\textup{send$_q$($s, d + W_{pq}$)}}
%% }



  \caption{\label{algo:add}Adding a partition by Process $p$.}
\end{algorithm}

Algorithm~\ref{algo:add} shows the instructions that implement the
rules of Definition~\ref{def:transitivedelivery} to converge towards
optimal partitioning. Figure~\ref{fig:add} illustrates its behavior on
a system comprising 4 processes $a$, $b$, $c$, and $d$. Two processes
$a$ and $c$ concurrently create a partition. They initialize their own
state with the lowest possible bound $0$ (see
Line~\ref{line:lowestbound}), and send a message to each of their
neighbors by accumulating the corresponding edge weight (see
Line~\ref{line:accumulator}). In Figure~\ref{fig:addC}, Process $b$
receives $\alpha_{c}^{1}$. Since it improves its own partition
distance, it keeps it and forwards it to its neighbors. Process $b$
discards $\alpha_{b}^{1.5}$, for it does not improve its partition
distance. Processes $c$ and $d$ will never hear of Process $a$'s
partition. At scale, such \emph{scoped flooding} saves bandwidth by
limiting generated traffic to subsets of
processes~\cite{wang2015prodiluvian}. In Figure~\ref{fig:addD},
processes discard last transiting messages. The system converged to
the optimal partitioning.  Transitive relationships ensure that each
process gets its closest source while accumulation of weights ensures
that messages propagation terminates.  Interestingly, while the
receipt order of messages does not impact on local control information
($r_b(\alpha_a^{1.5}) \cdot r_b(\alpha_a^{1}) \Leftrightarrow
r_b(\alpha_a^{1}) \cdot r_b(\alpha_a^{1.5}) \implies
\TODO{d_b(\alpha_a^{1})}$), it impacts on control information
broadcast in the network ($r_b(\alpha_a^{1.5}) \cdot r_b(\alpha_a^{1})
\implies b_b(\alpha_{b}^{1.5}) \cdot b_b(\alpha_{b}^{1})$ while
$r_b(\alpha_a^{1}) \cdot r_b(\alpha_a^{1.5}) \implies
b_b(\alpha_a^{1})$).

While only adding logical partitions to the distributed system is
straightforward, removing them introduces additional complexity. 

\begin{definition}[Dynamic optimal partitioning]
  A dynamic optimal partitioning is an optimal partitioning where the
  set of sources $S$ does not monotonically increase. We suffix this
  set with \underline{t}ime $t$ : $S(t)$.  Process $i$ joins the set
  of sources by \underline{a}dding its partition with messages
  $\alpha_i^{0}$: $Add(i \not\in S(t)) \implies i \in S(t+1)$; and
  leaves the set of sources by \underline{d}eleting its partition with
  messages $\delta_i$: $Del(i \in S(t)) \implies i \not\in S(t+1)$.
\end{definition}

Unfortunately, due to differences in receipt order of messages between
processes, stale information may stop prematurely the propagation of
relevant messages leading to inconsistent partitioning.
\TODO{Figure~\ref{fig:del} shows that.}

One could solve this issue by using a conflict-free replicated data
type (CRDT) for set data structures~\cite{shapiro2011crdts}. CRDTs
require eventual delivery of messages to ensure convergence. All
processes need to receive, deliver, hence broadcast all
operations. Some processes may receive operations unnecessary to their
correct functioning, for these operations did not happen in their
locality.  One could also solve this issue by removing partitions that
were not advertised for a defined
time~\cite{hemmati2015namebased}. \TODO{check distance-vector
  protocols.} However, relying on physical timeout could lead to
either premature removals of partitions when they are still operating;
or slow convergence where processes believe they belong to a partition
that was removed (\TODO{plot ?}). In addition, since timeouts imply a
continuous maintenance of partitions, such partitioning protocol
incurs an overhead even in quiet contexts without dynamic partitions.

\begin{problem}[\label{prob:intra}Consistent partitioning]
  How to enable \emph{consistent} dynamic optimal partitioning in
  dynamic network using scoped flooding?
\end{problem}

% \TODO{Dynamic Multiple-sources All-destinations shortest path on
%  dynamic graphs?}

\begin{figure}
  \begin{center}
    \begin{tikzpicture}[scale=1]

  \thickmuskip=0mu
  \medmuskip=0mu
  \thinmuskip=0mu
  
  \newcommand\X{50pt}
  \newcommand\Y{-50pt}

  \newcommand\ADD{\alpha}



  \draw (-1.5*\X, \Y + 5pt)
  -- node[above, font=\scriptsize, sloped]{$10ms$}
  (-1.5*\X, 0 - 5pt); %% a-e

  \draw (-1.5*\X + 5pt, 0)
  --
  (-1*\X, 0);

  \draw[dotted] (-1*\X, 0) --
  node[above, font=\scriptsize]{$300ms$} (-0.5*\X, 0);
  
  \draw (-0.5*\X, 0)
  --  
  (0 - 5pt, 0); %% a - b

  \draw (0 +5pt, 0)
  -- node[above, font=\scriptsize, sloped]{$15ms$}  
  (\X -5pt, 0); %% b - c
  
  \draw (0, \Y + 5pt)
  -- node[above, font=\scriptsize, sloped]{$20ms$}  
  (0, 0 - 5pt); %% b - d

  \draw (\X + 3pt, 0 - 5pt)
  -- node[above, font=\scriptsize, sloped]{$10ms$}  
  (0 + 5pt, \Y - 3pt); %% c - d



  \draw[dashed] (-0.75*\X, -0.5*\Y) -- (-0.75*\X, 1.5*\Y)
  node[above left]{\textit{(FR) Paris -- AS 1 \vphantom{g}}}
  node[above right]{\textit{AS 2 -- Wellington (NZ)}};

  \draw[fill=white] (-1.5*\X, 0) node[color=\PA]{$\bm{a}$} +(-5pt, -5pt) rectangle +(5pt, 5pt);
  \draw [left] (-1.5*\X - 5pt, 0) node{\textbf{e: 1}};
  \draw[fill=white] (0, 0) node[color=\PA]{$\bm{b}$} +(-5pt, -5pt) rectangle +(5pt, 5pt);
  \draw [above] ( 0, 5pt ) node{\textbf{e: 2}};
  \draw[fill=white] (\X, 0) node{$\bm{c}$} +(-5pt, -5pt) rectangle +(5pt, 5pt);
  \draw[fill=white] (0, \Y) node{$\bm{d}$} +(-5pt, -5pt) rectangle +(5pt, 5pt);

  \draw[fill=white] (-1.5*\X, \Y) node[color=\PA]{$\bm{e}$} +(-5pt, -5pt) rectangle +(5pt, 5pt);
  \draw [left] (-1.5*\X - 5pt, \Y) node{\textbf{e: 0}};
  
  
  % \draw (-\X, 5pt) node[above, color=\PA]{$\ADD_a^0$};
  % \draw (\X, 5pt) node[above, color=\PC]{$\ADD_c^0$};


\end{tikzpicture}

    \caption{\label{fig:ASmotivation}Scoped flooding extended to
      objects, processes do not know about all partitions that exist
      in the whole distributed system.}
  \end{center}
\end{figure}

To further improve scoped flooding in autonomous systems, processes
may belong to multiple partitions representing the objects they are
aware of. Figure~\ref{fig:ASmotivation} highlights the geo-distributed
nature of autonomous systems. Autonomous systems have topological
properties~\cite{nur2018geography} that processes can leverage to
avoid flooding the whole systems with control information about all
partitions. In Figure~\ref{fig:ASmotivation}, Process $e$ creates a
partition about a specific object, flooding Paris's AS to notify
neighboring processes of the object. However, it does not flood the
whole network, for control information stops at Process $b$. It
significantly saves bandwidth compared to broadcast since processes
below entrypoints of autonomous systems are order of magnitude more
numerous. Yet, Processes $c$ and $d$ know whom to ask when they need
to return an object they are unaware of. Processes remain efficient in
answering requests. \TODO{which is important for end users \REF.}

\begin{definition}[Partitions in asynchronous systems]
  Processes may hold multiple partitions \TODO{Use $\mathcal{P}$}
  \TODO{TODO} \TODO{or multi-partitioning?}
\end{definition}

\begin{problem}[\label{prob:inter}Inter-AS consistent partitioning]
  How to enable scoped flooding of objects in autonomous systems?
\end{problem}

Next Section presents our approach that solves these scientific
problems by relying on logical control information only.

%%% Local Variables: 
%%% mode: latex
%%% TeX-master: "../paper"
%%% ispell-local-dictionary: "english"
%%% End: 

